% (c) 2012 Claudio Carboncini - claudio.carboncini@gmail.com
% (c) 2012-2014 Dimitrios Vrettos - d.vrettos@gmail.com

\chapter{Riconoscimento di prodotti notevoli}

\section{Quadrato di un binomio}

Uno dei metodi più usati per la scomposizione di polinomi è legato al saper riconoscere i prodotti notevoli.
Se abbiamo un trinomio costituito da due termini che sono quadrati di due monomi ed il terzo termine è uguale al doppio prodotto
degli stessi due monomi, allora il trinomio può essere scritto sotto forma di quadrato di un binomio, secondo la regola che segue (sezione \ref{sect:quadrato_di_un_binomio} a pagina \pageref{sect:quadrato_di_un_binomio})
\begin{equation*}
(A+B)^{2}=A^{2}+2AB+B^{2}\quad \Rightarrow \quad A^{2}+2AB+B^{2}=(A+B)^{2}.
\end{equation*}
Analogamente nel caso in cui il monomio che costituisce il doppio prodotto sia negativo:
\begin{equation*}
(A-B)^{2}=A^{2}-2AB+B^{2}\quad \Rightarrow \quad A^{2}-2AB+B^{2}=(A-B)^{2}.
\end{equation*}
Poiché il quadrato di un numero è sempre positivo, valgono anche le seguenti uguaglianze
\begin{equation*}
(A+B)^{2}=(-A-B)^{2}\quad\Rightarrow\quad A^{2}+2AB+B^{2}=(A+B)^{2}=(-A-B)^{2}\phantom{.}
\end{equation*}
\begin{equation*}
(A-B)^{2}=(-A+B)^{2}\quad \Rightarrow \quad A^{2}-2AB+B^{2}=(A-B)^{2}=(-A+B)^{2}.
\end{equation*}

\begin{exrig}
 \begin{esempio}
Scomporre in fattori~$4a^{2}+12ab^{2}+9b^{4}$.

Notiamo che il primo ed il terzo termine sono quadrati, rispettivamente di~$2a$ e di~$3b^{2}$,
ed il secondo termine è il doppio prodotto degli stessi monomi, pertanto possiamo
scrivere:
\[4a^{2}+12{ab}^{2}+9b^{4}=(2a)^{2}+2\cdot (2a)\cdot (3b^{2})+\left(3b^{2}\right)^{2}=\left(2a+3b^{2}\right)^{2}.\]
 \end{esempio}

 \begin{esempio}
Scomporre in fattori~$x^{2}-6x+9$.

Il primo ed il terzo termine sono quadrati, il secondo termine compare con il segno ``meno''.
Dunque:~$x^{2}-6x+9=x^{2}-2\cdot 3\cdot x+3^{2}=(x-3)^{2}$, ma anche~$x^{2}-6x+9=(-x+3)^{2}$.
 \end{esempio}

 \begin{esempio}
Scomporre in fattori~$x^{4}+4x^{2}+4$.

Può accadere che tutti e tre i termini siano tutti quadrati. $x^{4}+4x^{2}+4$ è formato da tre quadrati, ma il secondo termine, quello di grado intermedio,
è anche il doppio prodotto dei due monomi di cui il primo ed il terzo termine sono i rispettivi quadrati.
Si ha dunque:
\[x^{4}+4x^{2}+4=\left(x^{2}\right)^{2}+2\cdot (2)\cdot (x^{2})+(2)^{2}=\left(x^{2}+2\right)^{2}.\]
 \end{esempio}
\end{exrig}

\begin{procedura}
Individuare il quadrato di un binomio:
\begin{enumeratea}
\item individuare le basi dei due quadrati;
\item verificare se il terzo termine è il doppio prodotto delle due basi;
\item scrivere tra parentesi le basi dei due quadrati e il quadrato fuori dalla parentesi;
\item mettere il segno ``più'' o ``meno'' in accordo al segno del termine che non è un quadrato.
\end{enumeratea}
\end{procedura}

Può capitare che i quadrati compaiano con il coefficiente negativo, ma si può rimediare mettendo in evidenza il segno ``meno''.

\begin{exrig}
 \begin{esempio}
Scomporre in fattori~$-9a^{2}+12{ab}-4b^{2}$.

Mettiamo~$-1$ a fattore comune~$-9a^{2}+12ab-4b^{2}=-(9a^{2}-12{ab}+4b^{2})=-(3a-2b)^{2}$.
 \end{esempio}

 \begin{esempio}
Scomporre in fattori~$-x^{4}-x^{2}-\frac{1}{4}$.
\[-x^{4}-x^{2}-\frac{1}{4}=-\left(x^{4}+x^{2}+\frac{1}{4}\right)=-\left(x^{2}+\frac{1}{2}\right)^{2}.\]
 \end{esempio}

 \begin{esempio}
Scomporre in fattori~$-x^{2}+6xy^{2}-9y^{4}$.
\[x^{2}+6xy^{2}-9y^{4}=-\left(x^{2}-6xy^{2}+9y^{4}\right)=-\left(x-3y^{2}\right)^{2}.\]
 \end{esempio}
\end{exrig}

Possiamo avere un trinomio che ``diventa'' quadrato di binomio dopo aver messo qualche fattore comune in evidenza.

\begin{exrig}
 \begin{esempio}
Scomporre in fattori~$2a^{3}+20a^{2}+50a$.

Mettiamo a fattore comune~$2a$, allora~$2a^{3}+20a^{2}+50a=2a(a^{2}+10a+25)=2a(a+5)^{2}$.
 \end{esempio}

 \begin{esempio}
Scomporre in fattori~$2a^{2}+4a+2$.
\[2a^{2}+4a+2=2\left(a^{2}+2a+1\right)=2(a+1)^{2}.\]
 \end{esempio}

 \begin{esempio}
Scomporre in fattori~$-12a^{3}+12a^{2}-3a$.
\[-12a^{3}+12a^{2}-3a=-3a\left(4a^{2}-4a+1\right)=-3a(2a-1)^{2}.\]
 \end{esempio}

 \begin{esempio}
Scomporre in fattori~$\frac{3}{8}a^{2}+3ab+6b^{2}$.

\[\frac{3}{8}a^{2}+3ab+6b^{2}=\frac{3}{2}\left(\frac{1}{4}a^{2}+2ab+4b^{2}\right)=\frac{3}{2}\left(\frac{1}{2}a+2b\right)^{2}\text{,}\]
o anche
\[\frac{3}{8}a^{2}+3ab+6b^{2}=\frac{3}{8}\left(a^{2}+8ab+16b^{2} \right)=\frac{3}{8}\left(a+4b\right)^{2}.\]
 \end{esempio}
\end{exrig}
\ovalbox{\risolvii \ref{ese:16.1}, \ref{ese:16.2}, \ref{ese:16.3}, \ref{ese:16.4}, \ref{ese:16.5}, \ref{ese:16.6}, \ref{ese:16.7}, \ref{ese:16.8}, \ref{ese:16.9}, \ref{ese:16.10},\ref{ese:16.11}, \ref{ese:16.12}}

\section{Quadrato di un polinomio}

Se siamo in presenza di sei termini, tre dei quali sono quadrati, verifichiamo se il polinomio è il quadrato di un
trinomio secondo le seguenti regole (sezione \ref{sect:quadrato_di_un_polinomio} a pagina \pageref{sect:quadrato_di_un_polinomio})
\begin{equation*}
(A+B+C)^{2}=A^{2}+B^{2}+C^{2}+2AB+2AC+2BC
\end{equation*}
\begin{equation*}
A^{2}+B^{2}+C^{2}+2AB+2AC+2BC=(A+B+C)^{2}=(-A-B-C)^{2}.
\end{equation*}
Notiamo che i doppi prodotti possono essere tutti e tre positivi, oppure uno positivo e due negativi:
indicano se i rispettivi monomi sono concordi o discordi.
%\newpage
\begin{exrig}
 \begin{esempio}
Scomporre in fattori~$16a^{4}+b^{2}+1+8a^{2}b+8a^{2}+2b$.

I primi tre termini sono quadrati rispettivamente di~$4a^{2}$, $b$ e~$1$ e si può verificare poi che gli altri tre termini sono
i doppi prodotti:~$16a^{4}+b^{2}+1+8a^{2}b+8a^{2}+2b=\left(4a^{2}+b+1\right)^{2}$.
 \end{esempio}

 \begin{esempio}
Scomporre in fattori~$x^{4}+y^{2}+z^{2}-2x^{2}y-2x^{2}z+2yz$.
\[x^{4}+y^{2}+z^{2}-2x^{2}y-2x^{2}z+2yz=\left(x^{2}-y-z\right)^{2}=\left(-x^{2}+y+z\right)^{2}.\]
 \end{esempio}

 \begin{esempio}
Scomporre in fattori~$x^{4}-2x^{3}+3x^{2}-2x+1$.

In alcuni casi anche un polinomio di cinque termini può essere il quadrato di un trinomio.
Per far venire fuori il quadrato del trinomio si può scindere il termine~$3x^{2}$ come somma:
\[3x^{2}=x^{2}+2x^{2}.\]
In questo modo si ha:
\[x^{4}-2x^{3}+3x^{2}-2x+1=x^{4}-2x^{3}+x^{2}+2x^{2}-2x+1=(x^{2}-x+1)^{2}.\]
 \end{esempio}
\end{exrig}

Nel caso di un quadrato di un polinomio la regola è sostanzialmente la stessa:
\begin{equation*}
(A+B+C+D)^{2}=A^{2}+B^{2}+C^{2}+D^{2}+2AB+2AC+2AD+2BC+2BD+2CD.
\end{equation*}
\ovalbox{\risolvii \ref{ese:16.13}, \ref{ese:16.14}, \ref{ese:16.15}, \ref{ese:16.16}, \ref{ese:16.17}, \ref{ese:16.18}, \ref{ese:16.19}}

\section{Cubo di un binomio}
I cubi di binomi sono di solito facilmente riconoscibili. Un quadrinomio è lo sviluppo del cubo di un binomio se due suoi termini sono i cubi
di due monomi e gli altri due termini sono i tripli prodotti tra uno dei due monomi ed il quadrato dell’altro, secondo le seguenti formule (sezione \ref{sect:cubo_di_un_binomio} a pagina \pageref{sect:cubo_di_un_binomio})
\begin{equation*}
(A+B)^{3}=A^{3}+3A^{2}B+3AB^{2}+B^{3}\quad \Rightarrow \quad A^{3}+3A^{2}B+3AB^{2}+B^{3}=(A+B)^{3}\phantom{.}
\end{equation*}
\begin{equation*}
(A-B)^{3}=A^{3}-3A^{2}B+3AB^{2}-B^{3}\quad \Rightarrow \quad A^{3}-3A^{2}B+3AB^{2}-B^{3}=(A-B)^{3}.
\end{equation*}
Per il cubo non si pone il problema, come per il quadrato, del segno della base, perché un numero elevato ad esponente dispari,
se è positivo rimane positivo e se è negativo rimane negativo.

\begin{exrig}
 \begin{esempio}
Scomporre in fattori~$8a^{3}+12a^{2}b+6{ab}^{2}+b^{3}$.

Notiamo che il primo ed il quarto termine sono cubi, rispettivamente di~$2a$ e di~$b$, il secondo termine è il triplo
prodotto tra il quadrato di~$2a$ e~$b$, mentre il terzo termine è il triplo prodotto tra~$2a$ e il quadrato di~$b$.
Abbiamo dunque:
\[8a^{3}+12a^{2}b+6ab^{2}+b^{3}=(2a)^{3}+3\cdot (2a)^{2}\cdot (b)+3\cdot (2a)\cdot (b)^{2}=(2a+b)^{3}.\]
 \end{esempio}

 \begin{esempio}
Scomporre in fattori~$-27x^{3}+27x^{2}-9x+1$.

Le basi del cubo sono il primo e il quarto termine, rispettivamente cubi di~$-3x$ e di~$1$.
Dunque:
\[-27x^{3}+27x^{2}-9x+1=(-3x)^{3}+3\cdot (-3x)^{2}\cdot 1+3\cdot (-3x)\cdot 1^{2}+1=(-3x+1)^{3}.\]
 \end{esempio}

 \begin{esempio}
Scomporre in fattori~$x^{6}-x^{4}+\frac{1}{3}x^{2}-\frac{1}{27}$.

Le basi del cubo sono~$x^{2}$ e~$-\frac{1}{3}$ i termini centrali sono i tripli prodotti,
quindi~$\left(x^{2}-\frac{1}{3}\right)^{3}$.
\end{esempio}
\end{exrig}
\ovalbox{\risolvii \ref{ese:16.20}, \ref{ese:16.21}, \ref{ese:16.22}, \ref{ese:16.23}, \ref{ese:16.24},\ref{ese:16.25}, \ref{ese:16.26}, \ref{ese:16.27}, \ref{ese:16.28}}

\section{Differenza di due quadrati}
Un binomio che sia la differenza dei quadrati di due monomi può essere scomposto come prodotto tra la somma dei due monomi
(basi dei quadrati) e la loro differenza (sezione \ref{sect:diffrenza_di_quadrati} a pagina \pageref{sect:diffrenza_di_quadrati})
\begin{equation*}
(A+B)\cdot (A-B)=A^{2}-B^{2}\quad \Rightarrow \quad A^{2}-B^{2}=(A+B)\cdot (A-B).
\end{equation*}

\begin{exrig}
 \begin{esempio}
Scomporre in fattori~$\frac{4}{9}a^{4}-25b^{2}$.
\[\frac{4}{9}a^{4}-25b^{2}=\left(\frac{2}{3}a^{2}\right)^{2}-\left(5b\right)^{2}=\left(\frac{2}{3}a^{2}+5b\right)
\cdot \left(\frac{2}{3}a^{{2}}-5b\right).\]
 \end{esempio}

 \begin{esempio}
Scomporre in fattori~$-x^{6}+16y^{2}$.
\[-x^{6}+16y^{2}=-\left(x^{3}\right)^{2}+\left(4y\right)^{2}=\left(x^{3}+4y\right)\cdot \left(-x^{3}+4y\right).\]
 \end{esempio}

 \begin{esempio}
Scomporre in fattori~$a^{2}-\left(x+1\right)^{2}$.
La formula precedente vale anche se~$A$ e~$B$ sono polinomi. Quindi $a^{2}-\left(x+1\right)^{2}=\left[a+(x+1)\right]\cdot \left[a-(x+1)\right]=(a+x+1)(a-x-1)$.
\end{esempio}

 \begin{esempio}
Scomporre in fattori~$\left(2a-b^{2}\right)^{2}-(4x)^{2}$.
\[\left(2a-b^{2}\right)^{2}-(4x)^{2}=\left(2a-b^{2}+4x\right)\cdot \left(2a-b^{2}-4x\right).\]
 \end{esempio}

 \begin{esempio}
Scomporre in fattori~$(a+3b)^{2}-(2x-5)^{2}$.
\[(a+3b)^{2}-(2x-5)^{2}=(a+3b+2x-5)\cdot (a+3b-2x+5).\]
 \end{esempio}
\end{exrig}

Per questo tipo di scomposizioni, la cosa più difficile è riuscire a riconoscere un quadrinomio o un polinomio
di sei termini come differenza di quadrati. Riportiamo i casi principali:
\begin{itemize*}
 \item $(A+B)^{2}-C^{2}=A^{{2}}+2AB+B^{2}-C^{2}$;
 \item $A^{2}-(B+C)^{2}=A^{2}-B^{2}-2BC-C^{2}$;
 \item $(A+B)^{2}-(C+D)^{2}=A^{2}+2AB+B^{2}-C^{2}-2CD-D^{2}$.
\end{itemize*}

\begin{exrig}
 \begin{esempio}
Scomporre in fattori~$4a^{2}-4b^{2}-c^{2}+4bc$.

Gli ultimi tre termini possono essere raggruppati per formare il quadrati di un binomio.
 \begin{equation*}
   \begin{split}
     4a^{2}-4b^{2}-c^{2}+4bc &=4a^{2}-\left(4b^{2}+c^{2}-4bc\right) \\
                 &= (2a)^{2}-(2b-c)^{2}=(2a+2b-c)\cdot (2a-2b+c).
   \end{split}
  \end{equation*}
 \end{esempio}

 \begin{esempio}
Scomporre in fattori~$4x^{4}-4x^{2}-y^{2}+1$.
\[4x^{4}-4x^{2}-y^{2}+1=\left(2x^{2}-1\right)^{2}-(y)^{2}=(2x^{2}-1+y)\cdot (2x^{2}-1-y).\]
 \end{esempio}

 \begin{esempio}
Scomporre in fattori~$a^{2}+1+2a+6bc-b^{2}-9c^{2}$.
 \begin{equation*}
   \begin{split}
     a^{2}+1+2a+6bc-b^{2}-9c^{2} &=\left(a^{2}+1+2a\right)-\left(b^{2}+9c^{2}-6{bc}\right) \\
                 &= (a+1)^{2}-(b-3c)^{2}=(a+1+b-3c)\cdot (a+1-b+3c).
   \end{split}
  \end{equation*}
 \end{esempio}
\end{exrig}
\ovalbox{\risolvii \ref{ese:16.29}, \ref{ese:16.30}, \ref{ese:16.31}, \ref{ese:16.32}, \ref{ese:16.33}, \ref{ese:16.34}, \ref{ese:16.35}, \ref{ese:16.36},%
\ref{ese:16.37}, \ref{ese:16.38}, \ref{ese:16.39}}

\vspazio\ovalbox{\ref{ese:16.40}, \ref{ese:16.41}}
\newpage
% (c) 2012 Silvia Cibola - silvia.cibola@gmail.com
% (c) 2012-2014 Dimitrios Vrettos - d.vrettos@gmail.com

Gli esercizi indicati con (\croce) sono tratti da \emph{Matematica~1}, Dipartimento di Matematica, ITIS V.~Volterra, San Donà di Piave, Versione [11-12][S-A11], pg.~90;
licenza CC,BY-NC-BD, per gentile concessione dei professori che hanno redatto il libro.
%Il libro è scaricabile da \url{http://www.istitutovolterra.it/dipartimenti/matematica/dipmath/docs/M1_1112.pdf}

\section{Esercizi}
\subsection{Problemi con i numeri}
\begin{multicols}{2}
\begin{esercizio}[\Ast]
\label{ese:16.1}
Determina quel numero che sottratto a~$185$ produce come risultato~$137$.
\end{esercizio}

\begin{esercizio}[\Ast]
\label{ese:16.2}
Determina due numeri, sapendo che la loro somma vale~$70$ e il secondo supera di~$16$ il doppio del primo.
\end{esercizio}

\begin{esercizio}[\Ast]
\label{ese:16.3}
Determina due numeri, sapendo che il secondo supera di~$17$ il triplo del primo e che la loro somma è~$101$.
\end{esercizio}

\begin{esercizio}[\Ast]
\label{ese:16.4}
Determinare due numeri dispari consecutivi sapendo che il minore supera di~$10$ i~$\frac{3}{7}$ del maggiore.
\end{esercizio}

\begin{esercizio}[\Ast]
\label{ese:16.5}
Sommando~$15$ al doppio di un numero si ottengono i~$\frac{7}{2}$ del numero stesso. Qual è il numero?
\end{esercizio}

\begin{esercizio}
\label{ese:16.6}
Determinare due numeri consecutivi sapendo che i~$\frac{4}{9}$ del maggiore superano di~$8$ i~$\frac{2}{13}$ del minore.
\end{esercizio}

\begin{esercizio}[\Ast]
\label{ese:16.7}
Se ad un numero sommiamo il suo doppio, il suo triplo, il suo quintuplo e sottraiamo~$21$, otteniamo~$100$. Qual è il numero?
\end{esercizio}

\begin{esercizio}[\Ast]
\label{ese:16.8}
Trova il prodotto tra due numeri, sapendo che: se al primo numero sottraiamo~$50$ otteniamo~$50$ meno il primo numero; se al doppio del secondo aggiungiamo il suo consecutivo, otteniamo~$151$.
\end{esercizio}

\begin{esercizio}[\Ast]
\label{ese:16.9}
Se a~$\frac{1}{25}$ sottraiamo un numero, otteniamo la quinta parte del numero stesso. Qual è questo numero?
\end{esercizio}

\begin{esercizio}[\Ast]
\label{ese:16.10}
Carlo ha~$152$ caramelle e vuole dividerle con le sue due sorelline. Quante caramelle resteranno a Carlo se le ha distribuite in modo che ogni sorellina ne abbia la metà delle sue?
\end{esercizio}

\begin{esercizio}[\Ast]
\label{ese:16.11}
Se a~$\frac{5}{2}$ sottraiamo un numero, otteniamo il numero stesso aumentato di~$\frac{2}{3}$. Di quale numero si tratta?
\end{esercizio}

\begin{esercizio}[\Ast]
\label{ese:16.12}
Determina il numero per il quale se ad esso si sottrae successivamente la sua metà, la sua quarta e la sua sesta parte si  ottiene~$7$.
\end{esercizio}

\begin{esercizio}[\Ast]
\label{ese:16.13}
Determina un numero sapendo che la differenza tra il suo quadruplo e~$27$ è~$93$.
\end{esercizio}

\begin{esercizio}[\Ast]
\label{ese:16.14}
La somma della metà più la terza parte più la quarta parte di un numero è~$104$. Qual è il numero?
\end{esercizio}

\begin{esercizio}[\Ast]
\label{ese:16.15}
Due numeri hanno per somma~$95$ e uno di loro è i due terzi dell'altro. Trovare i numeri.
\end{esercizio}

\begin{esercizio}[\Ast]
\label{ese:16.16}
Se ad un numero sottraiamo~$34$ e sommiamo~$75$, otteniamo~$200$. Qual è il numero?
\end{esercizio}

\begin{esercizio}[\Ast]
\label{ese:16.17}
Se alla terza parte di un numero sommiamo~$45$ e poi sottraiamo~$15$, otteniamo~$45$. Qual è il numero?
\end{esercizio}

\begin{esercizio}[\Ast]
\label{ese:16.18}
Se ad un numero sommiamo il doppio del suo consecutivo otteniamo~$77$. Qual è il numero?
\end{esercizio}

\begin{esercizio}[\Ast]
\label{ese:16.19}
Se alla terza parte di un numero sommiamo la sua metà, otteniamo il numero aumentato di~$2$. Qual è il numero?
\end{esercizio}

\begin{esercizio}[\Ast]
\label{ese:16.20}
Il doppio di un numero equivale alla metà del suo consecutivo più $1$. Qual è il numero?
\end{esercizio}

\begin{esercizio}[\Ast]
\label{ese:16.21}
%Un numero è uguale al suo consecutivo meno~$1$. Trova il numero.
Trova un numero che è uguale al suo consecutivo meno~$1$.
\end{esercizio}

\begin{esercizio}[\Ast]
\label{ese:16.22}
La somma tra un numero e il suo consecutivo è uguale al numero stesso aumentato di~$2$. Trova il numero.
\end{esercizio}

\begin{esercizio}[\Ast]
\label{ese:16.23}
La somma tra un numero ed il suo consecutivo aumentato di~$1$ è uguale a~$18$. Qual è il numero?
\end{esercizio}

\begin{esercizio}
\label{ese:16.24}
La somma tra un numero e lo stesso numero aumentato di~$3$ è uguale a~$17$. Qual è il numero?
\end{esercizio}

\begin{esercizio}[\Ast]
\label{ese:16.25}
La terza parte di un numero aumentata di~$3$ è uguale a~$27$. Trova il numero.
\end{esercizio}

\begin{esercizio}[\Ast]
\label{ese:16.26}
La somma tra due numeri~$x$ e~$y$ vale~$80$. Del numero~$x$ sappiamo che questo stesso numero aumentato della sua metà è uguale a~$108$.
\end{esercizio}

\begin{esercizio}[\Ast]
\label{ese:16.27}
Sappiamo che la somma fra tre numeri~$(x$, $y$, $z)$ è uguale a~$180$. Il numero~$x$ è uguale a se stesso diminuito di~$50$ e poi moltiplicato per~$6$. 
Il numero~$y$ aumentato di~$60$ è uguale a se stesso diminuito di~$40$ e poi moltiplicato per~$6$, trova~$x$, $y$, $z$.
\end{esercizio}

\begin{esercizio}[\Ast]
\label{ese:16.28}
La somma tra la terza parte di un numero e la sua quarta parte è uguale alla metà del numero aumentata di~$1$. Trova il numero.
\end{esercizio}

\begin{esercizio}
\label{ese:16.29}
Determina due numeri interi consecutivi tali che la differenza dei loro quadrati è uguale a~$49$.
\end{esercizio}

\begin{esercizio}
\label{ese:16.30}
Trova tre numeri dispari consecutivi tali che la loro somma sia uguale a~$87$.
\end{esercizio}

\begin{esercizio}
\label{ese:16.31}
Trova cinque numeri pari consecutivi tali che la loro somma sia uguale a~$1000$.
\end{esercizio}

\begin{esercizio}[\Ast]
\label{ese:16.32}
Determinare il numero naturale la cui metà, aumentata di~$20$, è uguale al triplo del numero stesso diminuito di~$95$.
\end{esercizio}

\begin{esercizio}[\Ast]
\label{ese:16.33}
Trova due numeri dispari consecutivi tali che la differenza dei loro cubi sua uguale a~$218$.
\end{esercizio}

\begin{esercizio}[\Ast]
\label{ese:16.34}
Trova un numero tale che se calcoliamo la differenza tra il quadrato del numero stesso e il quadrato del precedente otteniamo~$111$.
\end{esercizio}

\begin{esercizio}
\label{ese:16.35}
Qual è il numero che sommato alla sua metà è uguale a~$27$?
\end{esercizio}

\begin{esercizio}[\Ast]
\label{ese:16.36}
Moltiplicando un numero per~9 e sommando il risultato per la quarta parte del numero si ottiene~$74$. Qual è il numero?
\end{esercizio}

\begin{esercizio}
\label{ese:16.37}
La somma di due numeri pari e consecutivi è~$46$. Trova i due numeri.
\end{esercizio}

\begin{esercizio}[\Ast]
\label{ese:16.38}
La somma della metà di un numero con la sua quarta parte è uguale al numero stesso diminuito della sua quarta parte. Qual è il numero?
\end{esercizio}

\begin{esercizio}[\Ast]
\label{ese:16.39}
Di~$y$ sappiamo che il suo triplo è uguale al suo quadruplo diminuito di~$2$; trova~$y$.
\end{esercizio}

\begin{esercizio}
\label{ese:16.40}
Il numero~$z$ aumentato di~$60$ è uguale a se stesso diminuito di~$30$ e moltiplicato per~$4$.
\end{esercizio}

\begin{esercizio}[\Ast]
\label{ese:16.41}
Determinare un numero di tre cifre sapendo che la cifra delle centinaia è~$\frac{2}{3}$ di quella delle unità, la cifra delle decine è~$\frac{1}{3}$ delle unità e la somma delle tre cifre è~$12$.
\end{esercizio}

\begin{esercizio}[\Ast]
\label{ese:16.42}
Dividere il numero~$576$ in due parti tali che~$\frac{5}{6}$ della prima parte meno~$\frac{3}{4}$ della seconda parte sia uguale a~$138$.
\end{esercizio}

\begin{esercizio}[\Ast]
\label{ese:16.43}
Determina due numeri naturali consecutivi tali che la differenza dei loro quadrati è uguale a~$49$.
\end{esercizio}

\begin{esercizio}[\Ast]
\label{ese:16.44}
Una certa quantità, aumentata di~$180$ è uguale alla somma tra la sua metà e i suoi due terzi. Determina questa quantità.
\end{esercizio}

\begin{esercizio}[\Ast]
\label{ese:16.45}
Dividere il numero~$40$ in due parti tali che, aggiungendo~$10$ alla prima parte e togliendo~$10$ dalla seconda si ottiene lo stesso risultato.
\end{esercizio}
\end{multicols}

\pagebreak

\subsection{Problemi dalla realtà}
\begin{multicols}{2}
\begin{esercizio}[\Ast]
\label{ese:16.46}
Luca e Andrea posseggono rispettivamente \officialeuro~$200$ e \officialeuro~$180$; Luca spende \officialeuro~$10$ al giorno e Andrea \officialeuro~$8$ al giorno. Dopo quanti giorni avranno la stessa somma?
\end{esercizio}

\begin{esercizio}[\Ast]
\label{ese:16.47}
Ad un certo punto del campionato la Fiorentina ha il doppio dei punti della Juventus e l'Inter ha due terzi dei punti della Fiorentina. Sapendo che in totale i punti delle tre squadre sono~$78$, determinare i punti delle singole squadre.
\end{esercizio}

\begin{esercizio}[\Ast]
\label{ese:16.48}
Un vestito, una cravatta e un paio di scarpe costano complessivamente \officialeuro~$330$. Sapendo che le scarpe costano $8$~volte il prezzo della cravatta e il vestito $3$~volte quello delle scarpe, determinare il costo di ciascun oggetto.
\end{esercizio}

\begin{esercizio}[\Ast]
\label{ese:16.49}
Per organizzare una gita collettiva, vengono affittati due pulmini dello stesso modello, per i quali ciascun partecipante deve pagare \officialeuro~$12$. Sui pulmini restano, in tutto, quattro posti liberi. Se fossero stati occupati anche questi posti, ogni partecipante avrebbe risparmiato \officialeuro~$1,50$. Quanti posti vi sono su ogni pulmino? (``La settimana enigmistica'')
\end{esercizio}

\begin{esercizio}
\label{ese:16.50}
Un rubinetto, se aperto, riempie una vasca in~$5$ ore; un altro rubinetto riempie la stessa vasca in~$7$ ore. Se vengono aperti contemporaneamente, quanto tempo ci vorrà per riempire~$\frac{1}{6}$ della vasca?
\end{esercizio}

\begin{esercizio}[\Ast]
\label{ese:16.51}
L'età di Antonio è i~$\frac{3}{8}$ di quella della sua professoressa. Sapendo che tra~$16$ anni l'età della professoressa sarà doppia di quella di Antonio, quanti anni ha la professoressa?
\end{esercizio}

\begin{esercizio}[\Ast]
\label{ese:16.52}
Policrate, tiranno di Samos, domanda a Pitagora il numero dei suoi allievi. Pitagora risponde che: ``la metà studia le belle scienze matematiche; l'eterna Natura è oggetto dei lavori di un quarto; un settimo si esercita al silenzio e alla meditazione; vi sono inoltre tre donne''. Quanti allievi aveva Pitagora? (``Matematica dilettevole e curiosa'')
\end{esercizio}

\begin{esercizio}
\label{ese:16.53}
Trovare un numero di due cifre sapendo che la cifra delle decine è inferiore di~$3$ rispetto alla cifra delle unità e sapendo che invertendo l'ordine delle cifre e sottraendo il numero stesso, si ottiene~$27$. (``Algebra ricreativa'')
\end{esercizio}

\begin{esercizio}
\label{ese:16.54}
Al cinema ``Matematico'' hanno deciso di aumentare il biglietto del~$10\%$. Il numero degli spettatori è calato, però, del~$10\%$. È stato un affare?
\end{esercizio}

\begin{esercizio}
\label{ese:16.55}
A mezzogiorno le lancette dei minuti e delle ore sono sovrapposte. Quando saranno di nuovo sovrapposte?
\end{esercizio}

\begin{esercizio}
\label{ese:16.56}
Con due qualità di caffè da~$3$~\officialeuro/$\unit{kg}$ e~$5$~\officialeuro/$\unit{kg}$ si vuole ottenere un quintale di miscela da~$\np{3,25}$~\officialeuro/$\unit{kg}$. Quanti kg della prima e quanti della seconda qualità occorre prendere?
\end{esercizio}

\begin{esercizio}[\Ast]
\label{ese:16.57}
In un supermercato si vendono le uova in due diverse confezioni, che ne contengono rispettivamente~$10$ e~$12$. In un giorno è stato venduto un numero di contenitori da~$12$ uova doppio di quelli da~$10$, per un totale di~$544$ uova. Quanti contenitori da~$10$ uova sono stati venduti?
\end{esercizio}

\begin{esercizio}[\Ast]
\label{ese:16.58}
Ubaldo, per recarsi in palestra, passa sui mezzi di trasporto~$20$ minuti, tuttavia il tempo totale per completare il tragitto è maggiore a causa dei tempi di attesa. Sappiamo che Ubaldo utilizza~$3$ mezzi, impiega i~$\frac{3}{10}$ del tempo totale per l'autobus, i~$\frac{3}{5}$ del tempo totale per la metropolitana e~$10$ minuti per il treno. Quanti minuti è costretto ad aspettare i mezzi di trasporto? (\emph{poni x il tempo di attesa})
\end{esercizio}

\begin{esercizio}[\Ast]
\label{ese:16.59}
Anna pesa un terzo di Gina e Gina pesa la metà di Alfredo. Se la somma dei tre pesi è~$200\unit{kg}$, quanto pesa Anna?
\end{esercizio}

\begin{esercizio}
\label{ese:16.60}
In una partita a dama dopo i primi~$10$ minuti sulla scacchiera restano ancora~$18$ pedine. Dopo altri~$10$ minuti un giocatore perde~$4$ pedine nere e l'altro~$6$ pedine bianche ed entrambi rimangono con lo stesso numero di pedine. Calcolate quante pedine aveva ogni giocatore dopo i primi~$10$ minuti di gioco.
\end{esercizio}

\begin{esercizio}[\Ast]
\label{ese:16.61}
Due numeri naturali sono tali che la loro somma è~$16$ e il primo, aumentato di~$1$, è il doppio del secondo diminuito di~$3$. Trovare i due numeri.
\end{esercizio}

\begin{esercizio}
\label{ese:16.62}
Un dvd recoder ha due modalità di registrazione: SP e LP. Con la seconda modalità è possibile registrare il doppio rispetto alla modalità SP. Con un dvd dato per~$2$ ore in SP, come è possibile registrare un film della durata di~$3$ ore e un quarto? Se voglio registrare il più possibile in SP (di qualità migliore rispetto all'altra) quando devo necessariamente passare alla modalità LP?
\end{esercizio}

\begin{esercizio}[\Ast]
\label{ese:16.63}
Tizio si reca al casinò e gioca tutti i soldi che ha; dopo la prima giocata, perde la metà dei suoi soldi. Gli vengono prestati \officialeuro~$2$ e gioca ancora una volta tutti i suoi soldi; questa volta vince e i suoi averi vengono quadruplicati. Torna a casa con \officialeuro~$100$. Con quanti soldi era arrivato al casinò?
\end{esercizio}

\begin{esercizio}[\Ast]
\label{ese:16.64}
I sette nani mangiano in tutto~$127$ bignè; sapendo che il secondo ne ha mangiati il doppio del primo, il terzo il doppio del secondo e così via, quanti bignè ha mangiato ciascuno di loro?
\end{esercizio}

\begin{esercizio}[\Ast]
\label{ese:16.65}
Babbo Natale vuole mettere in fila le sue renne in modo tale che ogni fila abbia lo stesso numero di renne. Se le mette in fila per quattro le file sono due di meno rispetto al caso in cui le mette in fila per tre. Quante sono le renne?
\end{esercizio}

\begin{esercizio}[\Ast]
\label{ese:16.66}
Cinque fratelli si devono spartire un'eredità di \officialeuro~$\np{180000}$ in modo tale che ciascuno ottenga \officialeuro~$\np{8000}$ in più del fratello immediatamente minore. Quanto otterrà il fratello più piccolo?
\end{esercizio}

\begin{esercizio}[\Ast]
\label{ese:16.67}
Giovanni ha tre anni in più di Maria. Sette anni fa la somma delle loro età era~$19$. Quale età hanno attualmente?
\end{esercizio}

\begin{esercizio}[\Ast]
\label{ese:16.68}
Lucio ha acquistato un paio di jeans e una maglietta spendendo complessivamente \officialeuro~$518$. Calcolare il costo dei jeans e quello della maglietta, sapendo che i jeans costano \officialeuro~$88$ più della maglietta.
\end{esercizio}

\begin{esercizio}[\Ast]
\label{ese:16.69}
La somma di~\officialeuro~$\np{50000}$ è stata divisa fra 3 persone. La prima ha ricevuto \officialeuro~$\np{2500}$ in più della seconda che ne ha ricevuto \officialeuro~$\np{5000}$ in più della terza. Quanto ha ricevuto ogni persona?
\end{esercizio}

\begin{esercizio}[\Ast]
\label{ese:16.70}
Un'automobile supera su un'autostrada due autotreni, uno lungo~$15\unit{m}$ e l'altro lungo~$16\unit{m}$, distanziati tra di loro di~$74\unit{m}$. La manovra inizia $50\unit{m}$~prima e si conclude $85\unit{m}$~dopo. Sapendo che l'automobile viaggia a~$120\unit{km/h}$ e gli autotreni a~$60\unit{km/h}$, calcola quanti secondi occorrono all'automobile per completare la manovra di sorpasso e quanti metri ha percorso nel frattempo.
\end{esercizio}

\begin{esercizio}[\Ast]
\label{ese:16.71}
Francesca ha il triplo dell'età di Anna. Fra sette anni Francesca avrà il doppio dell'età di Anna. Quali sono le loro età attualmente?
\end{esercizio}

\begin{esercizio}[\Ast]
\label{ese:16.72}
In una fattoria ci sono tra polli e conigli~$40$ animali con~$126$ zampe. Quanti sono i conigli?
\end{esercizio}

\begin{esercizio}[\Ast]
\label{ese:16.73}
Due anni fa ho comprato un appartamento. Ho pagato alla consegna~$\frac{1}{3}$ del suo prezzo, dopo un anno~$\frac{3}{4}$ della rimanenza; oggi ho saldato il debito sborsando \officialeuro~$\np{40500}$. Qual è stato il prezzo dell'appartamento?
\end{esercizio}

\begin{esercizio}[\Ast]
\label{ese:16.74}
Un ciclista pedala in una direzione a~$30\unit{km/h}$, un marciatore parte a piedi dallo stesso punto e alla stessa ora e va nella direzione contraria a~$6\unit{km/h}$. Dopo quanto tempo saranno lontani~$150\unit{km}$?
\end{esercizio}

\begin{esercizio}[\Ast]
\label{ese:16.75}
Un banca mi offre il~$2\%$ di interesse su quanto depositato all'inizio dell'anno. Alla fine dell'anno vado a ritirare i soldi depositati più l'interesse: se ritiro \officialeuro~$\np{20400}$, quanto avevo depositato all'inizio? Quanto dovrebbe essere la percentuale di interesse per ricevere \officialeuro~$\np{21000}$ depositando i soldi calcolati al punto precedente?
\end{esercizio}

\begin{esercizio}[\Ast]
\label{ese:16.76}
Si devono distribuire \officialeuro~$\np{140800}$ fra~$11$ persone che hanno vinto un concorso. Alcune di esse rinunciano alla vincita e quindi la somma viene distribuita tra le persone rimanenti. Sapendo che ad ognuna di esse sono stati dati \officialeuro~$\np{4800}$ in più, quante sono le persone che hanno rinunciato al premio?
\end{esercizio}

\begin{esercizio}[\Ast]
\label{ese:16.77}
Un treno parte da una stazione e viaggia alla velocità costante di~$120\unit{km/h}$. Dopo~$80$ minuti parte un secondo treno dalla stessa stazione e nella stessa direzione alla velocità di~$150\unit{km/h}$. Dopo quanti~$\unit{km}$ il secondo raggiungerà il primo?
\end{esercizio}

\begin{esercizio}[\Ast]
\label{ese:16.78}
Un padre ha~$32$ anni, il figlio~$5$. Dopo quanti anni l'età del padre sarà~$10$ volte maggiore di quella del figlio? Si interpreti il risultato ottenuto.
\end{esercizio}

\begin{esercizio}[\Ast]
\label{ese:16.79}
Uno studente compra~$4$ penne, $12$ quaderni e~$7$ libri per un totale di \officialeuro~$180$. Sapendo che un libro costa quanto~$8$ penne e che~$16$ quaderni costano quanto~$5$ libri, determinare il costo dei singoli oggetti.
\end{esercizio}

\begin{esercizio}[\Ast]
\label{ese:16.80}
Un mercante va ad una fiera, riesce a raddoppiare il proprio capitale e vi spende \officialeuro~$500$; ad una seconda fiera triplica il suo avere e spende \officialeuro~$900$; ad una terza poi quadruplica il suo denaro e spende \officialeuro~$\np{1200}$. Dopo ciò gli sono rimasti \officialeuro~$800$. Quanto era all'inizio il suo capitale?
\end{esercizio}

\begin{esercizio}[\Ast]
\label{ese:16.81}
A un contadino fu chiesto, dal vigile del mercato, quante uova avesse da vendere quella mattina ed egli rispose: «se ne avessi $\frac{1}{2}$ più~$\frac{2}{3}$ più~$\frac{1}{4}$ di quante ne ho, ne avrei $180$ di più». Quante uova aveva?
\end{esercizio}

\begin{esercizio}[\Ast]
\label{ese:16.82}
A una contadina chiesero quanti buoi avesse ed ella rispose: «se ad~$\frac{1}{6}$ dei miei buoi ne aggiungete $9$, ottenete la metà dei buoi che possiedo meno uno». Trovare il numero dei buoi della contadina.
\end{esercizio}

\begin{esercizio}[\Ast]
\label{ese:16.83}
L'epitaffio di Diofanto. ``Viandante! Qui furono sepolti i resti di Diofanto. E i numeri possono mostrare, oh, miracolo! Quanto lunga fu la sua vita, la cui sesta parte costituì la sua felice infanzia. Aveva trascorso ormai la dodicesima parte della sua vita, quando di peli si coprì la guancia. E la settima parte della sua esistenza trascorse in un matrimonio senza figli. Passò ancora un quinquennio e gli fu fonte di gioia la nascita del suo primogenito, che donò il suo corpo, la sua bella esistenza alla terra, la quale durò solo la metà di quella del padre. Il quale, con profondo dolore discese nella sepoltura, essendo sopravvenuto solo quattro anni al proprio figlio. Dimmi quanti anni visse Diofanto.''
\end{esercizio}

\begin{esercizio}[\Ast, \croce]
\label{ese:16.84}
Un cane cresce ogni mese di~$\frac{1}{3}$ della sua altezza. Se dopo~$3$ mesi dalla nascita è alto~$64\unit{cm}$, quanto era alto appena nato?
\end{esercizio}

\begin{esercizio}[\Ast, \croce]
\label{ese:16.85}
La massa di una botte colma di vino è di~$192\unit{kg}$ mentre se la botte è riempita di vino per un terzo la sua massa è di~$74\unit{kg}$. Trovare la massa della botte vuota.
\end{esercizio}

\begin{esercizio}[\Ast, \croce]
\label{ese:16.86}
Carlo e Luigi percorrono in auto, a velocità costante, un percorso di~$400\unit{km}$, ma in senso opposto. Sapendo che partono alla stessa ora dagli estremi del percorso e che Carlo corre a~$120\unit{km/h}$ mentre Luigi viaggia a~$80\unit{km/h}$, calcolare dopo quanto tempo si incontrano.
\end{esercizio}

\begin{esercizio}[\Ast, \croce]
\label{ese:16.87}
Un fiorista ordina dei vasi di stelle di Natale che pensa di rivendere a \officialeuro~$12$ al vaso con un guadagno complessivo di \officialeuro~$320$. Le piantine però sono più piccole del previsto, per questo è costretto a rivendere ogni vaso a \officialeuro~$7$ rimettendoci complessivamente \officialeuro~$80$. Quanti sono i vasi comprati dal fiorista?
\end{esercizio}

\begin{esercizio}[\Ast]
\label{ese:16.88}
Disponendo di due termometri, uno in scala centigrada e l'altro in Fahrenheit, e sapendo che la relazione esistente tra le due misurazioni di temperatura è~$\frac{T_c}{T_F-32}=\frac{100}{180}$, calcolare per quale temperatura i due termometri segnano lo stesso valore numerico e per quale il Fahrenheit segna il doppio del centigrado.
\end{esercizio}

\begin{esercizio}[\Ast, \croce]
\label{ese:16.89}
Un contadino possiede~$25$ tra galline e conigli; determinare il loro numero sapendo che in tutto hanno~$70$ zampe.
\end{esercizio}

\begin{esercizio}[\Ast, \croce]
\label{ese:16.90}
Un commerciante di mele e pere carica nel suo autocarro~$139$ casse di frutta per un peso totale di~$\np{23,5}$ quintali. Sapendo che ogni cassa di pere e mele pesa rispettivamente~$20\unit{kg}$ e~$15\unit{kg}$, determinare il numero di casse per ogni tipo caricate.
\end{esercizio}

\begin{esercizio}[\Ast, \croce]
\label{ese:16.91}
Determina due numeri uno triplo dell'altro sapendo che dividendo il maggiore aumentato di~$60$ per l'altro diminuito di~$20$ si ottiene~$5$.
\end{esercizio}

\begin{esercizio}[\Ast, \croce]
\label{ese:16.92}
Un quinto di uno sciame di api si posa su una rosa, un terzo su una margherita. Tre volte la differenza dei due numeri vola sui fiori di pesco, e rimane una sola ape che si libra qua e là nell'aria. Quante sono le api dello sciame?
\end{esercizio}

\begin{esercizio}[\Ast, \croce]
\label{ese:16.93}
Per organizzare un viaggio di~$540$ persone un'agenzia si serve di~$12$ autobus, alcuni con~$40$ posti a sedere e altri con~$52$; quanti sono gli autobus di ciascun tipo?
\end{esercizio}

\begin{esercizio}[\Ast]
\label{ese:16.94}
Un padre e i suoi due figli hanno complessivamente $ 60 $ ~anni. Sapendo che uno dei figli ha un'età doppia rispetto a quella dell'altro e che il padre ha un'età pari ai~$ \frac{3}{2} $ della somma di quella dei figli determinare l'età di ciascuno.
\end{esercizio}

\begin{esercizio}[\Ast]
\label{ese:16.95}
Sopra due rami ci sono dei passeri: $ 49 $ sul più alto e~$ 27 $ sull'altro. Poco dopo sul ramo più alto c'è un numero di passeri triplo che sull'altro. Quanti passeri sono volati dal ramo inferiore a quello superiore?
\end{esercizio}

\begin{esercizio}[\Ast]
\label{ese:16.96}
Due treni partono contemporaneamente da due stazioni che distano $ 150\unit{Km} $ e si vanno incontro viaggiando, il primo a~$ 50\unit{Km/h} $ ed il secondo a~$ 75\unit{Km/h} $. Dopo quanto tempo e a che distanza dalle stazioni si incontrano?
\end{esercizio}

\begin{esercizio}[\Ast]
\label{ese:16.97}
in una scuola elementare i~$ \frac{2}{9} $ degli alunni frequentano la seconda classe, i~$ \frac{13}{72} $ la terza, $ \frac{1}{6} $ la quarta, $\frac{1}{8} $ la quinta e in~$22$ frequentano la prima classe. Quanti sono in tutto gli alunni? E quanti insegnanti vi sono, se il numero degli alunni diminuito di~$57$ è quintuplo del numero di insegnanti?
\end{esercizio}

\begin{esercizio}[\Ast]
\label{ese:16.98}
Per organizzare una festa si è deciso di chiedere un contributo di~\officialeuro~$\np{20,00}$ ad ogni ragazzo e~\officialeuro~$\np{10,00}$ ad ogni ragazza. Si sono incassati \officialeuro~$\np{580,00}$ e i ragazzi sono due in più delle ragazze. Quanti ragazzi e quante ragazze partecipano alla festa?
\end{esercizio}

\begin{esercizio}[\Ast]
\label{ese:16.99}
In un mazzo di~$56$ fiori vi sono garofani, rose e gigli. Il numero dei garofani è uguale alla somma del numero delle rose e dei gigli, il numero dei gigli è~$\frac{2}{5}$ del numero delle rose. Quante sono le rose, i gigli e i garofani?
\end{esercizio}

\begin{esercizio}[\Ast]
\label{ese:16.100}
Una persona ha a disposizione un capitale di~\officialeuro~$\np{9500,00}$, che impiega una parte al tasso di interesse dell'~$8\%$ e una parte all'~$11\%$. Al termine dell'anno ricava lo stesso interesse semplice dai due depositi. Quanto ha investito all'~$11\%$?
\end{esercizio}

\begin{esercizio}[\croce]
\label{ese:16.101}
Il papà di Paola ha venti volte l'età che lei avrà tra due anni e la mamma, cinque anni più giovane del marito, ha la metà dell'età che avrà quest'ultimo fra venticinque anni; dove si trova Paola oggi?
\end{esercizio}
\end{multicols}
\subsection{Problemi di geometria}
\begin{multicols}{2}
\begin{esercizio}[\Ast]
\label{ese:16.102}
Dividere il segmento~$AB$ di~$93\unit{cm}$ in cinque parti, in modo che ognuna sia il doppio della precedente.
\end{esercizio}

\begin{esercizio}[\Ast]
\label{ese:16.103}
Un segmento di~$43\unit{cm}$ è stato diviso in due parti tali che i~$\frac{3}{4}$ dell'una sono uguali ai~$\frac{7}{5}$ dell'altra. Trovare le due parti.
\end{esercizio}

\begin{esercizio}[\Ast]
\label{ese:16.104}
Calcolare la lunghezza di due segmenti sapendo che la loro somma è~$25\unit{cm}$ e uno di essi è il quadruplo dell'altro.
\end{esercizio}

\begin{esercizio}[\Ast]
\label{ese:16.105}
Calcolare la lunghezza di due segmenti sapendo che la loro differenza è~$18\unit{cm}$ e che uno è il triplo dell'altro.
\end{esercizio}

\begin{esercizio}[\Ast]
\label{ese:16.106}
Un triangolo isoscele ha il perimetro di~$104\unit{cm}$. Determinare le lunghezze dei dei suoi tre lati sapendo che la base è i~$\frac{3}{5}$ del lato obliquo.
\end{esercizio}

\begin{esercizio}[\Ast]
\label{ese:16.107}
Il perimetro di un triangolo è~$45,9\unit{cm}$. Trovare i tre lati, sapendo che uno è doppio dell'altro e il terzo è la semisomma degli altri due.
\end{esercizio}

\begin{esercizio}[\Ast]
\label{ese:16.108}
Calcolare la misura del lato di un quadrato sapendo che se si aumenta di~$4\unit{cm}$ il lato si ottiene un altro quadrato il cui perimetro è~$\frac{4}{3}$ di quello del quadrato di partenza.
\end{esercizio}

\begin{esercizio}[\Ast]
\label{ese:16.109}
In un rombo la somma delle diagonali misura~$140\unit{cm}$ e la diagonale minore è i~$\frac{2}{5}$ della maggiore. Determinare l'area del rombo.
\end{esercizio}

\begin{esercizio}[\Ast]
\label{ese:16.110}
In un trapezio isoscele una delle basi è~$\frac{3}{10}$ dell'altra, la loro differenza è~$30\unit{cm}$ e l'altezza è~$60\unit{cm}$. Calcola il perimetro del trapezio.
\end{esercizio}

\begin{esercizio}[\Ast]
\label{ese:16.111}
In un triangolo rettangolo uno degli angoli acuti è~$\frac{3}{7}$ dell'altro angolo acuto. Quanto misurano gli angoli del triangolo?
\end{esercizio}

\begin{esercizio}[\Ast]
\label{ese:16.112}
In un triangolo un angolo è il~$\frac{3}{4}$ del secondo angolo, il terzo angolo supera di~$10^{\circ}$ la somma degli altri due. Quanto misurano gli angoli?
\end{esercizio}

\begin{esercizio}
\label{ese:16.113}
In un triangolo~$ABC$, l'angolo in~$A$ è doppio dell'angolo in~$B$ e l'angolo in~$C$ è doppio dell'angolo in~$B$. Determina i tre angoli.
\end{esercizio}

\begin{esercizio}
\label{ese:16.114}
Un triangolo isoscele ha il perimetro di~$39$. Determina le lunghezze dei lati del triangolo sapendo che la base è~$\frac{3}{5}$ del lato.
\end{esercizio}

\begin{esercizio}[\Ast]
\label{ese:16.115}
Un triangolo isoscele ha il perimetro di~$122\unit{m}$, la base di~$24\unit{m}$. Quanto misura ciascuno dei due lati obliqui congruenti?
\end{esercizio}

\begin{esercizio}[\Ast]
\label{ese:16.116}
Un triangolo isoscele ha il perimetro di~$188\unit{cm}$, la somma dei due lati obliqui supera di~$25\unit{cm}$ i~$\frac{2}{3}$ della base. Calcola la lunghezza dei lati.
\end{esercizio}

\begin{esercizio}[\Ast]
\label{ese:16.117}
In un triangolo~$ABC$ di perimetro~$186\unit{cm}$ il lato~$AB$ è~$\frac{5}{7}$ di~$AC$ e~$BC$ è~$\frac{3}{7}$ di~$AC$. Quanto misurano i lati del triangolo?
\end{esercizio}

\begin{esercizio}[\Ast]
\label{ese:16.118}
Un trapezio rettangolo ha la base minore che è~$\frac{2}{5}$ della base maggiore e l'altezza è~$\frac{5}{4}$ della base minore. Sapendo che il perimetro è~$\np[m]{294,91}$, calcola l'area del trapezio. Poni~$AB=x$ e calcola~$BC$ con il teorema di Pitagora.
\end{esercizio}

\begin{esercizio}
\label{ese:16.119}
Determina l'area di un rettangolo che ha la base che è~$\frac{2}{3}$ dell'altezza, mentre il perimetro è~$144\unit{cm}$.
\end{esercizio}

\begin{esercizio}[\Ast]
\label{ese:16.120}
Un trapezio isoscele ha la base minore pari a~$\frac{7}{13}$ della base maggiore, il lato obliquo è pari ai~$\frac{5}{6}$ della differenza tra le due basi. Sapendo che il perimetro misura~$124\unit{cm}$, calcola l'area del trapezio.
\end{esercizio}

\begin{esercizio}[\Ast]
\label{ese:16.121}
Il rettangolo~$ABCD$ ha il perimetro di~$78\unit{cm}$, inoltre sussiste la seguente relazione tra i lati:~$\overline{AD}=\frac{8}{5}\overline{AB}+12\unit{cm}$. Calcola l'area del rettangolo.
\end{esercizio}

\begin{esercizio}[\Ast]
\label{ese:16.122}
Un rettangolo ha il perimetro che misura~$240\unit{cm}$, la base è tripla dell'altezza. Calcola l'area del rettangolo.
\end{esercizio}

\begin{esercizio}[\Ast]
\label{ese:16.123}
In un rettangolo l'altezza supera di~$3\unit{cm}$ i~$\frac{3}{4}$ della base, inoltre i~$\frac{3}{2}$ della base hanno la stessa misura dei~$\frac{2}{3}$ dell'altezza. Calcola le misure della base e dell'altezza.
\end{esercizio}

\begin{esercizio}[\Ast]
\label{ese:16.124}
In un triangolo isoscele la base è gli~$\frac{8}{5}$ del lato ed il perimetro misura~$108\unit{cm}$. Trovare l'area del triangolo e la misura dell'altezza relativa ad uno dei due lati obliqui.
\end{esercizio}

\begin{esercizio}[\Ast]
\label{ese:16.125}
In un rombo la differenza tra le due diagonali è di~$3\unit{cm}$. Sapendo che la diagonale maggiore è~$\frac{4}{3}$ della minore, calcolare il perimetro del rombo.
\end{esercizio}

\begin{esercizio}[\Ast]
\label{ese:16.126}
Determinare le misure delle dimensioni di un rettangolo, sapendo che la minore è uguale a~$\frac{1}{3}$ della maggiore e che la differenza tra il doppio della minore e la metà della maggiore è di~$10\unit{cm}$. Calcolare inoltre il lato del quadrato avente la stessa area del rettangolo dato.
\end{esercizio}

\begin{esercizio}[\Ast]
\label{ese:16.127}
Antonello e Gianluigi hanno avuto dal padre l'incarico di arare due campi, l'uno di forma quadrata e l'altro rettangolare. ``Io scelgo il campo quadrato -- dice Antonello, -- dato che il suo perimetro è di~$4$ metri inferiore a quello dell'altro''. ``Come vuoi! -- commenta il fratello -- Tanto, la superficie è la stessa, dato che la lunghezza di quello rettangolare è di~$18$ metri superiore alla larghezza''. Qual è l'estensione di ciascun campo?
\end{esercizio}

\begin{esercizio}[\Ast]
\label{ese:16.128}
In un trapezio rettangolo il lato obliquo e la base minore hanno la stessa lunghezza. La base maggiore supera di~$7\unit{cm}$ i~$\frac{4}{3}$ della base minore. Calcolare l'area del trapezio sapendo che la somma delle basi è~$42\unit{cm}$.
\end{esercizio}

\begin{esercizio}[\Ast]
\label{ese:16.129}
L'area di un trapezio isoscele è~$168\unit{cm^2}$, l'altezza è~$8\unit{cm}$, la base minore è~$\frac{5}{9}$ della maggiore. Calcolare le misure delle basi, del perimetro del trapezio e delle sue diagonali.
\end{esercizio}

\begin{esercizio}[\Ast]
\label{ese:16.130}
Le due dimensioni di un rettangolo differiscono di~$4\unit{cm}$. Trovare la loro misura sapendo che aumentandole entrambe di~$3\unit{cm}$ l'area del rettangolo aumenta di~$69\unit{cm^2}$.
\end{esercizio}

\begin{esercizio}[\Ast]
\label{ese:16.131}
In un quadrato~$ABCD$ il lato misura~$12\unit{cm}$. Detto~$M$ il punto medio del lato~$AB$, determinare sul lato opposto~$CD$ un punto~$N$ tale che l'area del trapezio~$AMND$ sia metà di quella del trapezio~$MBCN$.
\end{esercizio}

\begin{esercizio}[\Ast]
\label{ese:16.132}
Nel rombo~$ABCD$ la somma delle diagonali è~$20\unit{cm}$ ed il loro rapporto è~$\frac{2}{3}$. Determinare sulla diagonale maggiore~$AC$ un punto~$P$ tale che l'area del triangolo~$APD$ sia metà di quella del triangolo~$ABD$.
\end{esercizio}

\begin{esercizio}
\label{ese:16.133}
In un rettangolo~$ABCD$ si sa che~$\overline{AB}=91\unit{m}$ e~$\overline{BC}=27\unit{m}$; dal punto~$E$ del lato~$AB$, traccia la perpendicolare a~$DC$ e indica con~$F$ il punto d'intersezione con lo stesso lato. Determina la misura di~$AE$, sapendo che~$\Area(AEFD)=\frac{3}{4}\Area(EFCB)$.
\end{esercizio}
\end{multicols}
\pagebreak
\subsection{Risposte}
\begin{multicols}{2}
\paragraph{16.1.}
$48$.

\paragraph{16.2.}
$18$; $52$.

\paragraph{16.3.}
$21$; $80$.

\paragraph{16.4.}
$19$; $21$.

\paragraph{16.5.}
$10$.

\paragraph{16.7.}
$11$.

\paragraph{16.8.}
$\np{2500}$.

\paragraph{16.9.}
$\frac{1}{30}$.

\paragraph{16.10.}
$76$.

\paragraph{16.11.}
$\frac{11}{12}$.

\paragraph{16.12.}
$84$.

\paragraph{16.13.}
$30$.

\paragraph{16.14.}
$96$.

\paragraph{16.15.}
$57$; $38$.

\paragraph{16.16.}
$159$.

\paragraph{16.17.}
$45$.

\paragraph{16.18.}
$25$.

\paragraph{16.19.}
$-12$.

\paragraph{16.20.}
$1$.

\paragraph{16.21.}
Indeterminato.

\paragraph{16.22.}
$1$.

\paragraph{16.23.}
$8$.

\paragraph{16.25.}
$72$.

\paragraph{16.26.}
$72$; $8$.

\paragraph{16.27.}
$60$; $60$; $60$.

\paragraph{16.28.}
$12$.

\paragraph{16.32.}
$46$.

\paragraph{16.33.}
$5$; $7$.

\paragraph{16.34.}
$56$.

\paragraph{16.36.}
$8$.

\paragraph{16.38.}
Indeterminato.

\paragraph{16.39.}
$2$.

\paragraph{16.41.}
$426$.

\paragraph{16.42.}
$216$; $360$.

\paragraph{16.43.}
$24$; $25$.

\paragraph{16.44.}
$432$.

\paragraph{16.45.}
$10$; $30$.

\paragraph{16.46.}
$10$.

\paragraph{16.47.}
$36$; $24$; $18$.

\paragraph{16.48.}
\officialeuro~$240$;~\officialeuro~$10$; \officialeuro~$80$.

\paragraph{16.49.}
$16$.

\paragraph{16.51.}
$64$.

\paragraph{16.52.}
$28$.

\paragraph{16.57.}
$16$.

\paragraph{16.58.}
$80'$.

\paragraph{16.59.}
$20\;\unit{kg}$.

\paragraph{16.61.}
Impossibile.

\paragraph{16.63.}
\officialeuro~$46$.

\paragraph{16.64.}
$1\text{,~}2\text{,~}4\text{,~}6\text{,~}16\text{,~}\ldots$

\paragraph{16.65.}
$24$.

\paragraph{16.66.}
\officialeuro~$\np{20000}$.

\paragraph{16.67.}
$15$; $18$.

\paragraph{16.68.}
\officialeuro~$303$; \officialeuro~$215$.

\paragraph{16.69.}
\officialeuro~$\np{20000}$;~\officialeuro $\np{17500}$; \officialeuro~$\np{12500}$.

\paragraph{16.70.}
$14,45''$; $480\;\unit{m}$.

\paragraph{16.71.}
$7$; $21$.

\paragraph{16.72.}
$23$.

\paragraph{16.73.}
\officialeuro~$\np{243000}$.

\paragraph{16.74.}
$250'$.

\paragraph{16.75.}
\officialeuro~$\np{20000}$; $5\%$.

\paragraph{16.76.}
\officialeuro~$3$.

\paragraph{16.77.}
$800\;\unit{km}$.

\paragraph{16.78.}
$2$ anni fa.

\paragraph{16.79.}
\officialeuro~$2$ penna, \officialeuro~$16$ libro, \officialeuro~$5$ quaderno.

\paragraph{16.80.}
\officialeuro~$\np{483,33}$.

\paragraph{16.81.}
$432$.

\paragraph{16.82.}
$30$.

\paragraph{16.83.}
$84$.

\paragraph{16.84.}
$27\;\unit{cm}$.

\paragraph{16.85.}
$15\;\unit{kg}$.

\paragraph{16.86.}
$2$ ore.

\paragraph{16.87.}
$80$.

\paragraph{16.88.}
$-40^{\circ,f}$; $160^f$.

\paragraph{16.89.}
$15$ galline e~$10$ conigli.

\paragraph{16.90.}
$80$; $50$.

\paragraph{16.91.}
$240$; $80$.

\paragraph{16.92.}
$15$.

\paragraph{16.93.}
$7$ da~$40$ posti e~$5$ da~$52$.

\paragraph{16.94.}
$8$; $16$: $36$.

\paragraph{16.95.}
$8$.

\paragraph{16.96.}
$72''$; $60\;\unit{Km}$; $90\;\unit{Km}$.

\paragraph{16.97.}
$72$; $3$.

\paragraph{16.98.}
$20$; $18$.

\paragraph{16.99.}
$20$; $8$; $28$.

\paragraph{16.100.}
\officialeuro~$\np{4000}$.

\paragraph{16.102.}
$3\;\unit{cm}$; $6\;\unit{cm}$; $12\;\unit{cm}$; $24\;\unit{cm}$; $48\;\unit{cm}$.

\paragraph{16.103.}
$28\;\unit{cm}$; $15\;\unit{cm}$.

\paragraph{16.104.}
$5\;\unit{cm}$; $20\;\unit{cm}$.

\paragraph{16.105.}
$27\;\unit{cm}$; $9\;\unit{cm}$.

\paragraph{16.106.}
$24\;\unit{cm}$; $40\;\unit{cm}$; $40\;\unit{cm}$.

\paragraph{16.107.}
$\np[cm]{20,4}$; $\np[cm]{10,2}$: $\np[cm]{15,3}$.

\paragraph{16.108.}
$6\;\unit{cm}$.

\paragraph{16.109.}
$\np[cm^2]{2000}$.

\paragraph{16.110.}
$\np{179,42}\;\unit{cm^2}$.

\paragraph{16.111.}
$63^{\circ}$; $27^{\circ}$; $90^{\circ}$.

\paragraph{16.112.}
$\np{36,43}^{\circ}$; $\np{48,57}^{\circ}$; $95^{\circ}$.

\paragraph{16.115.}
$49\;\unit{m}$.

\paragraph{16.116.}
$\np[cm]{97,8}$; $\np[cm]{45,1}$; $\np[cm]{45,1}$.

\paragraph{16.117.}
$\np[cm]{32,82}$; $\np[cm]{45,95}$; $\np[cm]{107,22}$.

\paragraph{16.118.}
$\np[cm^2]{4235}$.

\paragraph{16.120.}
$\np[cm^2]{683,38}$.

\paragraph{16.121.}
$\np[cm^2]{297,16}$.

\paragraph{16.122.}
$\np[cm^2]{2700}$.

\paragraph{16.123.}
$2$; $\frac{9}{2}$.

\paragraph{16.124.}
$432\;\unit{cm^2}$; $\np[cm^2]{28,8}$.

\paragraph{16.125.}
$30\;\unit{cm}$.

\paragraph{16.126.}
$60\;\unit{cm}$; $20\;\unit{cm}$; $20\sqrt{3}\;\unit{cm}$.

\paragraph{16.127.}
$\np[m^2]{1600}$.

\paragraph{16.128.}
$189\;\unit{cm^2}$.

\paragraph{16.129.}
$27\;\unit{cm}$; $15\;\unit{cm}$; $62\;\unit{cm}$; $\np[cm]{22,47}$.

\paragraph{16.130.}
$12\;\unit{cm}$; $8\;\unit{cm}$.

\paragraph{16.131.}
$DN=2\;\unit{cm}$.

\paragraph{16.132.}
$AP=6\;\unit{cm}$.

\end{multicols}

\cleardoublepage
