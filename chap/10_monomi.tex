% (c) 2012-2014 Dimitrios Vrettos - d.vrettos@gmail.com
\chapter{Monomi}
\section{L'insieme dei monomi}

\begin{definizione}
Un'espressione letterale in cui numeri e lettere sono legati dalla sola moltiplicazione si chiama \emph{monomio}.
\end{definizione}

\begin{exrig}
 \begin{esempio}
L'espressione nelle due variabili~$a$ e~$b$, $E=5\cdot 2a^{2}\dfrac{3}{8}ab7b^{2}$
è un monomio perché numeri e lettere sono legate solo dalla moltiplicazione.
 \end{esempio}

 \begin{esempio}
L'espressione~$E=2a^{2}-ab^{2}$ non è un monomio poiché compare anche il segno di sottrazione.
 \end{esempio}
\end{exrig}

\ovalbox{\risolvi \ref{ese:10.1}}

\osservazione
Gli elementi di un monomio sono \emph{fattori}, perché sono termini
di una moltiplicazione ma possono comparire anche \emph{potenze},
infatti la potenza è una moltiplicazione di fattori uguali. Non
possono invece comparire esponenti negativi o frazionari. In un monomio
gli esponenti delle variabili devono essere numeri naturali.

\begin{definizione}
Un monomio si dice \emph{ridotto in forma normale} quando è scritto come prodotto di un solo fattore
numerico e di potenze letterali con basi diverse.
\end{definizione}

\begin{exrig}
 \begin{esempio}
Il monomio~$E=5\cdot 2a^{2}\dfrac{3}{8}ab7b^{2}$ non è scritto in
forma normale: tra i suoi fattori vi sono numeri diversi e le potenze
letterali hanno basi ripetute, la~$a$ e la~$b$ compaiono due volte ciascuna.

Moltiplichiamo tra loro i fattori numerici e otteniamo~$\dfrac{105}{4}$; eseguiamo il prodotto di potenze con la stessa base otteniamo
$a^{3}b^{3}$. Il monomio in forma normale è~$E=\dfrac{105}{4}a^{3}b^{3}$.
 \end{esempio}
\end{exrig}

\begin{procedura}
 Ridurre in forma normale un monomio:
 \begin{enumeratea}
 \item moltiplicare tra loro i fattori numerici;
 \item moltiplicare le potenze con la stessa base.
 \end{enumeratea}
\end{procedura}

\ovalbox{\risolvi \ref{ese:10.2}}

\begin{definizione}
La parte numerica del monomio ridotto a forma normale si chiama \emph{coefficiente} e
il complesso delle lettere ne costituisce la \emph{parte letterale}.
\end{definizione}

\begin{exrig}
 \begin{esempio}
 Nella tabella seguente sono segnati alcuni monomi con i rispettivi coefficienti e parti letterali.
\begin{center}
 \begin{tabular}{lcccc}
 \toprule
 monomio & $-{\dfrac{1}{2}}abc$ & $3x^{3}y^{5}$ & $a^{5}b^{7}$ & $-k^{2}$\\
coefficiente & $-{\dfrac{1}{2}}$ & $3$ & $1$ & $-1$ \\
parte letterale & $abc$ & $x^{3}y^{5}$ & $a^{5}b^{7}$ & $k^{2}$ \\
 \bottomrule
\end{tabular}
\end{center}
 \end{esempio}
\end{exrig}

\begin{definizione}
Se il coefficiente del monomio è zero il \emph{monomio} si dice \emph{nullo}.
\end{definizione}

\begin{exrig}
 \begin{esempio}
L'espressione letterale~$\frac{3}{5}a^{3}bc^{2}$ è un monomio;
il numero~$\frac{3}{5}$ e le lettere~$a^{3}$, $b$, $c^{2}$ sono legate
dall'operazione di moltiplicazione; il suo coefficiente è il numero~$\frac{3}{5}$ e la parte letterale è~$a^{3}bc^{2}$.
 \end{esempio}

 \begin{esempio}
 Controesempi:

 \begin{enumeratea}
 \item l'espressione letterale~$\frac{3}{5}a^{3}+bc^{2}$
 non è un monomio dal momento che numeri e lettere sono legati, oltre
che dalla moltiplicazione, anche dall'addizione;
\item l'espressione letterale~$\frac{3}{5}a^{-3}bc^{2}$
non è un monomio in quanto la potenza con esponente negativo
rappresenta una divisione, infatti~$a^{-3}=\frac{1}{a^{3}}$.
\end{enumeratea}
 \end{esempio}
\end{exrig}

\begin{definizione}
 Due o più monomi che hanno parte letterale identica si dicono \emph{simili}.
\end{definizione}

\begin{exrig}
 \begin{esempio}
Il monomio~$\frac{3}{5}a^{3}bc^{2}$ è simile
a~$68a^{3}bc^{2}$ e anche a~$-0,5a^{3}bc^{2}$, ma non è simile
a~$\frac{3}{5}a^{2}bc^{3}$. L'ultimo monomio ha le
stesse lettere degli altri ma sono elevate ad esponenti diversi.
 \end{esempio}
\end{exrig}

\osservazione Il monomio nullo si considera simile a qualunque altro monomio.

\begin{definizione}
Due monomi simili che hanno coefficiente opposto si dicono \emph{monomi opposti}.
\end{definizione}

%\begin{exrig}
 \begin{esempio}
I monomi~$\frac{3}{5}a^{3}bc^{2}$ e~$-{\frac{3}{5}}a^{3}bc^{2}$ sono opposti, infatti sono simili e hanno
coefficienti opposti.
\end{esempio}

\begin{esempio}
Non sono opposti~$\frac{3}{5}a^{3}bc^{2}$ e~$-7a^{3}bc^{2}$ ma semplicemente simili. I loro coefficienti hanno segno diverso, ma
non sono numeri opposti.
\end{esempio}
%\end{exrig}

%\ovalbox{\risolvi \ref{ese:10.3}}

\begin{definizione}
Quando il monomio è ridotto a forma normale, l'esponente di una sua variabile ci indica il
\emph{grado} del monomio \emph{rispetto a quella variabile}.

Il \emph{grado complessivo} di un monomio è la somma degli esponenti della parte letterale.
\end{definizione}

\begin{exrig}
 \begin{esempio}
Il monomio~$\frac{3}{5}a^{3}bc^{2}$ ha grado complessivo~6,
ottenuto sommando gli esponenti della sua parte letterale~$(3+1+2=6)$.
Rispetto alla variabile~$a$ è di terzo grado, rispetto alla
variabile~$b$ è di primo grado, rispetto alla variabile
$c$ è di secondo grado.
 \end{esempio}
\end{exrig}

Abbiamo detto che gli esponenti della parte letterale del monomio sono
numeri naturali, dunque possiamo anche avere una o più variabili
elevate ad esponente~0. Cosa succede allora nel monomio?

Consideriamo il monomio~$56a^{3}b^{0}c^{2}$, sappiamo che qualunque
numero diverso da zero elevato a zero è uguale a~1, quindi possiamo
sostituire la variabile~$b$ che ha esponente~0 con~1 e
otteniamo~$56a^{3}\cdot 1\cdot c^{2}=56a^{3}c^{2}$. Se in un monomio ogni
variabile ha esponente~0, il monomio rimane solamente con il suo
coefficiente numerico: per esempio~$-3a^{0}x^{0}=-3\cdot 1\cdot 1=-3$.

\osservazione Esistono \emph{monomi di grado~0}; essi presentano solo il
coefficiente e pertanto \emph{sono} equiparabili ai \emph{numeri razionali}.

\section{Valore di un monomio}

Poiché il monomio è un'espressione letterale,
possiamo calcolarne il valore quando alle sue variabili sostituiamo dei numeri.

\begin{exrig}
 \begin{esempio}
 Calcola il valore del monomio~$3x^{4}y^{5}z$ per i valori~$x=-3$, $y=5$ e~$z=0$.

Sostituendo i valori assegnati otteniamo~$3\cdot (-3)^{4}\cdot 5^{5}\cdot 0=0$ essendo uno dei fattori nullo.
 \end{esempio}
\end{exrig}

\osservazione Il valore di un monomio è nullo quando almeno una delle sue variabili
assume il valore~0.

Molte formule di geometria sono scritte sotto forma di monomi: area del
triangolo~$\frac{1}{2}bh$, area del quadrato~$l^{2}$,
perimetro del quadrato~$4l$, area del rettangolo~$bh$, volume del cubo~$l^{3}$, ecc.
Esse acquistano un valore quando alle lettere sostituiamo
i numeri che rappresentano le misure della figura considerata.

\ovalbox{\risolvii \ref{ese:10.3}, \ref{ese:10.4}, \ref{ese:10.5}, \ref{ese:10.6}, \ref{ese:10.7}, \ref{ese:10.8}, \ref{ese:10.9}, \ref{ese:10.10}, \ref{ese:10.11}}

\section{Moltiplicazione di due monomi}

Ci proponiamo ora di introdurre nell'insieme dei monomi
le operazioni di addizione, sottrazione, moltiplicazione, potenza e
divisione.

Ricordiamo che definire un'operazione all'interno di un insieme
significa stabilire una legge che associa a due elementi
dell'insieme un altro elemento
dell'insieme stesso.

La moltiplicazione di due monomi si indica con lo stesso simbolo della
moltiplicazione tra numeri; i suoi termini si chiamano \emph{fattori} e il
risultato si chiama \emph{prodotto}, proprio come negli insiemi numerici.

\begin{definizione}
 Il prodotto di due monomi è il monomio avente
per coefficiente il prodotto dei coefficienti e per parte letterale il
prodotto delle parti letterali dei monomi fattori.
\end{definizione}

\begin{exrig}
 \begin{esempio}
Assegnati i monomi~$m_{1}=-4x^{2}yz^{3}$ e~$m_{2}=\dfrac{5}{6}x^{3}z^{6}$,
il monomio prodotto è
\[m_{3}=\bigg(-4\cdot {\frac{5}{6}}\bigg)\big(x^{2}\cdot x^{3}\big)\cdot y\cdot \big(z^{3}\cdot z^{6}\big)=-\frac{10}{3}x^{5}yz^{9}.\]
 \end{esempio}
\end{exrig}


\begin{procedura}[per moltiplicare due monomi]
La moltiplicazione tra monomi si effettua moltiplicando prima i
coefficienti numerici e dopo le parti letterali:

\begin{enumeratea}
 \item nella moltiplicazione tra i coefficienti usiamo le regole note della
moltiplicazione tra numeri razionali;
 \item nella moltiplicazione tra le parti letterali applichiamo la regola
del prodotto di potenze con la stessa base.
\end{enumeratea}
\end{procedura}

\subsection{Proprietà della moltiplicazione}

\begin{enumeratea}
\item commutativa:~$m_{{1}}\cdot m_{2}=m_{2}\cdot m_{{1}}$;
\item associativa:~$m_{{1}}\cdot m_{2}\cdot m_{3}=(m_{{1}}\cdot m_{2})\cdot m_{3}=m_{{1}}\cdot (m_{2}\cdot m_{3})$;
\item 1 è l'elemento neutro:~$1\cdot m=m\cdot 1=m$;
\item se uno dei fattori è uguale a~0 il prodotto è~0, cioè~$0\cdot m=m\cdot 0=0$.
\end{enumeratea}

\ovalbox{\risolvii \ref{ese:10.12}, \ref{ese:10.13}, \ref{ese:10.14}, \ref{ese:10.15}}

\section{Potenza di un monomio}

Ricordiamo che tra i numeri l'operazione di elevamento a
potenza ha un solo termine, la base, sulla quale si agisce a seconda
dell'esponente.

\[\text{Potenza }=\text{ base }^\text{ esponente}= \underbrace{(\text{ base }\cdot \text{ base }\cdot\text{ base }\cdot\ldots\cdot \text{ base })}_{\text{tanti fattori quanti ne indica l'esponente}}.\]

Analogamente viene indicata la potenza di un monomio: la base è un
monomio e l'esponente è un numero naturale.

\begin{definizione}
La \emph{potenza di un monomio} è un monomio
avente per coefficiente la potenza del coefficiente e per parte
letterale la potenza della parte letterale.
\end{definizione}

\begin{exrig}
 \begin{esempio}
Calcoliamo il quadrato e il cubo del monomio~$m_{1}=-{\dfrac{1}{2}}a^{2}b$.
\[\text{elevo al quadrato}\quad\Rightarrow\quad\bigg(-{\frac{1}{2}}a^{2}b\bigg)^{2}
=\bigg(-{\frac{1}{2}}\bigg)^{2}\cdot\big(a^{2}\big)^{2}\cdot (b)^{2}=\frac{1}{4}a^{4}b^{2}.\]

\[\text{elevo al cubo}\quad\Rightarrow\quad\bigg(-{\frac{1}{2}}a^{2}b\bigg)^{3}
=\bigg(-{\frac{1}{2}}\bigg)^{3}\cdot\big(a^{2}\big)^{3}\cdot (b)^{3}
=-{\frac{1}{8}}a^{6}b^{3}.\]
 \end{esempio}

 \begin{esempio}
Calcoliamo il quadrato e il cubo del monomio~$m_{2}=5a^{3}b^{2}c^{2}$.
\[\text{elevo al quadrato}\quad\Rightarrow\quad\big(5a^{3}b^{2}c^{2}\big)^{2}
=\big(5\big)^{2}\cdot \big(a^{3}\big)^{2}\cdot\big(b^{2}\big)^{2}\cdot \big(c^{2}\big)^{2}
=25a^{6}b^{4}c^{4}.\]

\[\text{elevo al cubo}\quad\Rightarrow\quad\big(5a^{3}b^{2}c^{2}\big)^{3}
=\big(5\big)^{3}\cdot \big(a^{3}\big)^{3}\cdot\big(b^{2}\big)^{3}\cdot \big(c^{2}\big)^{3}
=125a^{9}b^{6}c^{6}.\]
 \end{esempio}
\end{exrig}

\begin{procedura}
Eseguire l'elevazione a potenza di un monomio:

\begin{enumeratea}
 \item applichiamo la proprietà relativa alla potenza di un prodotto,
eseguiamo cioè la potenza di ogni singolo fattore del monomio;
 \item applichiamo la proprietà relativa alla potenza di potenza,
moltiplicando l'esponente della variabile per l'esponente delle potenza.
\end{enumeratea}
\end{procedura}

\ovalbox{\risolvii \ref{ese:10.16}, \ref{ese:10.17}, \ref{ese:10.18}, \ref{ese:10.19}}

\section{Divisione di due monomi}

Premessa: ricordiamo che assegnati due numeri razionali~$d_{1}$
e~$d_{2}$ con~$d_{2}\neq~0$, eseguire la
divisione~$d_{1}:d_{2}$ significa determinare il numero~$q$
che moltiplicato per~$d_{2}$ dà~$d_{1}$.
Nell'insieme~$\insQ$ la condizione~$d_{2}\neq~0$ è sufficiente per
affermare che~$q$ esiste ed è un numero razionale.

\begin{definizione}
Assegnati due monomi~$m_{1}$ e~$m_{2}$ con~$m_{2}$ diverso dal monomio nullo, se
è possibile determinare il monomio~$q$ tale che~$m_{1} = q\cdot m_{2}$, si dice che~$m_{1}$ è
divisibile per~$m_{2}$ e~$q$ è il monomio \emph{quoziente}.
\end{definizione}

\begin{exrig}
 \begin{esempio}
$(36x^{5}y^{2}):(-18x^{3}y)$.

Per quanto detto sopra, vogliamo trovare, se esiste, il monomio~$q$ tale
che~$(36x^{5}y^{2})=q\cdot (-18x^{3}y)$
e ripensando alla moltiplicazione di monomi possiamo dire
che~$q=-2x^{2}y$. Infatti~$(-2x^{2}y)\cdot(-18x^{3}y)=(36x^{5}y^{2})$. Il monomio~$-2x^{2}y$
è quindi il quoziente della divisione assegnata.
 \end{esempio}
\end{exrig}

\begin{procedura}[Calcolare il quoziente di due monomi]
Il quoziente di due monomi è così composto:
\begin{enumeratea}
 \item il coefficiente è il quoziente dei coefficienti dei monomi dati;
 \item la parte letterale ha gli esponenti ottenuti sottraendo gli esponenti
delle stesse variabili;
 \item se la potenza di alcune lettere risulta negativa il risultato della
divisione non è un monomio.
\end{enumeratea}
\end{procedura}
\pagebreak
\begin{exrig}
 \begin{esempio}
$\bigg(\dfrac{7}{2}a^{3}x^{{4}}y^{2}\bigg):\bigg(-{\dfrac{21}{8}}ax^{2}y\bigg)$.

Seguiamo i passi descritti sopra
\[\bigg(\frac{7}{2}a^{3}x^{{4}}y^{2}\bigg):\bigg(-{\frac{21}{8}}ax^{2}y\bigg)=\frac{7}{2}\cdot%
\bigg(-{\frac{8}{21}}\bigg)a^{3-1}x^{4-2}y^{2-1}=-{\frac{4}{3}}a^{2}x^{2}y.\]

Nell'eseguire la divisione non abbiamo tenuto conto
della condizione che il divisore deve essere diverso dal monomio nullo;
questa condizione ci obbliga a stabilire per la divisione le Condizioni
di Esistenza,~$\CE:a\neq0\text{ e }x\neq~0\text{ e }y\neq~0$.
 \end{esempio}

 \begin{esempio}
$\bigg(\dfrac{9}{20}a^{2}b^{4}\bigg):\bigg(-{\dfrac{1}{8}}a^{5}b^{2}\bigg)$.

La~$\CE a\neq~0\text{ e }b\neq~0$, il quoziente è
\[\bigg(\frac{9}{20}a^{2}b^{4}\bigg):\bigg(-{\frac{1}{8}}a^{5}b^{2}\bigg)=%
\bigg(\frac{9}{20}\bigg)\cdot(-8)a^{2-5}b^{4-2}=-{\frac{18}{5}}a^{-3}b^{2}.\]

Osserviamo che il quoziente ottenuto non è un monomio perché
l'esponente della variabile~$a$ è negativo. Il
risultato è un'espressione frazionaria o fratta.
 \end{esempio}
\end{exrig}

In conclusione, l'operazione di divisione tra due monomi
ha come risultato un monomio se ogni variabile del dividendo ha
esponente maggiore o uguale all'esponente con cui
compare nel divisore.


\vspazio\ovalbox{\risolvii \ref{ese:10.20}, \ref{ese:10.21}, \ref{ese:10.22}}

\section{Addizione di due monomi}

L'addizione di due monomi si indica con lo stesso
simbolo dell'addizione tra numeri; i suoi termini si
chiamano \emph{addendi} e il risultato si chiama \emph{somma}.

\subsection{Addizione di due monomi simili}

La somma di due monomi simili è un monomio simile agli addendi e
avente come coefficiente la somma dei coefficienti.

\begin{exrig}
 \begin{esempio}
Calcoliamo~$3x^{3}+(-6x^{3})$.

I due addendi sono monomi simili dunque la somma è ancora un monomio
ed è simile ai singoli addendi. Precisamente
$3x^{3}+(-6x^{3})=(3+(-6))x^{3}=-3x^{3}$.

Osserva che la somma di monomi simili si riduce alla somma algebrica di numeri.
 \end{esempio}
\end{exrig}

\ovalbox{\risolvi \ref{ese:10.23}}

\subsubsection{Proprietà dell'addizione}

\begin{enumeratea}
 \item commutativa:~$m_{{1}}+m_{2}=m_{2}+m_{{1}}$;
 \item associativa:~$m_{{1}}+m_{2}+m_{3}=(m_{{1}}+m_{2})+m_{3}=m_{{1}}+(m_{2}+m_{3})$;
 \item 0 è l'elemento neutro:~$0+m=m+0=m$;
 \item per ogni monomio m esiste il monomio \emph{opposto}, cioè un
 monomio~$m\Ast$ tale che
 \[m + m\Ast = m\Ast +m=0.\]
\end{enumeratea}

L'ultima proprietà enunciata ci permette di definire
nell'insieme dei monomi simili anche la sottrazione di
monomi. Essa si indica con lo stesso segno della sottrazione tra numeri
e il suo risultato si chiama \emph{differenza}.

\osservazione Per sottrarre due monomi simili si aggiunge al primo
l'opposto del secondo.

\begin{exrig}
 \begin{esempio}
Assegnati~$m_{{1}}=\dfrac{1}{2}a^{2}b$, $m_{2}=-\text{5a}^{2}b$ determina~$m_{1} - m_{2}$.

L'operazione richiesta
$\dfrac{1}{2}a^{2}b-(-5a^{2}b)$ diventa
$\dfrac{1}{2}a^{2}b+5a^{2}b=\dfrac{11}{2}a^{2}b$.
 \end{esempio}
\end{exrig}

Sulla base di quanto detto, possiamo unificare le due operazioni di
addizione e sottrazione di monomi simili in un'unica
operazione che chiamiamo \emph{somma algebrica di monomi}.

\osservazione La somma algebrica di due monomi simili è un monomio simile agli
addendi avente per coefficiente la somma algebrica dei coefficienti.

\begin{exrig}
 \begin{esempio}
Determiniamo la somma~$\dfrac{3}{5}x^{{4}}-\dfrac{1}{3}x^{{4}}+x^{{4}}+\dfrac{4}{5}x^{{4}}-2x^{{4}}-\dfrac{1}{2}x^{{4}}$.

Osserviamo che tutti gli addendi sono tra loro simili dunque:
\[\frac{3}{5}x^{{4}}-\frac{1}{3}x^{{4}}+x^{{4}}+\frac{4}{5}x^{{4}}-2x^{{4}}-\frac{1}{2}x^{{4}}=\left(\frac{3}{5}-\frac{1}{3}+1+\frac{4}{5}-2-\frac{1}{2}\right)x^{{4}}=-{\frac{13}{30}}x^{{4}}.\]
\end{esempio}
\end{exrig}
%\newpage
\subsection{Addizione di monomi non simili}

Analizziamo il caso della seguente
addizione:~$7a^{3}b^{2}-5a^{2}b^{3}+a^{3}b^{2}$. Si vuole determinare
la somma. I monomi addendi non sono tutti tra loro simili; lo sono
però il primo e il terzo.

Le proprietà associativa e commutativa ci consentono di riscrivere
l'addizione precedente
``avvicinando'' i monomi simili e
sostituendo ad essi la loro
somma:
\[7a^{3}b^{2}-5a^{2}b^{3}+a^{3}b^{2}=(7a^{3}b^{2}+a^{3}b^{2})-5a^{2}b^{3}=8a^{3}b^{2}-5a^{2}b^{3}.\]

L'espressione così ottenuta è la somma richiesta.

%\vspazio\ovalbox{\risolvi \ref{ese:10.24}}\vspazio

Il procedimento che abbiamo seguito per determinare il risultato
dell'addizione assegnata viene chiamato
\emph{riduzione dei termini simili}.

In conclusione, l'operazione di addizione tra monomi ha
come risultato un monomio solo se gli addendi sono monomi simili; in
caso contrario la somma viene effettuata riducendo i monomi simili e
lasciando indicata l'addizione tra gli altri monomi.
\pagebreak
\begin{exrig}
 \begin{esempio}
Calcola la seguente somma:~$3a-7a+2a+a$.

Il risultato è un monomio poiché gli addendi sono monomi
simili, e vale $-a$.
 \end{esempio}

 \begin{esempio}
Calcola la seguente somma:
$\dfrac{1}{2}a^{3}+b-\dfrac{3}{4}a^{3}-\dfrac{6}{5}b$.

Il risultato non è un monomio poiché gli addendi non sono
monomi simili: $-{\dfrac{1}{4}}a^{3}-\dfrac{1}{5}b$.
 \end{esempio}
\end{exrig}

\ovalbox{\risolvii \ref{ese:10.24}, \ref{ese:10.25}, \ref{ese:10.26}, \ref{ese:10.27}, \ref{ese:10.28}, \ref{ese:10.29}, \ref{ese:10.30}, \ref{ese:10.31}}

\section{Espressioni con i monomi}
Consideriamo l'espressione letterale
$E=\left(-{\frac{1}{2}}a^{2}b\right)^{3}:(a^{5}b)+(-2ab)\cdot\left(\frac{1}{2}b+b\right)+5ab^{2}$.

Vediamo che è in due variabili, le variabili sono infatti~$a$ e~$b$. Inoltre, i termini delle operazioni che vi compaiono sono monomi.

Se volessimo calcolare il valore di~$E$ per~$a = 10$ e $b = -2$ dovremmo
sostituire nell'espressione tali valori e risolvere
l'espressione numerica che ne risulta. Inoltre se
dovessimo calcolare il valore di~$E$ per altre coppie dovremmo ogni volta
applicare questo procedimento.

Dal momento che abbiamo studiato come eseguire le operazioni razionali
con i monomi, prima di sostituire i numeri alle lettere, applichiamo le
regole del calcolo letterale in modo da ridurre~$E$, se possibile,
in un'espressione più semplice.

Prima di procedere, essendovi una divisione, poniamo innanzi tutto
la~$\CE a \neq~0$ e~$b \neq~0$ ed eseguiamo rispettando la precedenza
delle operazioni come facciamo nelle espressioni numeriche.
%\newpage
\begin{exrig}
 \begin{esempio}
Calcola $\bigg(-{\dfrac{1}{2}}a^{2}b\bigg)^{3}:(a^{5}b)+(-2ab)\cdot\bigg(\dfrac{1}{2}b+b\bigg)+5ab^{2}$ per $a=10$ e $b=-2$.
 \begin{align*}
 &\text{sviluppiamo per prima il cubo} && = \bigg(-{\frac{1}{8}}a^{6}b^{3}:a^{5}b\bigg)+(-2ab)\cdot{\frac{3}{2}}b+5ab^{2} \\
 &\text{eseguiamo divisione e moltiplicazione} && = -{\frac{1}{8}}ab^{2}-3ab^{2}+5ab^{2}\\
 &\text{sommiamo i monomi simili} && = \frac{15}{8}ab^{2}.
 \end{align*}
 Ora è più semplice calcolarne il valore per~$a=10$ e~$b=-2$: si
 ha~$=\frac{15}{8}\cdot 10\cdot(-2)^{2}=\frac{15}{8}\cdot 10\cdot 4=75$.
 \end{esempio}

 \begin{esempio}
Riduci l'espressione $\bigg(\dfrac{2}{3}ab^{2}c\bigg)^{2}:\big(-3ab^{3}\big)-\dfrac{2}{9}abc^{2}$.
 \begin{align*}
 &\text{Sviluppiamo le potenze} && = \frac{4}{9}a^{2}b^{4}c^{2}:\big(-3ab^{3}\big)-\frac{2}{9}abc^{2}\\
 &\text{eseguiamo la divisione e moltiplichiamo le frazioni} && = -{\frac{4}{27}}abc^{2}-\frac{2}{9}abc^{2}\\
 &\text{sommiamo i monomi simili} && = \frac{-4-6}{27}abc^{2}\\
 &\text{il risultato è} && = -{\frac{10}{27}}abc^{2}.
 \end{align*}
 \end{esempio}

 \begin{esempio}
Riduci l'espressione $\Bigg[\bigg(-{\dfrac{14}{16}}x^{2}y^{2}\bigg):\bigg(-{\dfrac{14}{4}}xy\bigg)\Bigg]^{3}+\dfrac{1}{2}xy\cdot{\dfrac{1}{4}}x^{2}y^{2}$.
 \begin{align*}
 &\text{Eseguiamo la divisione tra le parentesi quadre} && = \bigg[+{\frac{14}{16}}\cdot{\frac{4}{14}}xy\bigg]^{3}+\frac{1}{2}xy\cdot {\frac{1}{4}}x^{2}y^{2}\\
 &\text{eseguiamo la moltiplicazione tra le frazioni} && = \bigg[\frac{1}{4}xy\bigg]^{3}+\frac{1}{2}xy\cdot{\frac{1}{4}}x^{2}y^{2}\\
 &\text{sviluppiamo il cubo} && = \frac{1}{64}x^{3}y^{3}+\frac{1}{2}xy\cdot {\frac{1}{4}}x^{2}y^{2}\\
 &\text{moltiplichiamo i due monomi} && = \frac{1}{64}x^{3}y^{3}+\frac{1}{8}x^{3}y^{3}\\
 &\text{sommiamo i monomi simili} && = \frac{1+8}{64}x^{3}y^{3}\\
 &\text{il risultato è} && = \frac{9}{64}x^{3}y^{3}.
\end{align*}
 \end{esempio}
\end{exrig}

\ovalbox{\risolvii \ref{ese:10.32}, \ref{ese:10.33}, \ref{ese:10.34}, \ref{ese:10.35}, \ref{ese:10.36}, \ref{ese:10.37}, \ref{ese:10.38}, \ref{ese:10.39}, \ref{ese:10.40}, \ref{ese:10.41},}

\vspazio\ovalbox{\ref{ese:10.42}, \ref{ese:10.43}}

\section{Massimo Comune Divisore e minimo comune multiplo tra monomi}
\subsection{Massimo Comune Divisore}

Il calcolo del minimo comune multiplo e del Massimo Comune Divisore,
studiato per i numeri, si estende anche ai monomi. Premettiamo intanto
le seguenti definizioni.

\begin{definizione}
 Un monomio~$A$ si dice \emph{multiplo} di un monomio~$B$ se esiste un
monomio~$C$ per il quale si ha~$A=B\cdot C$; in questo caso diremo anche che~$B$
è \emph{divisore} del monomio~$A$.
\end{definizione}

\begin{definizione}
 Il \emph{massimo comune divisore} ($\mcd$) tra due o più monomi è il
monomio che, tra tutti i divisori comuni dei monomi dati, ha grado
massimo.
\end{definizione}

Il coefficiente numerico può essere un qualunque numero reale: se i
coefficienti sono tutti interi è opportuno scegliere il loro~$\mcd$,
se non sono interi è opportuno scegliere~1.

\begin{exrig}
 \begin{esempio}
Dati i monomi~$12a^{3}b^{2}$ e~$16a^{2}b$ sono divisori
comuni:
\[1\text{,~}2\text{,~}4\text{,~}a\text{,~}a^{2}\text{,~}b\text{,~}ab\text{,~}a^{2}b\text{,~}2a\text{,~}2a^{2}\text{,~}2b\text{,~}2ab\text{,~}2a^{2}b\text{,~}4a\text{,~}4a^{2}\text{,~}4b\text{,~}4ab\text{,~}4a^{2}b.\]

Il monomio di grado massimo è~$a^{2}b$, il~$\mcd$ tra i coefficienti
è~4. Pertanto il~$\mcd$ dei monomi è~$4a^{2}b$.
 \end{esempio}
\end{exrig}

\begin{procedura}[Calcolare il~$\mcd$ tra monomi]

Il~$\mcd$ di un gruppo di monomi è il monomio che ha:

\begin{enumeratea}
 \item per coefficiente numerico il~$\mcd$ dei valori assoluti dei
coefficienti dei monomi qualora
questi siano numeri interi, se non sono interi si prende~1;
 \item la parte letterale formata da tutte le lettere comuni ai monomi
dati, ciascuna presa una sola volta e con l'esponente minore con cui compare.
\end{enumeratea}
\end{procedura}

\begin{exrig}
 \begin{esempio}
Calcolare il~$\mcd (14a^{3}b^{4}c^{2}\text{,~}4ab^{2}\text{,~}8a^{2}b^{3}c)$.

Per prima cosa calcoliamo il~$\mcd$ tra i coefficienti numerici~14, 4 e
8 che è~2. Per ottenere la parte letterale si mettono insieme tutte
le lettere comuni, ciascuna con l'esponente minore con
cui compare:~$ab^{2}$.

In definitiva, quindi, il~$\mcd (14a^{3}b^{4}c^{2}\text{,~}4ab^{2}\text{,~}8a^{2}b^{3}c)=2ab^{2}$.
 \end{esempio}

 \begin{esempio}
Calcolare il massimo comune divisore tra~$5x^{3}y^{2}z^{3}$, $-\frac{1}{8}xy^{2}z^{2}$ e $7x^{3}yz^{2}$.

Si osservi che i coefficienti numerici dei monomi non sono numeri interi
quindi si prende~1 come coefficiente del~$\mcd$.
Le lettere in comune sono~$x$, $y$ e $z$. Prese ciascuna con
l'esponente minore con cui compaiono si ha~$xyz^{2}$.

Quindi, il~$\mcd (5x^{3}y^{2}z^{3}\text{,~}-\frac{1}{8}xy^{2}z^{2}\text{,~}7x^{3}yz^{2})=xyz^{2}$.
 \end{esempio}
\end{exrig}

\osservazione La scelta di porre uguale a~1 il coefficiente numerico del~$\mcd$, nel
caso in cui i monomi abbiano coefficienti razionali, è dovuta al
fatto che una qualsiasi frazione divide tutte le altre e quindi una
qualsiasi frazione potrebbe essere il coefficiente del~$\mcd$. Ad essere
più precisi, occorrerebbe, quando si parla di monomi e polinomi,
chiarire a quale degli insiemi numerici~$\insN$, $\insZ$, $\insQ$ e~$\insR$ appartengono i loro
coefficienti. Qui stiamo considerando coefficienti numerici in~$\insR$.

\begin{definizione}
 Due monomi si dicono \emph{monomi primi tra loro} se il loro~$\mcd$ è~1.
\end{definizione}


\subsection{Minimo comune multiplo}

Estendiamo ora ai monomi la nozione di minimo comune multiplo.

\begin{definizione}
 Il \emph{minimo comune multiplo} ($\mcm$) di due o più monomi
è il monomio che, tra tutti i monomi multipli comuni dei monomi dati,
ha il grado minore.
\end{definizione}

Il coefficiente numerico può essere un qualunque numero reale: se i
coefficienti sono tutti interi è opportuno scegliere il loro~$\mcm$,
se non lo sono è opportuno scegliere~1.

%\begin{exrig}
 \begin{esempio}
Per calcolare il minimo comune multiplo tra~$5a^{3}b$ e~$10a^{2}b^{2}$ dovremmo costruire i loro multipli finché non
incontriamo quello comune che ha coefficiente numerico positivo più
piccolo e grado minore:

%\[\text{ alcuni multipli di }5a^{3}b\text{ sono: }10a^{3}b\text{,~}10a^{3}b^{2}\text{,~}10a^{4}b\text{,~}15a^{3}b\text{,~}\ldots\]
%\[\text{ alcuni multipli di }10a^{2}b^{2}\text{ sono: }10a^{2}b^{3}\text{,~}10a^{3}b^{2}\text{,~}10a^{4}b^{2}\text{,~}20a^{2}b^{2}\text{,~}\ldots\]

%Il minimo comune multiplo è~$10a^{3}b^{2}$.

\begin{equation*}
\begin{split}
& \text{ alcuni multipli di }5a^{3}b\text{ sono: } 10a^{3}b\text{,~}10a^{3}b^{2}\text{,~}10a^{4}b\text{,~}15a^{3}b\text{,~}\ldots \\
& \text{ alcuni multipli di }10a^{2}b^{2}\text{ sono: } 10a^{2}b^{3}\text{,~}10a^{3}b^{2}\text{,~}10a^{4}b^{2}\text{,~}20a^{2}b^{2}\text{,~}\ldots \\
& \text{Il minimo comune multiplo è }10a^{3}b^{2}.
\end{split}
\end{equation*}
 \end{esempio}
%\end{exrig}

In realtà, applicando la definizione è poco pratico calcolare il~$\mcm$, è utile invece la seguente procedura.

\begin{procedura}[Calcolo del~$\mcm$ tra due o più monomi]
Il~$\mcm$ di un gruppo di monomi è il monomio che ha:

\begin{enumeratea}
 \item per coefficiente numerico il~$\mcm$ dei valori assoluti dei
coefficienti dei monomi qualora
questi siano numeri interi, se non sono interi si prende~1;
 \item la parte letterale formata da tutte le lettere comuni e non comuni
ai monomi dati, ciascuna
presa una sola volta e con l'esponente maggiore con
cui compare.
\end{enumeratea}
\end{procedura}

\begin{exrig}
 \begin{esempio}
Calcola il minimo comune multiplo tra
$5a^{3}bc$, $12ab^{2}c$ e~$10a^{3}bc^{2}$.

Il~$\mcm$ tra i coefficienti~5, 12, 10 è~60. Per ottenere la parte
letterale osservo il grado maggiore delle lettere componenti i
monomi, riporto tutte le lettere, comuni e non comuni, una sola volta
con il grado maggiore con cui ciascuna compare:~$a^{3}b^{2}c^{2}$.

In definitiva, il
$\mcm(5a^{3}bc\text{,~}12ab^{2}c\text{,~}10a^{3}bc^{2})=60a^{3}b^{2}c^{2}$.
 \end{esempio}

 \begin{esempio}
Calcola il~$\mcm(6x^{2}y\text{,~}-\frac{1}{2}xy^{2}z\text{,~}\frac{2}{3}x^{3}yz)$.

I coefficienti numerici dei monomi non sono interi, quindi il~$\mcm$
avrà come coefficiente~1.

La parte letterale si costruisce mettendo insieme tutte le lettere che
compaiono ($x$, $y$, $z$), ciascuna presa con
l'esponente massimo, quindi~$x^{3}y^{2}z$.

In definitiva
$\mcm(6x^{2}y\text{,~}-\frac{1}{2}xy^{2}z\text{,~}\frac{2}{3}x^{3}yz)=x^{3}y^{2}z$.
 \end{esempio}
\end{exrig}

Assegnati due monomi, per esempio~$x^{2}y$ e~$xy^{2}z$,
calcoliamo~$\mcd$ e $\mcm$.
\begin{itemize*}
\item $\mcd(x^{2}y\text{,~}xy^{2}z)=xy$;
\item $\mcm(x^{2}y\text{,~}xy^{2}z)=x^{2}y^{2}z$.
\end{itemize*}
Moltiplichiamo ora~$\mcd$ e~$\mcm$. Abbiamo:~$xy\cdot x^{2}y^{2}z= x^{3}y^{3}z.$
Moltiplichiamo ora i monomi assegnati. Abbiamo:~$(x^{2}y)\cdot (xy^{2}z)=x^{3}y^{3}z.$
Il prodotto dei due monomi è uguale al prodotto tra il~$\mcd$ e
il~$\mcm$. Si può dimostrare che questa proprietà vale in generale.

\begin{proprieta}
 Dati due monomi, il prodotto tra il loro massimo comune
divisore e il loro minimo comune multiplo è uguale al prodotto tra i
monomi stessi.
\end{proprieta}

\ovalbox{\risolvii \ref{ese:10.44}, \ref{ese:10.45}, \ref{ese:10.46}, \ref{ese:10.47}, \ref{ese:10.48}, \ref{ese:10.49}, \ref{ese:10.50}}

\newpage
% (c) 2012-2014 Dimitrios Vrettos - d.vrettos@gmail.com
% (c) 2012, 2014 Claudio Carboncini - claudio.carboncini@gmail.com
% (c) 2012 Silvia Cibola - silvia.cibola@gmail.com
\section{Esercizi}
\subsection{Esercizi dei singoli paragrafi}
\subsubsection*{10.1 - Definizioni fondamentali}
\begin{multicols}{2}
\begin{esercizio}
\label{ese:10.1}
Riduci in forma normale il seguente polinomio:
\[5a^3-4ab-1+2a^3+2ab-a-3a^3.\]
\emph{Svolgimento}: Evidenziamo i termini simili e sommiamoli tra di loro:
%\[\underline{5a^3}-\overline{4ab}+1+\underline{2a^3}+\overline{2ab}-a-\underline{3a^3}\]
\[\mmevid{ev_rosso}{5a^{3}}-\mmevid{ev_verde}{4ab}+1+\mmevid{ev_rosso}{2a^{3}}+\mmevid{ev_verde}{2ab}-a-\mmevid{ev_rosso}{3a^{3}}\]
così otteniamo \dotfill Il termine noto è \dotfill
\end{esercizio}

\begin{esercizio}
\label{ese:10.2}
Il grado di:
\begin{enumeratea}
\item $x^2y^2−3y^3+5yx−6y^2x^3$ rispetto alla lettera~$y$ è \dotfill, il grado complessivo è \dotfill
\item $5a^2−b+4ab$ rispetto alla~$b$ è \dotfill,\\ il grado complessivo è \dotfill
\end{enumeratea}
\end{esercizio}


\begin{esercizio}
\label{ese:10.3}
Stabilire quali dei seguenti polinomi sono omogenei:

\begin{enumeratea}
\item $x^3y+2y^2x^2−4x^4$;
\item $2x+3−xy$;
\item $2x^3y^3−y^4x^2+5x^6$.
\end{enumeratea}
\end{esercizio}

\begin{esercizio}
\label{ese:10.4}
Individuare quali dei seguenti polinomi sono ordinati rispetto alla lettera~$x$ con potenze crescenti:

\begin{enumeratea}
\item $2-\dfrac{1}{2}x^2+x$;
\item $\dfrac{2}{3}-x+3x^2+5x^3$;
\item $3x^4-\dfrac{1}{2}x^3+2x^2-x+\dfrac{7}{8}$.
\end{enumeratea}
\end{esercizio}

\begin{esercizio}
\label{ese:10.5}
Relativamente al polinomio~$b^2+a^4+a^3+a^2$:
\begin{itemize*}
\item Il grado massimo è \ldots. Il grado rispetto alla lettera~$a$ è \ldots. Rispetto alla lettera~$b$ è \ldots
\item il polinomio è ordinato rispetto alla $a$? %\tab\qquad\boxV\qquad\boxF
\item è completo? %\tab\qquad\boxV\qquad\boxF
\item è omogeneo? %\tab\qquad\boxV\qquad\boxF
\end{itemize*}
\end{esercizio}

\begin{esercizio}
\label{ese:10.6}
Scrivere un polinomio di terzo grado nelle variabili~$a$ e~$b$ che sia omogeneo.
\end{esercizio}

\begin{esercizio}
\label{ese:10.7}
Scrivere un polinomio di quarto grado nelle variabili~$x$ e~$y$ che sia omogeneo e ordinato secondo le
potenze decrescenti della seconda indeterminata.
\end{esercizio}

\begin{esercizio}
\label{ese:10.8}
Scrivere un polinomio di quinto grado nelle variabili~$r$ e~$s$ che sia omogeneo e ordinato secondo le
potenze crescenti della prima indeterminata.
\end{esercizio}

\begin{esercizio}
\label{ese:10.9}
Scrivere un polinomio di quarto grado nelle variabili~$z$ e~$w$ che sia omogeneo e ordinato secondo le
potenze crescenti della prima indeterminata e decrescenti della seconda.
\end{esercizio}

\begin{esercizio}
\label{ese:10.10}
Scrivere un polinomio di sesto grado nelle variabili~$x$, $y$ e~$z$ che sia completo e ordinato secondo le
potenze decrescenti della seconda variabile.
\end{esercizio}


\begin{esercizio}
\label{ese:10.11}
Calcola il valore numerico dei polinomi per i valori a fianco indicati.

\begin{enumeratea}
\item $x^2+x$ per $x=-1$;
\item $2x^2-3x+1$ per $x=0$;
\item $3x^2-2x-1$ per $x=2$;
\item $3x^3-2x+x$ per $x=-2$;
\item $\dfrac{3}{4}a+\dfrac{1}{2}b-\dfrac{1}{6}ab$ per $a=-\dfrac{1}{2}$, $b=3$;
\item $4x-6y+\dfrac{1}{5}x^2$ per $x=-5$, $y=\dfrac{1}{2}$.
\end{enumeratea}
\end{esercizio}
\end{multicols}


\subsubsection*{10.2 - Somma algebrica di polinomi}
\begin{esercizio}
\label{ese:10.12}
Calcolare la somma dei due polinomi:~$2x^2+5−3y^2x$, $x^2−xy+2−y^2x+y^3$.

\emph{Svolgimento}: Indichiamo la somma~$(2x^2+5−3y^2x)+(x^2−xy+2−y^2x+y^3)$, eliminando le parentesi otteniamo
il polinomio~$2x^2+5−3y^2x+x^2−xy+2−y^2x+y^3$, sommando i monomi simili otteniamo~$3x^2−4x^{\ldots}y^{\ldots}-\ldots xy+y^3+\ldots$
\end{esercizio}
%\newpage
\begin{esercizio}
\label{ese:10.13}
 Esegui le seguenti somme di polinomi.
 \begin{enumeratea}
 \item $a+b-b$;
 \item $a+b-2b$;
 \item $a+b-(-2b)$;
 \item $a-(b-2b)$;
 \item $2a+b+(3a+b)$;
 \item $2a+2b+(2a+b)+2a$;
 \item $2a+b-(-3a-b)$;
 \item $2a-3b-(-3b-2a)$;
 \item $(a+1)-(a-3)$.
\end{enumeratea}
\end{esercizio}


\begin{esercizio}[\Ast]
\label{ese:10.14}
 Esegui le seguenti somme di polinomi.

 \begin{enumeratea}
 \item $\left(2a^{2}-3b\right)+\left(4b+3a^{2}\right)+\left(a^{2}-2b\right)$;
 \item $\left(3a^{3}-3b^{2}\right)+\left(6a^{3}+b^{2}\right)+\left(a^{3}-b^{2}\right)$;
 \item $\left(\dfrac{1}{5}x^{3}-5x^{2}+\dfrac{1}{5}x-1\right)-\left(3x^{3}-\dfrac{7}{3}x^{2}+\dfrac{1}{4}x-1\right)$;
 \item $\left(\dfrac{1}{2}+2a^{2}+x\right)-\left(\dfrac{2}{5}a^{2}+\dfrac{1}{2}{ax}\right)+\left[-\left(-{\dfrac{3}{2}}-2{ax}+x^{2}\right)+\dfrac{1}{3}a^{2}\right]-\left(\dfrac{3}{2}{ax}+2\right)$;
 \item $\left(\dfrac{3}{4}a+\dfrac{1}{2}b-\dfrac{1}{6}{ab}\right)-\left(\dfrac{9}{8}{ab}+\dfrac{1}{2}a^{2}-2b\right)+{ab}-\dfrac{3}{4}a$.
\end{enumeratea}
\end{esercizio}

\subsubsection*{10.3 - Prodotto di un polinomio per un monomio}

\begin{esercizio}
\label{ese:10.15}
 Esegui i seguenti prodotti di un monomio per un polinomio.
 \begin{multicols}{3}
\begin{enumeratea}
 \item $(a + b)b$;
 \item $(a - b)b$;
 \item $(a +b)(-b)$;
 \item $(a - b + 51)b$;
 \item $(-a - b -51)(-b)$;
 \item $(a^{2} - a)a$;
 \item $(a^{2} - a)(-a)$;
 \item $(a^{2}- a - 1)a^{2}$;
 \item $(a^{2}b-ab - 1)(ab)$;
 \item $(ab- ab - 1)(ab)$;
 \item $(a^{2}b- ab -1)(a^{2}b^{2})$;
 \item $(a^{2}b-ab - 1)(ab)^{2}$;
 \item $ab(a^{2}b- ab -1)ab$;
 \item $-2a(a^{2} - a - 1)(-a^{2})$;
 \item $(x^{2}a- ax+2)(2x^{2}a^{3})$.
\end{enumeratea}
\end{multicols}
\end{esercizio}3

\begin{esercizio}
\label{ese:10.16}
 Esegui i seguenti prodotti di un monomio per un polinomio.
 \begin{multicols}{2}
\begin{enumeratea}
 \item $\dfrac{3}{4}x^{2}y\cdot\left(2{xy}+\dfrac{1}{3}x^{3}y^{2}\right)$;
 \item $\left(\dfrac{a^{4}}{4}+\dfrac{a^{3}}{8}+\dfrac{a^{2}}{2}\right)\left(2a^{2}\right)$;
 \item $\left(\dfrac{1}{2}a-3+a^{2}\right)\left(-{\dfrac{1}{2}}a\right)$;
 \item $\left(5x+3{xy}+\dfrac{1}{2}y^{2}\right)\left(3x^{2}y\right)$;
 \item $\left(\dfrac{2}{3}xy^{2}+\dfrac{1}{2}x^{3}-\dfrac{3}{4}{xy}\right)(6{xy})$;
 \item $-\dfrac{1}{3}y\left(6x^{2}y-3{xy}\right)$;
 \item $-3xy^2\left(\dfrac{1}{3}x+1\right)$;
 \item $\left(\dfrac{7}{3}b-b\right)\left(a-\dfrac{1}{2}b+1\right)(3a-2a)$.
\end{enumeratea}
\end{multicols}
\end{esercizio}
%\newpage
\subsubsection*{10.4 - Quoziente tra un polinomio e un monomio}
\begin{esercizio}
\label{ese:10.17}
 Svolgi le seguenti divisioni tra polinomi e monomi.
 \begin{multicols}{2}
\begin{enumeratea}
 \item $\left(2x^{2}y+8{xy}^{2}\right):\left(2{xy}\right)$;
 \item $\left(a^{2}+a\right):a$;
 \item $\left(a^{2}-a\right):(-a)$;
 \item $\left(\dfrac{1}{2}a-\dfrac{1}{4}\right):\dfrac{1}{2}$;
 \item $\left(\dfrac{1}{2}a-\dfrac{1}{4}\right):2$;
 \item $(2a-2):\dfrac{1}{2}$;
 \item $\left(\dfrac{1}{2}a-\dfrac{a^{2}}{4}\right):\dfrac{a}{2}$.
\end{enumeratea}
\end{multicols}
\end{esercizio}

\begin{esercizio}
\label{ese:10.18}
 Svolgi le seguenti divisioni tra polinomi e monomi.
 \begin{multicols}{2}
\begin{enumeratea}
 \item $\left(a^{2}-a\right):a$;
 \item $\left(a^{3}+a^{2}-a\right):a$;
 \item $\left(8a^{3}+4a^{2}-2a\right):2a$;
 \item $\left(a^{3}b^{2}+a^{2}b-ab\right):b$;
 \item $\left(a^{3}b^{2}-a^{2}b^{3}-ab^{4}\right):(-{ab}^{2})$;
 \item $\left(a^{3}b^{2}+a^{2}b-ab\right):ab$;
 \item $\left(16x^{4}-12x^{3}+24x^{2}\right):\left(4x^{2}\right)$.
 \item $\left(-x^{3}+3x^{2}-10x+5\right):(-5)$;
\end{enumeratea}
\end{multicols}
\end{esercizio}

\begin{esercizio}
\label{ese:10.19}
 Svolgi le seguenti divisioni tra polinomi e monomi.

\begin{enumeratea}
 \item $\left[\left(-3a^{2}b^{3}-2a^{2}b^{2}+6a^{3}b^{2}\right):(-3{ab})\right]\cdot\left(\dfrac{1}{2}b^{2}\right)$;
 \item $\left(\dfrac{4}{3}a^{2}b^{3}-\dfrac{3}{4}a^{3}b^{2}\right):\left(-{\dfrac{3}{2}a^{2}b^{2}}\right)$;
 \item $\left(2a+\dfrac{a^{2}}{2}-\dfrac{a^{3}}{4}\right):\left(\dfrac{a}{2}\right)$;
 \item $\left(\dfrac{1}{2}a-\dfrac{a^{2}}{4}-\dfrac{a^{3}}{8}\right):\left(\dfrac{1}{2}a\right)$;
 \item $\left(-4x+\dfrac{1}{2}x^{3}\right)\left(2x^{2}-3x+\dfrac{1}{2}\right)$;
 \item $\left(a^{3}b^{2}-a^{4}b+a^{2}b^{3}\right):\left(a^{2}b\right)$;
 \item $\left(a^{2}-a^{4}+a^{3}\right):\left(a^{2}\right)$.
\end{enumeratea}
\end{esercizio}

\subsubsection*{10.5 - Prodotto di polinomi}
\begin{esercizio}
Esegui i seguenti prodotti di polinomi.
\label{ese:10.20}
\begin{multicols}{2}
\begin{enumeratea}
 \item $\left(\dfrac{1}{2}a^{2}b-2{ab}^{2}+\dfrac{3}{4}a^{3}b\right)\cdot\left(\dfrac{1}{2}{ab}\right)$;
 \item $\left(x^{3}-x^{2}+x-1\right)({x}-1)$;
 \item $\left(a^{2}+2{ab}+b^{2}\right)(a+b)$;
 \item $(a-1)(a-2)(a-3)$;
 \item $(a+1)(2a-1)(3a-1)$;
 \item $(a+1)\left(a^{2}+a\right)\left(a^{3}-a^{2}\right)$.
\end{enumeratea}
\end{multicols}
\end{esercizio}


\subsection{Esercizi riepilogativi}

\begin{esercizio}[\Ast]
Risolvi le seguenti espressioni con i polinomi.
 \begin{enumeratea}
 \item $(-a-1-2)-(-3-a+a)$;
 \item $\left(2a^{2}-3b\right)-\left[\left(4b+3a^{2}\right)-\left(a^{2}-2b\right)\right]$;
 \item $\left(2a^{2}-5b\right)-\left[\left(2b+4a^{2}\right)-\left(2a^{2}-2b\right)\right]-9b$;
 \item $3a\left[2(a-2{ab})+3a\left(\dfrac{1}{2}-3b\right)-\dfrac{1}{2}a(3-5b)\right]$;
 \item $2(x-1)(3x+1)-\left(6x^{2}+3x+1\right)+2x(x-1)$.
 \end{enumeratea}
\end{esercizio}

\begin{esercizio}
Risolvi le seguenti espressioni con i polinomi.
 \begin{enumeratea}
 \item $\left(\dfrac{1}{3}x-1\right)(3x+1)-2x\left(\dfrac{5}{4}x-\dfrac{1}{2}\right)(x+1)-\dfrac{1}{2}x\left(x-\dfrac{2}{3}\right)$;
 \item $\left(b^{3}-b\right)(x-b)+(x+b)\left(ab^{2}-a\right)+(b+a)\left(ab-ab^{3}\right)+2ab\left(b-b^{3}\right)$;
 \item $ab\left(a^{2}-b^{2}\right)+2b\left(x^{2}-a^{2}\right)(a-b)-2bx^{2}(a-b)$;
 \item $\left(\dfrac{3}{2}x^{2}y-\dfrac{1}{2}{xy}\right)\left(2x-\dfrac{1}{3}y\right)4x$;
 \item $\left(\dfrac{1}{2}a-\dfrac{1}{2}a^{2}\right)(1-a)\left[a^{2}+2a-\left(a^{2}+a+1\right)\right]$.
 \end{enumeratea}
\end{esercizio}

\begin{esercizio}
Risolvi le seguenti espressioni con i polinomi.
 \begin{enumeratea}
 \item $(1-3x)(1-3x)-(-3x)^{2}+5(x+1)-3(x+1)-7$;
 \item $3\left(x-\dfrac{1}{3}y\right)\left[2x+\dfrac{1}{3}y-(x-2y)\right]-2\left(x-\dfrac{1}{3}y+2\right)(2x+3y)$;
 \item $\dfrac{1}{24}(29x+7)-\dfrac{1}{2}x^{2}+\dfrac{1}{2}(x-3)(x-3)-2-\left[\dfrac{1}{3}-\dfrac{3}{2}\left(\dfrac{3}{4}x+\dfrac{2}{3}\right)\right]$;
 \item $-{\dfrac{1}{4}}\left(2 abx+2a^{2}b^{2}+3 ax\right)+a^{2}(b^{2}+x^{2})-\left[\left(\dfrac{1}{3} ax\right)^{2}-\left(\dfrac{2}{3}bx\right)^{2}\right]$;
 \item $\left(\dfrac{1}{3}x+\dfrac{1}{2}y-\dfrac{3}{5}\right)\left(\dfrac{1}{3}x-\dfrac{1}{2}y+\dfrac{3}{5}\right)-\left[\left(\dfrac{1}{3}x\right)^{2}-\left(\dfrac{1}{2}y\right)^{2}\right]$.
 \end{enumeratea}
\end{esercizio}

\begin{esercizio}
Risolvi le seguenti espressioni con i polinomi.
 \begin{enumeratea}
 \item $\left(\dfrac{1}{2}x-1\right)\left(\dfrac{1}{4}x^{2}+\dfrac{1}{2}x+1\right)+\left(-{\dfrac{1}{2}}x\right)^{3}+2\left(\dfrac{1}{2}x+1\right)$;
 \item $(3a-2)(3a+2)-(a-1)(2a-2)+a(a-1)\left(a^{2}+a+1\right)$;
 \item $-4x(5-2x)+\left(1-4x+x^{2}\right)\left(1-4x-x^{2}\right)$;
 \item $-(2x-1)(2x-1)+\left[x^{2}-\left(1+x^{2}\right)\right]^{2}-\left(x^{2}-1\right)\left(x^{2}+1\right)$.
 \end{enumeratea}
\end{esercizio}

\begin{esercizio}
Risolvi le seguenti espressioni con i polinomi.
 \begin{enumeratea}
 \item $4(x+1)-3x(1-x)-(x+1)(x-1)-\left(4+2x^{2}\right)$;
 \item $\dfrac{1}{2}(x+1)+\dfrac{1}{4}(x+1)(x-1)-\left(x^{2}-1\right)$;
 \item $(3x+1)\left(\dfrac{5}{2}+x\right)-(2x-1)(2x+1)(x-2)+2x^{3}$.
 \end{enumeratea}
\end{esercizio}

\begin{esercizio}[\Ast]
Risolvi le seguenti espressioni con i polinomi.
 \begin{enumeratea}
 \item $\left(a-\dfrac{1}{2}b\right)a^{3}-\left(\dfrac{1}{3}{ab}-1\right)\left[2a^{2}(a-b)-a\left(a^{2}-2{ab}\right)\right]$;
 \item $\left(3x^2+6xy-4y^2\right)\left(\dfrac{1}{2}xy-\dfrac{2}{3}y^2\right)$;
 \item $(2a-3b)\left(\dfrac{5}{4}a^{2}+\dfrac{1}{2}{ab}-\dfrac{1}{6}b^{2}\right)-\dfrac{1}{6}a\left(12a^{2}-\dfrac{18}{5}b^{2}\right)+\dfrac{37}{30}ab^{2}-\dfrac{1}{2}a\left(a^{2}-\dfrac{11}{2}{ab}\right)$;
 \item $\dfrac{1}{3}xy\left[\left(x-y^{2}\right)\left(x^{2}-\dfrac{1}{2}y\right)-3x\left(-{\dfrac{1}{9}xy}\right)\left(3y\right)\right]-\dfrac{1}{3}x\left(x^{3}y+\dfrac{1}{4}xy^{2}\right)$.
 \end{enumeratea}
\end{esercizio}

\begin{esercizio}[\Ast]
Risolvi la seguente espressione con i polinomi.
\begin{multline*}
\dfrac{1}{2}x\left[\left(x-y^{2}\right)\left(x^{2}+\dfrac{1}{2}y\right)-5x\left(-{\dfrac{1}{10}}{xy}\right)(4y)\right]-\dfrac{1}{2}x\left(x^{3}y+\dfrac{1}{2}xy^{2}\right)+\\
-\dfrac{1}{2}x^{2}\left(x^{2}+\dfrac{1}{2}y+{xy}^{2}\right)+\dfrac{1}{4}{xy}\left(y^{2}+2x^{3}+{xy}\right).
\end{multline*}
\end{esercizio}

\begin{esercizio}[\Ast]
Risolvi la seguente espressione con i polinomi.
\begin{multline*}
\left(\dfrac{2}{3}a-2b\right)\left(\dfrac{3}{2}a+2b\right)\left(\dfrac{9}{4}a^{2}+4b^{2}\right)-\dfrac{3}{4}\left(\dfrac{9}{4}a^{2}\right)-a^{2}\left(\dfrac{9}{4}a^{2}-5b^{2}\right)+\\
+5{ab}\left(\dfrac{3}{4}a^{2}+\dfrac{4}{3}b^{2}\right).
\end{multline*}
\end{esercizio}

\begin{esercizio}[\Ast]
Risolvi la seguente espressione con i polinomi.
\begin{multline*}
\left(\dfrac{1}{2}x+2y\right)\left(\dfrac{1}{2}x-2y\right)\left(\dfrac{1}{4}x^{2}-4y^{2}\right)-\dfrac{1}{4}x\left(\dfrac{27}{4}x^{3}-\dfrac{61}{3}xy^{2}\right)+\\
-16\left(y^{4}+x^{4}\right)-\dfrac{37}{12}x^{2}y^{2}+\dfrac{141}{8}x^{4}.
\end{multline*}
\end{esercizio}

\begin{esercizio}[\Ast]
Risolvi la seguente espressione con i polinomi.
\begin{multline*}
x\left(\dfrac{2}{3}y^{2}-\dfrac{27}{8}x^{2}\right)-\left[-\left(\dfrac{3}{2}x-\dfrac{2}{3}y\right)\left(\dfrac{9}{4}x^{2}+xy+\dfrac{4}{3}y^{2}\right)+\dfrac{2}{3}x^{2}\left(\dfrac{9}{4}y^{2}+\dfrac{1}{3}y\right)\right]+\\
+\dfrac{2}{9}y\left(x^{2}+4y^{2}-9xy\right).
\end{multline*}
\end{esercizio}

\begin{esercizio}[\Ast]
Risolvi la seguente espressione con i polinomi.
\begin{multline*}
\left(\dfrac{1}{2}ab+\dfrac{2}{3}xy\right)\left(\dfrac{1}{2}ab-\dfrac{2}{3}xy\right)-\left[\left(\dfrac{1}{2}ab\right)^{2}-\left(\dfrac{2}{3}xy\right)^{2}\right]\left(\dfrac{1}{2}ax\right)+\dfrac{3}{2}ax\left(\dfrac{2}{3}a-\dfrac{2}{3}y\right)+\\
-x\left(\dfrac{1}{2}ax+\dfrac{3}{4}xy\right)-\dfrac{2}{9}x^{2}y^{2}(ax-2)+\dfrac{1}{4}a^{2}b^{2}\left(\dfrac{1}{2}ax-1\right)+\dfrac{3}{4}x^{2}\left(y+\dfrac{2}{3}a\right).
\end{multline*}
\end{esercizio}

\begin{esercizio}[\Ast]
Risolvi la seguente espressione con i polinomi.
\begin{multline*}
\dfrac{1}{6}ab-\dfrac{1}{3}a^{2}-\left\{\dfrac{3}{4}ab+\dfrac{1}{2}a\left[\dfrac{3}{2}b-\left(\dfrac{1}{6}a-\dfrac{4}{5}a\cdot {\dfrac{25}{3}a}\right)\left(-{\dfrac{2}{3}ab}\right)-\left(-{\dfrac{8}{3}ab}\right)\left(-{\dfrac{9}{8}b}\right)\right]\right\}+\\
+\dfrac{1}{3}a\left(a-5b-9a^{3}b+\dfrac{1}{6}a^{2}b\right).
\end{multline*}
\end{esercizio}

\begin{esercizio}[\Ast]
Risolvi la seguente espressione con i polinomi.
\begin{multline*}
\dfrac{1}{5}x^{2}+\left\{\left[2x-\left(\dfrac{3}{2}x^{2}y-\dfrac{7}{4}xy+\dfrac{1}{8}y^{3}\right):\left(-{\dfrac{1}{2}y}\right)\right] 2x-\dfrac{7}{10}xy\right\}\left(-{\dfrac{1}{6}x^{2}}\right)+\\
+x^{2}y-\dfrac{1}{3}x\left(\dfrac{3}{5}x\right)-x^{2}\left(y-x^{3}-\dfrac{1}{12}xy^{2}\right).
\end{multline*}
\end{esercizio}
\newpage
\begin{esercizio}
Se $A=x-1$, $B=2x+2$, $C=x^2-1$ determina
\begin{multicols}{3}
\begin{enumeratea}
\item $A+B+C$;
\item $A\cdot B-C$;
\item $A+B\cdot C$;
\item $A\cdot B\cdot C$;
\item $2AC-2BC$;
\item $(A+B)\cdot C$.
\end{enumeratea}
\end{multicols}
\end{esercizio}

\begin{esercizio}[\Ast]
 Operazioni tra polinomi con esponenti letterali.

\begin{enumeratea}
\item $\left(a^{n+1}-a^{n+2}+a^{n+3}\right):\left(a^{1+n}\right)$;
\item $\left(1+a^{n+1}\right)\left(1-a^{n-1}\right)$;
\item $\left(16a^{n+1}b^{n+2}-2a^{2n}b^{n+3}+5a^{n+2}b^{n+1}\right):\left(2a^{n}b^{n}\right)$;
\item $\left(a^{n+1}-a^{n+2}+a^{n+3}\right)\left(a^{n+1}-a^{n}\right)$;
\item $\left(a^{n}-a^{n+1}+a^{n+2}\right)\left(a^{n+1}-a^{n-1}\right)$;
\item $\left(a^{n}+a^{n+1}+a^{n+2}\right)\left(a^{n+1}-a^{n}\right)$;
\item $\left(a^{n+2}+a^{n+1}\right)\left(a^{n+1}+a^{n+2}\right)$;
\item $\left(1+a^{n+1}\right)\left(a^{n+1}-2\right)$;
\item $\left(a^{n+1}-a^{n}\right)\left(a^{n+1}+a^{n}\right)\left(a^{2n+2}+a^{2n}\right)$;
\item $\left(\dfrac{1}{2}x^{n}-\dfrac{3}{2}x^{2n}\right)\left(\dfrac{1}{3}x^{n}-\dfrac{1}{2}\right)-\left(\dfrac{1}{3}x^{n}-1\right)\left(x^{n}+x\right)$.
\end{enumeratea}
\end{esercizio}
\begin{multicols}{2}
\begin{esercizio}
 Se si raddoppiano i lati di un rettangolo, come varia il suo
perimetro?
\end{esercizio}

\begin{esercizio}
 Se si raddoppiano i lati di un triangolo rettangolo, come varia la sua
area?
\end{esercizio}

\begin{esercizio}
 Se si raddoppiano gli spigoli~$a$, $b$ e~$c$ di un parallelepipedo, come
varia il suo volume?
\end{esercizio}

\begin{esercizio}
 Come varia l'area di un cerchio se si triplica il suo
raggio?
\end{esercizio}

\begin{esercizio}
 Determinare l'area di un rettangolo avente come
dimensioni~$\frac{1}{2}a$ e~$\frac{3}{4}a^{2}b$.
\end{esercizio}

\begin{esercizio}
 Determinare la superficie laterale di un cilindro avente raggio di
base~$x^{2}y$ e altezza~$\frac{1}{5}{xy}^{2}$.
\end{esercizio}
\end{multicols}

\subsection{Risposte}
\begin{multicols}{2}
\paragraph{10.14.} d)~$-x^{2}+x+\frac{29}{15}a^{2}$,\protect\\ e)~$-{\frac{a^{2}}{2}}-\frac{7}{24}ab+\frac{5}{2}b$.
\paragraph{10.21.} a)~$-a$,\quad b)~$-9b$,\quad c)~$-18b$,\protect\\ d)~$6a^{2}-\frac{63}{2}a^{2}b$,\quad e)~$2x^2-9x-3$.
\paragraph{10.26.} a)~$a^{4}-\frac{1}{2}a^{3}b-\frac{1}{3}a^{4}b+a^{3}$,\protect\\ b)~$\frac{3}{2}x^{3}y+x^{2}y^{2}-6{xy}^{3}+\frac{8}{3}y^{4}$,\protect\\ c)~$\frac{1}{2}b^{3}$,\quad d)~$\frac{1}{6}xy^{4}-\frac{1}{4}x^{2}y^{2}$.
\paragraph{10.27.} $0$.
\paragraph{10.28.} $-16b^{4}-\frac{27}{16}a^{2}$.
\paragraph{10.29.} $0$.
\paragraph{10.30.} $-\frac{3}{2}x^{2}y^{2}$.
\paragraph{10.31.} $a^{2}x-axy$.
\paragraph{10.32.} $-\frac{7}{9}a^{4}b+\frac{3}{2}a^2b^2-3ab$.
\paragraph{10.33.} $\frac{1}{2}x^{4}+\frac{7}{60}x^{3}y$.
\paragraph{10.35.} a)~$1-a+a^{2}$,\protect\\ b)~$1-a^{n-1}+a^{n+1}-a^{2}n$,\protect\\ c)~$8ab^2-a^nb^3+\frac{5}{2}a^2b$, \protect \\
d)~$a^{2n+4}-2a^{2n+3}+2a^{2n+2}-a^{2n+1}$,\protect\\ e)~$a^{2n+3}-a^{2n+2}-a^{2n-1}+a^{2n}$,\protect\\ f)~$-a^{2}n+a^{2n+3}$,\quad
\protect\\ g)~$a^{2n+4}+2a^{2n+3}+a^{2n+2}$,\protect\\ i)~$a^{2n+2}-a^{n+1}-2$,\quad h)~$a^{4n+4}-a^{4n}$,\protect\\
j)~$\frac{7}{12}x^{2n}+\frac{3}{4}x^{n}-\frac{1}{2}x^{3n}-\frac{1}{3}x^{n+1}+x$.
\end{multicols}

\cleardoublepage
