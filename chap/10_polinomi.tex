% (c) 2012 - 2014 Dimitrios Vrettos - d.vrettos@gmail.com
% (c) 2012 Silvia Cibola - silvia.cibola@gmail.com

\chapter{Polinomi}
\section{Definizioni fondamentali}

\begin{definizione}
Un \emph{polinomio} è un'espressione algebrica letterale che consiste in una somma algebrica di monomi.
\end{definizione}

\begin{exrig}
\begin{esempio}
Sono polinomi:~$6a+2b$, $\:5a^2b+3b^2$, $\:6x^2-5y^2x-1$, $\:7ab-2a^2b^3+4$.
\end{esempio}
\end{exrig}

Se tra i termini di un polinomio non sono presenti monomi simili, il polinomio si dice in \emph{forma normale} o
\emph{ridotto}; se al contrario si presentano dei termini simili, possiamo eseguire la riduzione del polinomio
sommando algebricamente i termini simili. Tutti i polinomi sono quindi riducibili in forma normale.

Un polinomio in forma normale può presentare, tra i suoi termini, un monomio di grado~0 che viene
comunemente chiamato \emph{termine noto}.

\begin{exrig}
\begin{esempio}
Il polinomio~$3ab+b^2−2ba+4−6ab^2+5b^2$ ridotto in forma normale diventa~$ab+6b^2−6ab^2+4$. Il termine noto è~$4$.
\end{esempio}
\end{exrig}

\ovalbox{\risolvi\ref{ese:10.1}}\vspazio

Un polinomio può anche essere costituito da un unico termine, pertanto un monomio è anche un polinomio.
Un polinomio che, ridotto in forma normale, è somma algebrica di due, tre, quattro monomi non nulli si dice
rispettivamente binomio, trinomio, quadrinomio.

\begin{exrig}
\begin{esempio}
Binomi, trinomi, quadrinomi.
\begin{enumeratea}
\item $xy−5x^3y^2$ è un binomio;
\item $3ab^2 +a−4a^3$ è un trinomio;
\item $a−6ab^2+3ab−5b$ è un quadrinomio.
\end{enumeratea}
\end{esempio}
\end{exrig}

\begin{definizione}
Due polinomi, ridotti in forma normale, formati da termini uguali si dicono \emph{uguali}, più precisamente vale il \emph{principio di identità dei polinomi}:
due polinomi~$p(x)$ e~$q(x)$ sono uguali se, e solo se, sono uguali
i coefficienti dei rispettivi termini simili.

Se due polinomi sono invece formati da tutti termini opposti, allora si dicono polinomi \emph{opposti}.

Definiamo, inoltre, un polinomio \emph{nullo} quando i suoi termini sono a coefficienti nulli. Il polinomio nullo
coincide con il monomio nullo e quindi con il numero~0.
\end{definizione}
\pagebreak
\begin{exrig}
\begin{esempio}
Polinomi uguali, opposti, nulli.
\begin{enumeratea}
\item I polinomi\quad $\frac{1}{3}xy+2y^3−x$\quad e \quad$2y^3-x+\frac{1}{3}xy$ \quad sono uguali;
\item i polinomi\quad $6ab−3a+2b$\quad e \quad$3a−2b−6ab$ \quad sono opposti;
\item il polinomio \quad $7ab+4a^2−ab+b^3−4a^2−2b^3−6ab+b^3$ \quad è un polinomio nullo, infatti riducendolo in forma normale otteniamo il monomio nullo~$0$.
\end{enumeratea}
\end{esempio}
\end{exrig}

\begin{definizione}
Il \emph{grado complessivo} (o semplicemente \emph{grado}) di un polinomio è il massimo dei gradi complessivi dei suoi
termini. Si chiama, invece, \emph{grado di un polinomio rispetto ad una data lettera} l'esponente maggiore con
cui quella lettera compare nel polinomio, dopo che è stato ridotto a forma normale.
\end{definizione}

\begin{exrig}
\begin{esempio} Grado di un polinomio.
\begin{itemize*}
\item Il polinomio~$2ab+3−4a^2b^2$ ha grado complessivo~$4$ perché il monomio con grado massimo è~$−4a^2b^2 $, che è un monomio di quarto grado;
\item il grado del polinomio~$a^3+3b^2a−4ba^2$ rispetto alla lettera~$a$ è~$3$ perché l'esponente più grande con cui tale lettera compare è~$3$.
\end{itemize*}
\end{esempio}
\end{exrig}

\ovalbox{\risolvi\ref{ese:10.2}}

\begin{definizione}
Un polinomio si dice \emph{omogeneo} se tutti i termini che lo compongono sono dello stesso grado.
\end{definizione}

\begin{exrig}
\begin{esempio}
Il polinomio~$a^3−b^3+ab^2$ è un polinomio omogeneo di grado~$3$.
\end{esempio}
\end{exrig}

\ovalbox{\risolvi\ref{ese:10.3}}

\begin{definizione}
Un polinomio si dice \emph{ordinato secondo le potenze decrescenti (crescenti) di una lettera}, quando i suoi
termini sono ordinati in maniera tale che gli esponenti di tale lettera decrescono (crescono), leggendo il
polinomio da sinistra verso destra.
\end{definizione}

\begin{exrig}
\begin{esempio}
Il polinomio~$\frac{1}{2}x^3+\frac{3}{4}x^2y−2xy^2+\frac{3}{8}y^3$ è ordinato secondo le potenze decrescenti della lettera~$x$, e secondo le potenze crescenti della lettera~$y$.
\end{esempio}
\end{exrig}

\begin{definizione}
Un polinomio di grado~$n$ rispetto ad una data lettera si dice \emph{completo} se contiene tutte le potenze di tale lettera di grado inferiore a~$n$, compreso il termine noto.
\end{definizione}
%\newpage
\begin{exrig}
\begin{esempio}
Il polinomio~$x^4-3x^3+5x^2+\frac{1}{2}x-\frac{3}{5}$ è completo di grado~$4$ e inoltre risulta ordinato rispetto alla lettera~$x$. Il termine noto è~$-\frac{3}{5}$.
\end{esempio}
\end{exrig}

\osservazione
Ogni polinomio può essere scritto sotto forma ordinata e completa: l'ordinamento si può effettuare in virtù della proprietà commutativa della somma, mentre
la completezza si può ottenere mediante l'introduzione dei termini dei gradi mancanti con coefficiente uguale a~$0$.

Per esempio, il polinomio~$x^4−x+1+4x^2$ può essere scritto sotto forma ordinata e completa come~$x^4+0x^3+4x^2−x+1$.

\vspazio\ovalbox{\risolvii \ref{ese:10.4}, \ref{ese:10.5}, \ref{ese:10.6}, \ref{ese:10.7}, \ref{ese:10.8}, \ref{ese:10.9}, \ref{ese:10.10}}

\section{Somma algebrica di polinomi}

I polinomi sono somme algebriche di monomi e quindi le espressioni letterali che si ottengono dalla somma
o differenza di polinomi sono ancora somme algebriche di monomi.

\begin{definizione}
La \emph{somma algebrica di due o più polinomi} è un polinomio avente per termini tutti i termini dei polinomi addendi.
\end{definizione}

La differenza di polinomi si può trasformare in somma del primo polinomio con l'opposto del secondo polinomio.

\begin{exrig}
\begin{esempio}
Differenza di polinomi.
\begin{equation*}
\begin{split}
3a^2+2b-\frac{1}{2}ab-\left(2a^2+ab-\frac{1}{2}b\right)&=3a^2+2b-\frac{1}{2}ab-2a^2-ab+\frac{1}{2}b\\
&=a^2+\frac{-1-2}{2}ab+\frac{4+1}{2}b\\
&=a^2-\frac{3}{2}ab+\frac{5}{2}b.
\end{split}
\end{equation*}
\end{esempio}
\end{exrig}

\ovalbox{\risolvii \ref{ese:10.11}, \ref{ese:10.12}, \ref{ese:10.13}}

\section{Prodotto di un polinomio per un monomio}

Per eseguire il prodotto tra il monomio~$3x^{2}y$ e il polinomio
$2{xy}+5x^{3}y^{2}$; indichiamo il prodotto con
$\left(3x^{2}y\right)\cdot \left(2{xy}+5x^{3}y^{2}\right)$.
Applichiamo la proprietà distributiva della moltiplicazione rispetto
all'addizione:~$\left(3x^{2}y\right)\cdot
\left(2{xy}+5x^{3}y^{2}\right)=6x^{3}y^{2}+15x^{5}y^{3}$.

\osservazione Il prodotto di un monomio per un polinomio è
un polinomio avente come termini i prodotti del monomio per ciascun
termine del polinomio.

%\newpage
\begin{exrig}
 \begin{esempio}
 Prodotto di un monomio per un polinomio.

 \begin{equation*}
\begin{split}
 \left(3x^{3}y\right)\cdot\left(\frac{1}{2}x^{2}y^{2}+\frac{4}{3}{xy}^{3}\right)&=\left(3x^{3}y\right)\cdot\left(\frac{1}{2}x^{2}y^{2}\right)+\left(3x^{3}y\right)\cdot%
\left(\frac{4}{3}{xy}^{3}\right)\\
&=\frac{3}{2}x^{5}y^{3}+4x^{4}y^{4}.
\end{split}
\end{equation*}
 \end{esempio}
\end{exrig}

\ovalbox{\risolvii \ref{ese:10.14}, \ref{ese:10.15}}

\section{Quoziente tra un polinomio e un monomio}

Il quoziente tra un polinomio e un monomio si calcola applicando la
proprietà distributiva della divisione rispetto
all'addizione.

\begin{definizione}
 Si dice che un \emph{polinomio è divisibile per un monomio}, non
nullo, se esiste un polinomio che, moltiplicato per il monomio, dà
come risultato il polinomio dividendo; il monomio si dice
\emph{divisore} del polinomio.
\end{definizione}

\begin{exrig}
 \begin{esempio}
 Quoziente tra un polinomio e un monomio.
 \[\left(6x^{5}y+9x^{3}y^{2}\right):\left(3x^{2}y\right)=2x^{(5-2)}y^{(1-1)}+3x^{(3-2)}y^{(2-1)}=2x^{3}+3{xy}.\]
 \end{esempio}
\end{exrig}
\osservazione

\begin{enumeratea}
\item Poiché ogni monomio è divisibile per qualsiasi numero diverso
da zero, allora anche ogni polinomio è divisibile per un qualsiasi
numero diverso da zero;
\item un polinomio è divisibile per un monomio, non nullo, se ogni
fattore letterale del monomio divisore compare, con grado uguale o
maggiore, in ogni monomio del polinomio dividendo;
\item la divisione tra un polinomio e un qualsiasi monomio non nullo è
sempre possibile, tuttavia il risultato è un polinomio solo nel caso
in cui il monomio sia divisore di tutti i termini del polinomio;
\item il quoziente tra un polinomio e un monomio suo divisore è un
polinomio ottenuto dividendo ogni termine del polinomio per il monomio
divisore.
\end{enumeratea}

\ovalbox{\risolvii \ref{ese:10.16}, \ref{ese:10.17}, \ref{ese:10.18}}

\section{Prodotto di polinomi}

Il prodotto di due polinomi è il polinomio che si ottiene
moltiplicando ogni termine del primo polinomio per ciascun termine del
secondo polinomio.

%\newpage
\begin{exrig}
 \begin{esempio}
 Prodotto di polinomi.

 \begin{enumeratea}
   \item $\left(a^{2}b+3a-4{ab}\right)\left(\frac{1}{2}a^{2}b^{2}-a+3{ab}^{2}\right).$ Riducendo i termini simili:
 \begin{multline*}
  \left(a^{2}b+3a-4{ab}\right)\left(\frac{1}{2}a^{2}b^{2}-a+3{ab}^{2}\right)=%
   \frac{1}{2}a^{4}b^{3}-a^{3}b+3a^{3}b^{3}+\frac{3}{2}a^{3}b^{2}-3a^{2}+\\
   +9a^{2}b^{2} -2a^{3}b^{3}+4a^{2}b-12a^{2}b^{3}\\
  =\frac{1}{2}a^{4}b^{3}-a^{3}b+a^{3}b^{3}+\frac{3}{2}a^{3}b^{2}-3a^{2}+9a^{2}b^{2}+4a^{2}b-12a^{2}b^{3}.
 \end{multline*}
 \item  $\left(x-y^{2}-3{xy}\right) \left(-2x^{2}y-3y\right).$
 Moltiplicando ogni termine del primo polinomio per ogni termine del
secondo otteniamo.
\[\left(x-y^{2}-3{xy}\right)\left(-2x^{2}y-3y\right)=-2x^{3}y+3{xy}+2x^{2}y^{3}-3y^{3}+6x^{3}y^{2}+9{xy}^{2};\]

 \item $\left(\frac{1}{2}x^{3}-2x^{2}\right)\left(\frac{3}{4}x+1\right)$.
 \[\left(\frac{1}{2}x^{3}-2x^{2}\right)\left(\frac{3}{4}x+1\right)=\frac{3}{8}x^{4}+\frac{1}{2}x^{3}-\frac{3}{2}x^{3}-2x^{2}=\frac{3}{8}x^{4}-x^{3}-2x^{2}.\]
 \end{enumeratea}
 \end{esempio}
\end{exrig}
\ovalbox{\risolvi \ref{ese:10.19}}
\newpage
% (c) 2012-2014 Dimitrios Vrettos - d.vrettos@gmail.com
% (c) 2012, 2014 Claudio Carboncini - claudio.carboncini@gmail.com
% (c) 2012 Silvia Cibola - silvia.cibola@gmail.com
\section{Esercizi}
\subsection{Esercizi dei singoli paragrafi}
\subsubsection*{10.1 - Definizioni fondamentali}
\begin{multicols}{2}
\begin{esercizio}
\label{ese:10.1}
Riduci in forma normale il seguente polinomio:
\[5a^3-4ab-1+2a^3+2ab-a-3a^3.\]
\emph{Svolgimento}: Evidenziamo i termini simili e sommiamoli tra di loro:
%\[\underline{5a^3}-\overline{4ab}+1+\underline{2a^3}+\overline{2ab}-a-\underline{3a^3}\]
\[\mmevid{ev_rosso}{5a^{3}}-\mmevid{ev_verde}{4ab}+1+\mmevid{ev_rosso}{2a^{3}}+\mmevid{ev_verde}{2ab}-a-\mmevid{ev_rosso}{3a^{3}}\]
così otteniamo \dotfill Il termine noto è \dotfill
\end{esercizio}

\begin{esercizio}
\label{ese:10.2}
Il grado di:
\begin{enumeratea}
\item $x^2y^2−3y^3+5yx−6y^2x^3$ rispetto alla lettera~$y$ è \dotfill, il grado complessivo è \dotfill
\item $5a^2−b+4ab$ rispetto alla~$b$ è \dotfill,\\ il grado complessivo è \dotfill
\end{enumeratea}
\end{esercizio}


\begin{esercizio}
\label{ese:10.3}
Stabilire quali dei seguenti polinomi sono omogenei:

\begin{enumeratea}
\item $x^3y+2y^2x^2−4x^4$;
\item $2x+3−xy$;
\item $2x^3y^3−y^4x^2+5x^6$.
\end{enumeratea}
\end{esercizio}

\begin{esercizio}
\label{ese:10.4}
Individuare quali dei seguenti polinomi sono ordinati rispetto alla lettera~$x$ con potenze crescenti:

\begin{enumeratea}
\item $2-\dfrac{1}{2}x^2+x$;
\item $\dfrac{2}{3}-x+3x^2+5x^3$;
\item $3x^4-\dfrac{1}{2}x^3+2x^2-x+\dfrac{7}{8}$.
\end{enumeratea}
\end{esercizio}

\begin{esercizio}
\label{ese:10.5}
Relativamente al polinomio~$b^2+a^4+a^3+a^2$:
\begin{itemize*}
\item Il grado massimo è \ldots. Il grado rispetto alla lettera~$a$ è \ldots. Rispetto alla lettera~$b$ è \ldots
\item il polinomio è ordinato rispetto alla $a$? %\tab\qquad\boxV\qquad\boxF
\item è completo? %\tab\qquad\boxV\qquad\boxF
\item è omogeneo? %\tab\qquad\boxV\qquad\boxF
\end{itemize*}
\end{esercizio}

\begin{esercizio}
\label{ese:10.6}
Scrivere un polinomio di terzo grado nelle variabili~$a$ e~$b$ che sia omogeneo.
\end{esercizio}

\begin{esercizio}
\label{ese:10.7}
Scrivere un polinomio di quarto grado nelle variabili~$x$ e~$y$ che sia omogeneo e ordinato secondo le
potenze decrescenti della seconda indeterminata.
\end{esercizio}

\begin{esercizio}
\label{ese:10.8}
Scrivere un polinomio di quinto grado nelle variabili~$r$ e~$s$ che sia omogeneo e ordinato secondo le
potenze crescenti della prima indeterminata.
\end{esercizio}

\begin{esercizio}
\label{ese:10.9}
Scrivere un polinomio di quarto grado nelle variabili~$z$ e~$w$ che sia omogeneo e ordinato secondo le
potenze crescenti della prima indeterminata e decrescenti della seconda.
\end{esercizio}

\begin{esercizio}
\label{ese:10.10}
Scrivere un polinomio di sesto grado nelle variabili~$x$, $y$ e~$z$ che sia completo e ordinato secondo le
potenze decrescenti della seconda variabile.
\end{esercizio}


\begin{esercizio}
\label{ese:10.11}
Calcola il valore numerico dei polinomi per i valori a fianco indicati.

\begin{enumeratea}
\item $x^2+x$ per $x=-1$;
\item $2x^2-3x+1$ per $x=0$;
\item $3x^2-2x-1$ per $x=2$;
\item $3x^3-2x+x$ per $x=-2$;
\item $\dfrac{3}{4}a+\dfrac{1}{2}b-\dfrac{1}{6}ab$ per $a=-\dfrac{1}{2}$, $b=3$;
\item $4x-6y+\dfrac{1}{5}x^2$ per $x=-5$, $y=\dfrac{1}{2}$.
\end{enumeratea}
\end{esercizio}
\end{multicols}


\subsubsection*{10.2 - Somma algebrica di polinomi}
\begin{esercizio}
\label{ese:10.12}
Calcolare la somma dei due polinomi:~$2x^2+5−3y^2x$, $x^2−xy+2−y^2x+y^3$.

\emph{Svolgimento}: Indichiamo la somma~$(2x^2+5−3y^2x)+(x^2−xy+2−y^2x+y^3)$, eliminando le parentesi otteniamo
il polinomio~$2x^2+5−3y^2x+x^2−xy+2−y^2x+y^3$, sommando i monomi simili otteniamo~$3x^2−4x^{\ldots}y^{\ldots}-\ldots xy+y^3+\ldots$
\end{esercizio}
%\newpage
\begin{esercizio}
\label{ese:10.13}
 Esegui le seguenti somme di polinomi.
 \begin{enumeratea}
 \item $a+b-b$;
 \item $a+b-2b$;
 \item $a+b-(-2b)$;
 \item $a-(b-2b)$;
 \item $2a+b+(3a+b)$;
 \item $2a+2b+(2a+b)+2a$;
 \item $2a+b-(-3a-b)$;
 \item $2a-3b-(-3b-2a)$;
 \item $(a+1)-(a-3)$.
\end{enumeratea}
\end{esercizio}


\begin{esercizio}[\Ast]
\label{ese:10.14}
 Esegui le seguenti somme di polinomi.

 \begin{enumeratea}
 \item $\left(2a^{2}-3b\right)+\left(4b+3a^{2}\right)+\left(a^{2}-2b\right)$;
 \item $\left(3a^{3}-3b^{2}\right)+\left(6a^{3}+b^{2}\right)+\left(a^{3}-b^{2}\right)$;
 \item $\left(\dfrac{1}{5}x^{3}-5x^{2}+\dfrac{1}{5}x-1\right)-\left(3x^{3}-\dfrac{7}{3}x^{2}+\dfrac{1}{4}x-1\right)$;
 \item $\left(\dfrac{1}{2}+2a^{2}+x\right)-\left(\dfrac{2}{5}a^{2}+\dfrac{1}{2}{ax}\right)+\left[-\left(-{\dfrac{3}{2}}-2{ax}+x^{2}\right)+\dfrac{1}{3}a^{2}\right]-\left(\dfrac{3}{2}{ax}+2\right)$;
 \item $\left(\dfrac{3}{4}a+\dfrac{1}{2}b-\dfrac{1}{6}{ab}\right)-\left(\dfrac{9}{8}{ab}+\dfrac{1}{2}a^{2}-2b\right)+{ab}-\dfrac{3}{4}a$.
\end{enumeratea}
\end{esercizio}

\subsubsection*{10.3 - Prodotto di un polinomio per un monomio}

\begin{esercizio}
\label{ese:10.15}
 Esegui i seguenti prodotti di un monomio per un polinomio.
 \begin{multicols}{3}
\begin{enumeratea}
 \item $(a + b)b$;
 \item $(a - b)b$;
 \item $(a +b)(-b)$;
 \item $(a - b + 51)b$;
 \item $(-a - b -51)(-b)$;
 \item $(a^{2} - a)a$;
 \item $(a^{2} - a)(-a)$;
 \item $(a^{2}- a - 1)a^{2}$;
 \item $(a^{2}b-ab - 1)(ab)$;
 \item $(ab- ab - 1)(ab)$;
 \item $(a^{2}b- ab -1)(a^{2}b^{2})$;
 \item $(a^{2}b-ab - 1)(ab)^{2}$;
 \item $ab(a^{2}b- ab -1)ab$;
 \item $-2a(a^{2} - a - 1)(-a^{2})$;
 \item $(x^{2}a- ax+2)(2x^{2}a^{3})$.
\end{enumeratea}
\end{multicols}
\end{esercizio}3

\begin{esercizio}
\label{ese:10.16}
 Esegui i seguenti prodotti di un monomio per un polinomio.
 \begin{multicols}{2}
\begin{enumeratea}
 \item $\dfrac{3}{4}x^{2}y\cdot\left(2{xy}+\dfrac{1}{3}x^{3}y^{2}\right)$;
 \item $\left(\dfrac{a^{4}}{4}+\dfrac{a^{3}}{8}+\dfrac{a^{2}}{2}\right)\left(2a^{2}\right)$;
 \item $\left(\dfrac{1}{2}a-3+a^{2}\right)\left(-{\dfrac{1}{2}}a\right)$;
 \item $\left(5x+3{xy}+\dfrac{1}{2}y^{2}\right)\left(3x^{2}y\right)$;
 \item $\left(\dfrac{2}{3}xy^{2}+\dfrac{1}{2}x^{3}-\dfrac{3}{4}{xy}\right)(6{xy})$;
 \item $-\dfrac{1}{3}y\left(6x^{2}y-3{xy}\right)$;
 \item $-3xy^2\left(\dfrac{1}{3}x+1\right)$;
 \item $\left(\dfrac{7}{3}b-b\right)\left(a-\dfrac{1}{2}b+1\right)(3a-2a)$.
\end{enumeratea}
\end{multicols}
\end{esercizio}
%\newpage
\subsubsection*{10.4 - Quoziente tra un polinomio e un monomio}
\begin{esercizio}
\label{ese:10.17}
 Svolgi le seguenti divisioni tra polinomi e monomi.
 \begin{multicols}{2}
\begin{enumeratea}
 \item $\left(2x^{2}y+8{xy}^{2}\right):\left(2{xy}\right)$;
 \item $\left(a^{2}+a\right):a$;
 \item $\left(a^{2}-a\right):(-a)$;
 \item $\left(\dfrac{1}{2}a-\dfrac{1}{4}\right):\dfrac{1}{2}$;
 \item $\left(\dfrac{1}{2}a-\dfrac{1}{4}\right):2$;
 \item $(2a-2):\dfrac{1}{2}$;
 \item $\left(\dfrac{1}{2}a-\dfrac{a^{2}}{4}\right):\dfrac{a}{2}$.
\end{enumeratea}
\end{multicols}
\end{esercizio}

\begin{esercizio}
\label{ese:10.18}
 Svolgi le seguenti divisioni tra polinomi e monomi.
 \begin{multicols}{2}
\begin{enumeratea}
 \item $\left(a^{2}-a\right):a$;
 \item $\left(a^{3}+a^{2}-a\right):a$;
 \item $\left(8a^{3}+4a^{2}-2a\right):2a$;
 \item $\left(a^{3}b^{2}+a^{2}b-ab\right):b$;
 \item $\left(a^{3}b^{2}-a^{2}b^{3}-ab^{4}\right):(-{ab}^{2})$;
 \item $\left(a^{3}b^{2}+a^{2}b-ab\right):ab$;
 \item $\left(16x^{4}-12x^{3}+24x^{2}\right):\left(4x^{2}\right)$.
 \item $\left(-x^{3}+3x^{2}-10x+5\right):(-5)$;
\end{enumeratea}
\end{multicols}
\end{esercizio}

\begin{esercizio}
\label{ese:10.19}
 Svolgi le seguenti divisioni tra polinomi e monomi.

\begin{enumeratea}
 \item $\left[\left(-3a^{2}b^{3}-2a^{2}b^{2}+6a^{3}b^{2}\right):(-3{ab})\right]\cdot\left(\dfrac{1}{2}b^{2}\right)$;
 \item $\left(\dfrac{4}{3}a^{2}b^{3}-\dfrac{3}{4}a^{3}b^{2}\right):\left(-{\dfrac{3}{2}a^{2}b^{2}}\right)$;
 \item $\left(2a+\dfrac{a^{2}}{2}-\dfrac{a^{3}}{4}\right):\left(\dfrac{a}{2}\right)$;
 \item $\left(\dfrac{1}{2}a-\dfrac{a^{2}}{4}-\dfrac{a^{3}}{8}\right):\left(\dfrac{1}{2}a\right)$;
 \item $\left(-4x+\dfrac{1}{2}x^{3}\right)\left(2x^{2}-3x+\dfrac{1}{2}\right)$;
 \item $\left(a^{3}b^{2}-a^{4}b+a^{2}b^{3}\right):\left(a^{2}b\right)$;
 \item $\left(a^{2}-a^{4}+a^{3}\right):\left(a^{2}\right)$.
\end{enumeratea}
\end{esercizio}

\subsubsection*{10.5 - Prodotto di polinomi}
\begin{esercizio}
Esegui i seguenti prodotti di polinomi.
\label{ese:10.20}
\begin{multicols}{2}
\begin{enumeratea}
 \item $\left(\dfrac{1}{2}a^{2}b-2{ab}^{2}+\dfrac{3}{4}a^{3}b\right)\cdot\left(\dfrac{1}{2}{ab}\right)$;
 \item $\left(x^{3}-x^{2}+x-1\right)({x}-1)$;
 \item $\left(a^{2}+2{ab}+b^{2}\right)(a+b)$;
 \item $(a-1)(a-2)(a-3)$;
 \item $(a+1)(2a-1)(3a-1)$;
 \item $(a+1)\left(a^{2}+a\right)\left(a^{3}-a^{2}\right)$.
\end{enumeratea}
\end{multicols}
\end{esercizio}


\subsection{Esercizi riepilogativi}

\begin{esercizio}[\Ast]
Risolvi le seguenti espressioni con i polinomi.
 \begin{enumeratea}
 \item $(-a-1-2)-(-3-a+a)$;
 \item $\left(2a^{2}-3b\right)-\left[\left(4b+3a^{2}\right)-\left(a^{2}-2b\right)\right]$;
 \item $\left(2a^{2}-5b\right)-\left[\left(2b+4a^{2}\right)-\left(2a^{2}-2b\right)\right]-9b$;
 \item $3a\left[2(a-2{ab})+3a\left(\dfrac{1}{2}-3b\right)-\dfrac{1}{2}a(3-5b)\right]$;
 \item $2(x-1)(3x+1)-\left(6x^{2}+3x+1\right)+2x(x-1)$.
 \end{enumeratea}
\end{esercizio}

\begin{esercizio}
Risolvi le seguenti espressioni con i polinomi.
 \begin{enumeratea}
 \item $\left(\dfrac{1}{3}x-1\right)(3x+1)-2x\left(\dfrac{5}{4}x-\dfrac{1}{2}\right)(x+1)-\dfrac{1}{2}x\left(x-\dfrac{2}{3}\right)$;
 \item $\left(b^{3}-b\right)(x-b)+(x+b)\left(ab^{2}-a\right)+(b+a)\left(ab-ab^{3}\right)+2ab\left(b-b^{3}\right)$;
 \item $ab\left(a^{2}-b^{2}\right)+2b\left(x^{2}-a^{2}\right)(a-b)-2bx^{2}(a-b)$;
 \item $\left(\dfrac{3}{2}x^{2}y-\dfrac{1}{2}{xy}\right)\left(2x-\dfrac{1}{3}y\right)4x$;
 \item $\left(\dfrac{1}{2}a-\dfrac{1}{2}a^{2}\right)(1-a)\left[a^{2}+2a-\left(a^{2}+a+1\right)\right]$.
 \end{enumeratea}
\end{esercizio}

\begin{esercizio}
Risolvi le seguenti espressioni con i polinomi.
 \begin{enumeratea}
 \item $(1-3x)(1-3x)-(-3x)^{2}+5(x+1)-3(x+1)-7$;
 \item $3\left(x-\dfrac{1}{3}y\right)\left[2x+\dfrac{1}{3}y-(x-2y)\right]-2\left(x-\dfrac{1}{3}y+2\right)(2x+3y)$;
 \item $\dfrac{1}{24}(29x+7)-\dfrac{1}{2}x^{2}+\dfrac{1}{2}(x-3)(x-3)-2-\left[\dfrac{1}{3}-\dfrac{3}{2}\left(\dfrac{3}{4}x+\dfrac{2}{3}\right)\right]$;
 \item $-{\dfrac{1}{4}}\left(2 abx+2a^{2}b^{2}+3 ax\right)+a^{2}(b^{2}+x^{2})-\left[\left(\dfrac{1}{3} ax\right)^{2}-\left(\dfrac{2}{3}bx\right)^{2}\right]$;
 \item $\left(\dfrac{1}{3}x+\dfrac{1}{2}y-\dfrac{3}{5}\right)\left(\dfrac{1}{3}x-\dfrac{1}{2}y+\dfrac{3}{5}\right)-\left[\left(\dfrac{1}{3}x\right)^{2}-\left(\dfrac{1}{2}y\right)^{2}\right]$.
 \end{enumeratea}
\end{esercizio}

\begin{esercizio}
Risolvi le seguenti espressioni con i polinomi.
 \begin{enumeratea}
 \item $\left(\dfrac{1}{2}x-1\right)\left(\dfrac{1}{4}x^{2}+\dfrac{1}{2}x+1\right)+\left(-{\dfrac{1}{2}}x\right)^{3}+2\left(\dfrac{1}{2}x+1\right)$;
 \item $(3a-2)(3a+2)-(a-1)(2a-2)+a(a-1)\left(a^{2}+a+1\right)$;
 \item $-4x(5-2x)+\left(1-4x+x^{2}\right)\left(1-4x-x^{2}\right)$;
 \item $-(2x-1)(2x-1)+\left[x^{2}-\left(1+x^{2}\right)\right]^{2}-\left(x^{2}-1\right)\left(x^{2}+1\right)$.
 \end{enumeratea}
\end{esercizio}

\begin{esercizio}
Risolvi le seguenti espressioni con i polinomi.
 \begin{enumeratea}
 \item $4(x+1)-3x(1-x)-(x+1)(x-1)-\left(4+2x^{2}\right)$;
 \item $\dfrac{1}{2}(x+1)+\dfrac{1}{4}(x+1)(x-1)-\left(x^{2}-1\right)$;
 \item $(3x+1)\left(\dfrac{5}{2}+x\right)-(2x-1)(2x+1)(x-2)+2x^{3}$.
 \end{enumeratea}
\end{esercizio}

\begin{esercizio}[\Ast]
Risolvi le seguenti espressioni con i polinomi.
 \begin{enumeratea}
 \item $\left(a-\dfrac{1}{2}b\right)a^{3}-\left(\dfrac{1}{3}{ab}-1\right)\left[2a^{2}(a-b)-a\left(a^{2}-2{ab}\right)\right]$;
 \item $\left(3x^2+6xy-4y^2\right)\left(\dfrac{1}{2}xy-\dfrac{2}{3}y^2\right)$;
 \item $(2a-3b)\left(\dfrac{5}{4}a^{2}+\dfrac{1}{2}{ab}-\dfrac{1}{6}b^{2}\right)-\dfrac{1}{6}a\left(12a^{2}-\dfrac{18}{5}b^{2}\right)+\dfrac{37}{30}ab^{2}-\dfrac{1}{2}a\left(a^{2}-\dfrac{11}{2}{ab}\right)$;
 \item $\dfrac{1}{3}xy\left[\left(x-y^{2}\right)\left(x^{2}-\dfrac{1}{2}y\right)-3x\left(-{\dfrac{1}{9}xy}\right)\left(3y\right)\right]-\dfrac{1}{3}x\left(x^{3}y+\dfrac{1}{4}xy^{2}\right)$.
 \end{enumeratea}
\end{esercizio}

\begin{esercizio}[\Ast]
Risolvi la seguente espressione con i polinomi.
\begin{multline*}
\dfrac{1}{2}x\left[\left(x-y^{2}\right)\left(x^{2}+\dfrac{1}{2}y\right)-5x\left(-{\dfrac{1}{10}}{xy}\right)(4y)\right]-\dfrac{1}{2}x\left(x^{3}y+\dfrac{1}{2}xy^{2}\right)+\\
-\dfrac{1}{2}x^{2}\left(x^{2}+\dfrac{1}{2}y+{xy}^{2}\right)+\dfrac{1}{4}{xy}\left(y^{2}+2x^{3}+{xy}\right).
\end{multline*}
\end{esercizio}

\begin{esercizio}[\Ast]
Risolvi la seguente espressione con i polinomi.
\begin{multline*}
\left(\dfrac{2}{3}a-2b\right)\left(\dfrac{3}{2}a+2b\right)\left(\dfrac{9}{4}a^{2}+4b^{2}\right)-\dfrac{3}{4}\left(\dfrac{9}{4}a^{2}\right)-a^{2}\left(\dfrac{9}{4}a^{2}-5b^{2}\right)+\\
+5{ab}\left(\dfrac{3}{4}a^{2}+\dfrac{4}{3}b^{2}\right).
\end{multline*}
\end{esercizio}

\begin{esercizio}[\Ast]
Risolvi la seguente espressione con i polinomi.
\begin{multline*}
\left(\dfrac{1}{2}x+2y\right)\left(\dfrac{1}{2}x-2y\right)\left(\dfrac{1}{4}x^{2}-4y^{2}\right)-\dfrac{1}{4}x\left(\dfrac{27}{4}x^{3}-\dfrac{61}{3}xy^{2}\right)+\\
-16\left(y^{4}+x^{4}\right)-\dfrac{37}{12}x^{2}y^{2}+\dfrac{141}{8}x^{4}.
\end{multline*}
\end{esercizio}

\begin{esercizio}[\Ast]
Risolvi la seguente espressione con i polinomi.
\begin{multline*}
x\left(\dfrac{2}{3}y^{2}-\dfrac{27}{8}x^{2}\right)-\left[-\left(\dfrac{3}{2}x-\dfrac{2}{3}y\right)\left(\dfrac{9}{4}x^{2}+xy+\dfrac{4}{3}y^{2}\right)+\dfrac{2}{3}x^{2}\left(\dfrac{9}{4}y^{2}+\dfrac{1}{3}y\right)\right]+\\
+\dfrac{2}{9}y\left(x^{2}+4y^{2}-9xy\right).
\end{multline*}
\end{esercizio}

\begin{esercizio}[\Ast]
Risolvi la seguente espressione con i polinomi.
\begin{multline*}
\left(\dfrac{1}{2}ab+\dfrac{2}{3}xy\right)\left(\dfrac{1}{2}ab-\dfrac{2}{3}xy\right)-\left[\left(\dfrac{1}{2}ab\right)^{2}-\left(\dfrac{2}{3}xy\right)^{2}\right]\left(\dfrac{1}{2}ax\right)+\dfrac{3}{2}ax\left(\dfrac{2}{3}a-\dfrac{2}{3}y\right)+\\
-x\left(\dfrac{1}{2}ax+\dfrac{3}{4}xy\right)-\dfrac{2}{9}x^{2}y^{2}(ax-2)+\dfrac{1}{4}a^{2}b^{2}\left(\dfrac{1}{2}ax-1\right)+\dfrac{3}{4}x^{2}\left(y+\dfrac{2}{3}a\right).
\end{multline*}
\end{esercizio}

\begin{esercizio}[\Ast]
Risolvi la seguente espressione con i polinomi.
\begin{multline*}
\dfrac{1}{6}ab-\dfrac{1}{3}a^{2}-\left\{\dfrac{3}{4}ab+\dfrac{1}{2}a\left[\dfrac{3}{2}b-\left(\dfrac{1}{6}a-\dfrac{4}{5}a\cdot {\dfrac{25}{3}a}\right)\left(-{\dfrac{2}{3}ab}\right)-\left(-{\dfrac{8}{3}ab}\right)\left(-{\dfrac{9}{8}b}\right)\right]\right\}+\\
+\dfrac{1}{3}a\left(a-5b-9a^{3}b+\dfrac{1}{6}a^{2}b\right).
\end{multline*}
\end{esercizio}

\begin{esercizio}[\Ast]
Risolvi la seguente espressione con i polinomi.
\begin{multline*}
\dfrac{1}{5}x^{2}+\left\{\left[2x-\left(\dfrac{3}{2}x^{2}y-\dfrac{7}{4}xy+\dfrac{1}{8}y^{3}\right):\left(-{\dfrac{1}{2}y}\right)\right] 2x-\dfrac{7}{10}xy\right\}\left(-{\dfrac{1}{6}x^{2}}\right)+\\
+x^{2}y-\dfrac{1}{3}x\left(\dfrac{3}{5}x\right)-x^{2}\left(y-x^{3}-\dfrac{1}{12}xy^{2}\right).
\end{multline*}
\end{esercizio}
\newpage
\begin{esercizio}
Se $A=x-1$, $B=2x+2$, $C=x^2-1$ determina
\begin{multicols}{3}
\begin{enumeratea}
\item $A+B+C$;
\item $A\cdot B-C$;
\item $A+B\cdot C$;
\item $A\cdot B\cdot C$;
\item $2AC-2BC$;
\item $(A+B)\cdot C$.
\end{enumeratea}
\end{multicols}
\end{esercizio}

\begin{esercizio}[\Ast]
 Operazioni tra polinomi con esponenti letterali.

\begin{enumeratea}
\item $\left(a^{n+1}-a^{n+2}+a^{n+3}\right):\left(a^{1+n}\right)$;
\item $\left(1+a^{n+1}\right)\left(1-a^{n-1}\right)$;
\item $\left(16a^{n+1}b^{n+2}-2a^{2n}b^{n+3}+5a^{n+2}b^{n+1}\right):\left(2a^{n}b^{n}\right)$;
\item $\left(a^{n+1}-a^{n+2}+a^{n+3}\right)\left(a^{n+1}-a^{n}\right)$;
\item $\left(a^{n}-a^{n+1}+a^{n+2}\right)\left(a^{n+1}-a^{n-1}\right)$;
\item $\left(a^{n}+a^{n+1}+a^{n+2}\right)\left(a^{n+1}-a^{n}\right)$;
\item $\left(a^{n+2}+a^{n+1}\right)\left(a^{n+1}+a^{n+2}\right)$;
\item $\left(1+a^{n+1}\right)\left(a^{n+1}-2\right)$;
\item $\left(a^{n+1}-a^{n}\right)\left(a^{n+1}+a^{n}\right)\left(a^{2n+2}+a^{2n}\right)$;
\item $\left(\dfrac{1}{2}x^{n}-\dfrac{3}{2}x^{2n}\right)\left(\dfrac{1}{3}x^{n}-\dfrac{1}{2}\right)-\left(\dfrac{1}{3}x^{n}-1\right)\left(x^{n}+x\right)$.
\end{enumeratea}
\end{esercizio}
\begin{multicols}{2}
\begin{esercizio}
 Se si raddoppiano i lati di un rettangolo, come varia il suo
perimetro?
\end{esercizio}

\begin{esercizio}
 Se si raddoppiano i lati di un triangolo rettangolo, come varia la sua
area?
\end{esercizio}

\begin{esercizio}
 Se si raddoppiano gli spigoli~$a$, $b$ e~$c$ di un parallelepipedo, come
varia il suo volume?
\end{esercizio}

\begin{esercizio}
 Come varia l'area di un cerchio se si triplica il suo
raggio?
\end{esercizio}

\begin{esercizio}
 Determinare l'area di un rettangolo avente come
dimensioni~$\frac{1}{2}a$ e~$\frac{3}{4}a^{2}b$.
\end{esercizio}

\begin{esercizio}
 Determinare la superficie laterale di un cilindro avente raggio di
base~$x^{2}y$ e altezza~$\frac{1}{5}{xy}^{2}$.
\end{esercizio}
\end{multicols}

\subsection{Risposte}
\begin{multicols}{2}
\paragraph{10.14.} d)~$-x^{2}+x+\frac{29}{15}a^{2}$,\protect\\ e)~$-{\frac{a^{2}}{2}}-\frac{7}{24}ab+\frac{5}{2}b$.
\paragraph{10.21.} a)~$-a$,\quad b)~$-9b$,\quad c)~$-18b$,\protect\\ d)~$6a^{2}-\frac{63}{2}a^{2}b$,\quad e)~$2x^2-9x-3$.
\paragraph{10.26.} a)~$a^{4}-\frac{1}{2}a^{3}b-\frac{1}{3}a^{4}b+a^{3}$,\protect\\ b)~$\frac{3}{2}x^{3}y+x^{2}y^{2}-6{xy}^{3}+\frac{8}{3}y^{4}$,\protect\\ c)~$\frac{1}{2}b^{3}$,\quad d)~$\frac{1}{6}xy^{4}-\frac{1}{4}x^{2}y^{2}$.
\paragraph{10.27.} $0$.
\paragraph{10.28.} $-16b^{4}-\frac{27}{16}a^{2}$.
\paragraph{10.29.} $0$.
\paragraph{10.30.} $-\frac{3}{2}x^{2}y^{2}$.
\paragraph{10.31.} $a^{2}x-axy$.
\paragraph{10.32.} $-\frac{7}{9}a^{4}b+\frac{3}{2}a^2b^2-3ab$.
\paragraph{10.33.} $\frac{1}{2}x^{4}+\frac{7}{60}x^{3}y$.
\paragraph{10.35.} a)~$1-a+a^{2}$,\protect\\ b)~$1-a^{n-1}+a^{n+1}-a^{2}n$,\protect\\ c)~$8ab^2-a^nb^3+\frac{5}{2}a^2b$, \protect \\
d)~$a^{2n+4}-2a^{2n+3}+2a^{2n+2}-a^{2n+1}$,\protect\\ e)~$a^{2n+3}-a^{2n+2}-a^{2n-1}+a^{2n}$,\protect\\ f)~$-a^{2}n+a^{2n+3}$,\quad
\protect\\ g)~$a^{2n+4}+2a^{2n+3}+a^{2n+2}$,\protect\\ i)~$a^{2n+2}-a^{n+1}-2$,\quad h)~$a^{4n+4}-a^{4n}$,\protect\\
j)~$\frac{7}{12}x^{2n}+\frac{3}{4}x^{n}-\frac{1}{2}x^{3n}-\frac{1}{3}x^{n+1}+x$.
\end{multicols}

\cleardoublepage
