% (c) 2012 -2014 Dimitrios Vrettos - d.vrettos@gmail.com
\chapter{Generalità sugli insiemi}
\section{Insiemi ed elementi}

In matematica usiamo la parola \textit{insieme} per indicare un
raggruppamento, una collezione, una raccolta di oggetti, individui,
simboli, numeri, figure che sono detti \textit{elementi}
dell'insieme e che sono ben definiti e distinti tra di
loro.

La nozione di insieme e quella di elemento di un insieme in matematica
sono considerate nozioni primitive, nozioni che si preferisce non
definire mediante altre più semplici.

\begin{exrig}
 \begin{esempio}
 Sono insiemi:
 \begin{enumeratea}
  \item l'insieme delle lettere della parola RUOTA;
  \item l'insieme delle canzoni che ho ascoltato la settimana scorsa;
  \item l'insieme delle città della Puglia con più di~$\np{15000}$ abitanti;
  \item l'insieme delle lettere dell'alfabeto italiano;
  \item l'insieme dei numeri~1, 2, 3, 4, 5;
  \item l'insieme delle montagne d'Italia più alte di~$\np{1000}$ metri.
 \end{enumeratea}
 \end{esempio}
\end{exrig}

Per poter assegnare un insieme occorre soddisfare le seguenti condizioni:

\begin{itemize*}
\item bisogna poter stabilire con certezza e oggettività se un oggetto
è o non è un elemento dell'insieme;
\item gli elementi di uno stesso insieme devono essere differenti tra
loro, cioè un elemento non può essere ripetuto più volte nello stesso insieme.
\end{itemize*}

Non possono essere considerati insiemi:
\begin{itemize*}
 \item i film interessanti (non c'è un criterio oggettivo per stabilire se un film è interessante oppure no, uno stesso film
può risultare interessante per alcune persone e non interessante per altre);
 \item le ragazze simpatiche di una classe (non possiamo stabilire in maniera oggettiva se una ragazza è simpatica);
 \item le montagne più alte d'Italia (non possiamo dire se una montagna è tra le più alte poiché non è fissata
un'altezza limite);
 \item l'insieme delle grandi città d'Europa (non c'è un criterio per
stabilire se una città è grande);
\end{itemize*}

\ovalbox{\risolvi \ref{ese:5.1}}\vspazio


In generale, gli insiemi si indicano con lettere maiuscole~$A$, $B$, $C$, \ldots e
gli elementi con lettere minuscole~$a$, $b$, $c$, \ldots

Se un elemento~$a$ sta nell'insieme~$A$ si scrive~$a\in A$ e si legge ``$a$ appartiene ad~$A$''.
Il simbolo ``$\in$'' si chiama simbolo di \textit{appartenenza}.

Se un elemento~$b$ non sta nell'insieme~$A$ si scrive~$b\notin A$ e
si legge ``$b$ non appartiene ad~$A$''. Il simbolo ``$\notin$''
si chiama simbolo di \textit{non appartenenza}.

Il criterio che stabilisce se un elemento appartiene a un insieme si chiama \textit{proprietà caratteristica} dell'insieme.

Un altro modo per definire un insieme, oltre a quello di indicare la sua proprietà caratteristica, è quello di elencare i suoi elementi separati da virgole e racchiusi tra parentesi graffe. Ad esempio: $A=\{a\text{, }b\text{, }c\text{, }d\}$.

Per indicare alcuni insiemi specifici vengono utilizzati simboli particolari:
\begin{itemize*}
 \item $\insN$ si utilizza per indicare l'insieme dei numeri naturali:~$\insN=\{0\text{, }1\text{, }2\text{, }3\text{, }\ldots\}$;
 \item $\insZ$ si utilizza per indicare i numeri interi relativi:~$\insZ=\{\ldots\text{, }-2\text{, }-1\text{, }0\text{, }+1\text{, }+2\text{, }\ldots\}$;
 \item $\insQ$ si utilizza per indicare i numeri razionali:~$\insQ=\{\dfrac{1}{2}\text{, }-\dfrac{3}{5}\text{, }\dfrac{5}{1}\text{, }-\dfrac{4}{17}\text{, }\np{12,34}\text{, }8\text{, }\np{0,}\overline{25}\text{, }\ldots\}$.
 \end{itemize*}


 \begin{exrig}
 \begin{esempio}
Indica con il simbolo opportuno quali dei seguenti elementi appartengono
o non appartengono all'insieme $A$ dei giorni della
settimana: lunedì, martedì, gennaio, giovedì, dicembre, estate.

Gennaio e dicembre sono mesi dell'anno, perciò scriviamo:
\[\text{lunedì}\in A\text{,\:}\text{martedì}\in A\text{,\:}\text{gennaio}\notin A\text{,\:}\text{giovedì}\in A\text{,\:}\text{dicembre}\notin A\text{,\:}\text{estate}\notin A.\]
 \end{esempio}
\end{exrig}

Consideriamo l'insieme~$A=\{\text{r, s, t}\}$ e l'insieme~$B$ delle consonanti della parola
``risate''. Possiamo osservare che~$A$ e~$B$ sono due insiemi costituiti dagli stessi
elementi; diremo che sono \textit{insiemi uguali}.

\begin{definizione}
 Due insiemi~$A$ e~$B$ si dicono \emph{uguali} se sono formati dagli stessi elementi, anche se
disposti in ordine diverso. In simboli si scrive~$A=B$. Altrimenti i due insiemi si dicono \emph{diversi}, in simboli~$A\neq B$.
\end{definizione}

\ovalbox{\risolvii \ref{ese:5.2}, \ref{ese:5.3}, \ref{ese:5.4}, \ref{ese:5.5}, \ref{ese:5.6}, \ref{ese:5.7}, \ref{ese:5.8}, \ref{ese:5.9}}

\section{Insieme vuoto, insieme universo, cardinalità}\label{sect:universo}

Consideriamo l'insieme~$A=\{\text{consonanti della parola ``AIA''}\}$. Poiché la parola ``AIA'' non
contiene consonanti, l'insieme~$A$ è privo di elementi.

\begin{definizione}
 Un insieme privo di elementi si chiama \emph{insieme vuoto} e lo si indica con il simbolo~$\emptyset$ o $\{ \}$.
\end{definizione}

\osservazione $\{ \}=\emptyset$ ma~$\{\emptyset \}\neq\emptyset$ dato che la scrittura~$\{\emptyset \}$ rappresenta un insieme che ha
come unico elemento l'insieme vuoto.

\begin{exrig}
 \begin{esempio}
 Alcuni insiemi vuoti.
 \begin{enumeratea}
  \spazielenx
\item L'insieme dei numeri negativi maggiori di~5 è vuoto;
\item l'insieme delle capitali europee con meno di~50 abitanti è vuoto;
\item l'insieme dei numeri naturali minori di~0 è vuoto.
 \end{enumeratea}
 \end{esempio}
\end{exrig}

La frase <<l'insieme degli studenti che vengono a scuola con il motorino>> non definisce un
insieme particolare. Occorre definire il contesto, l'ambiente che fa individuare gli elementi
dell'insieme. Se l'ambiente è la classe~1\textsuperscript{a}C gli elementi considerati saranno certamente diversi, e probabilmente meno numerosi, di quelli che compongono l'ambiente di un'intera scuola o di un'intera
città. Quando si identifica un insieme, occorre indicare anche l'ambiente di riferimento da cui trarre gli elementi che appartengono al nostro insieme. Questo insieme si chiama \emph{insieme universo} e rappresenta il contesto, l'ambiente su cui faremo le nostre osservazioni. In generale l'insieme universo per un insieme~$A$ è semplicemente un insieme che contiene~$A$. Solitamente l'insieme universo viene indicato con~$U$.

\subsection{Cardinalità}

\begin{definizione}
 Si definisce \emph{cardinalità} (o \emph{potenza}) di un insieme finito il numero
degli elementi dell'insieme. Essa viene indicata con uno dei seguenti simboli~$\valass{A}$, \#($A$) o~$\card(A)$.
\end{definizione}

Per poter parlare di cardinalità di un insieme qualsiasi, che
comprenda anche insiemi infiniti come gli insiemi numerici, occorre una
definizione più complessa che qui non daremo.

\begin{exrig}
 \begin{esempio}
 Esempi di cardinalità.
 \begin{enumeratea}
  \item L'insieme~$A$ delle vocali dell'alfabeto italiano ha~5 elementi, quindi~$\card(A)=5$;
  \item l'insieme~$B$ dei multipli di~3 minori di~10 ha~3 elementi, quindi~$\card(B)=3$.
 \end{enumeratea}
 \end{esempio}
\end{exrig}

\ovalbox{\risolvii \ref{ese:5.10}, \ref{ese:5.11}, \ref{ese:5.12}, \ref{ese:5.13}, \ref{ese:5.14}, \ref{ese:5.15}}

\newpage
% (c) 2012 Dimitrios Vrettos - d.vrettos@gmail.com
% (c) 2012 Claudio Carboncini - claudio.carboncini@gmail.com
\section{Esercizi}
\subsection{Esercizi dei singoli paragrafi}
\subsubsection*{\thechapter.1 - Insiemi ed elementi}

\begin{esercizio}
 \label{ese:5.1}
 Barra con una crocetta i raggruppamenti che ritieni siano degli insiemi.
 \begin{multicols}{2}
 \begin{enumeratea}
\item I fiumi più lunghi d'Italia;
\item le persone con più di~30 anni;
\item i numeri~1, 20, 39, 43, 52;
\item i libri più pesanti nella tua cartella;
\item i punti di una retta;
\item gli animali con~2 zampe;
\item le vocali dell'alfabeto italiano;
\item i professori bravi;
\item i gatti con due code;
\item i calciatori che hanno fatto pochi gol.
\end{enumeratea}
\end{multicols}
\end{esercizio}

%%%%%%%%%%%%%%%%%%%%%%%%%%%%%%%%%%%%%%%%%%%%%%%%%%%%%%%%%%%%%%%%%%%%%
\begin{esercizio}
 \label{ese:5.2}
Considerando l'insieme $A$ delle lettere dell'alfabeto italiano, per ciascuno dei seguenti casi inserisci il simbolo adatto fra ``$\in$'' e ``$\notin$''.

b \ldots $A$, i \ldots $A$, j \ldots $A$, e \ldots $A$, w \ldots $A$, z \ldots $A$.
\end{esercizio}

\begin{esercizio}
\label{ese:5.3}
Le vocali delle parole che seguono formano insiemi uguali, tranne in un caso. Quale?
\begin{center}
 \boxA\quad sito\quad\boxB\quad micio\quad\boxC\quad zitto\quad\boxD\quad fiocco\quad\boxE\quad lecito\quad\boxF\quad dito.
\end{center}
\end{esercizio}

\begin{esercizio}
\label{ese:5.4}
Individua tra i seguenti insiemi quelli che sono uguali:
\begin{multicols}{2}
\begin{enumeratea}
 \item vocali della parola ``SASSO'';
 \item consonanti della parola ``SASSO'';
 \item vocali della parola ``PIETRA'';
 \item vocali della parola ``PASSO''.
\end{enumeratea}
\end{multicols}
\end{esercizio}

\begin{esercizio}
\label{ese:5.5}
Quali delle seguenti frasi rappresentano criteri oggettivi per individuare un insieme? Spiega perché.
\TabPositions{8.5cm}
\begin{enumeratea}
\item Le città che distano meno di~100 km da Lecce; \tab\boxV\quad\boxF
\item i laghi d'Italia;  \tab\boxV\quad\boxF
\item le città vicine a Roma; \tab\boxV\quad\boxF
\item i calciatori della Juventus;  \tab\boxV\quad\boxF
\item i libri di Mauro;  \tab\boxV\quad\boxF
\item i professori bassi della tua scuola;  \tab\boxV\quad\boxF
\item i tuoi compagni di scuola il cui nome inizia per M; \tab\boxV\quad\boxF
\item i tuoi compagni di classe che sono gentili; \tab\boxV\quad\boxF
\item gli zaini neri della tua classe.  \tab\boxV\quad\boxF
\end{enumeratea}
\end{esercizio}

\begin{esercizio}
\label{ese:5.6}
Scrivi al posto dei puntini il simbolo mancante tra ``$\in$'' e ``$\notin$''.

\begin{enumeratea}
\item La Polo \ldots\ldots all'insieme delle automobili Fiat;
\item il cane \ldots\ldots all'insieme degli animali domestici;
\item la Puglia \ldots\ldots all'insieme delle regioni italiane;
\item Firenze \ldots\ldots all'insieme delle città francesi;
\item il numero~10 \ldots\ldots all'insieme dei numeri naturali;
\item il numero~3 \ldots\ldots all'insieme dei numeri pari.
\end{enumeratea}
\end{esercizio}
\pagebreak
\begin{esercizio}
\label{ese:5.7}
Quali delle seguenti proprietà sono caratteristiche per un insieme?
\TabPositions{8.5cm}
\begin{enumeratea}
\item Essere una città italiana il cui nome inizia per W; \tab\boxV\quad\boxF
\item essere un bravo cantante; \tab\boxV\quad\boxF
\item essere un monte delle Alpi;  \tab\boxV\quad\boxF
\item essere un ragazzo felice; \tab\boxV\quad\boxF
\item essere un numero naturale grande;\tab\boxV\quad\boxF
\item essere un ragazzo nato nel~1985; \tab\boxV\quad\boxF
\item essere un alunno della classe~1\textsuperscript{a}C; \tab\boxV\quad\boxF
\item essere una lettera dell'alfabeto inglese; \tab\boxV\quad\boxF
\item essere una retta del piano; \tab\boxV\quad\boxF
\item essere un libro interessante della biblioteca; \tab\boxV\quad\boxF
\item essere un italiano vivente nato nel~1850; \tab\boxV\quad\boxF
\item essere un italiano colto. \tab\boxV\quad\boxF
\end{enumeratea}
\end{esercizio}

\begin{esercizio}
\label{ese:5.8}
Scrivi al posto dei puntini il simbolo mancante tra ``$=$'' e ``${\neq}$''.
\begin{enumeratea}
\item L'insieme delle lettere della parola ``CANE'' e della parola ``PANE'' sono \ldots\ldots;
\item l'insieme delle vocali della parola ``INSIEME'' e della parola ``MIELE'' sono \ldots\ldots;
\item l'insieme delle consonanti della parola ``LETTO'' e della parola ``TETTO'' sono \ldots\ldots;
\item l'insieme delle lettere della parola ``CONTRO'' e della parola ``TRONCO'' sono \ldots\ldots;
\item l'insieme delle vocali della parola ``LIBRO'' e della parola ``MINISTRO'' sono \ldots\ldots;
\item l'insieme delle vocali della parola ``DIARIO'' e della parola ``RAMO'' sono \ldots\ldots;
\item l'insieme delle lettere della parola ``MOUSE'' e della parola ``MUSEO'' sono \ldots\ldots;
\item l'insieme delle consonanti della parola ``SEDIA'' e della parola ``ADESSO'' sono \ldots\ldots;
\item l'insieme dei numeri pari minori di~5 e l'insieme vuoto sono \ldots\ldots;
\item l'insieme dei numeri pari e l'insieme dei multipli di~2 sono \ldots\ldots
\end{enumeratea}
\end{esercizio}

\begin{esercizio}
\label{ese:5.9}
Le stelle dell'universo formano un insieme. Le stelle visibili a occhio nudo formano un insieme? Spiega il tuo punto di vista.
\end{esercizio}

\subsubsection*{\thechapter.2 - Insieme vuoto, insieme universo, cardinalità}
\begin{esercizio}
\label{ese:5.10}
Indica se gli insiemi~$G =\{\text{gatti con~6 zampe}\}$ e~$P = \{\text{polli con~2 zampe}\}$ sono o non sono vuoti.
\end{esercizio}

\begin{esercizio}
\label{ese:5.11}
Barra con una croce gli insiemi vuoti.
\begin{enumeratea}
 \item L'insieme dei numeri positivi minori di~0;
 \item l'insieme dei numeri negativi minori di~100;
 \item l'insieme dei numeri pari minori di~100;
 \item l'insieme delle capitali europee della regione Lombardia;
 \item l'insieme dei triangoli con quattro angoli;
 \item l'insieme delle capitali italiane del Lazio;
 \item l'insieme dei punti di intersezione di due rette parallele.
 \end{enumeratea}
\end{esercizio}

\begin{esercizio}
\label{ese:5.12}
Quali delle seguenti scritture sono corrette per indicare
l'insieme vuoto?
\begin{center}
 \boxA\quad~$\emptyset $ \quad\boxB\quad~0 \quad\boxC\quad~$\{\emptyset \}$ \quad\boxD\quad~$\{0\}$ \quad\boxE\quad \{ \}.
\end{center}
\end{esercizio}
\pagebreak
\begin{esercizio}
\label{ese:5.13}
Quali dei seguenti insiemi sono vuoti? Per gli insiemi non vuoti indica la cardinalità.
\begin{enumeratea}
\item L'insieme degli uccelli con~6 ali;
\item l'insieme delle lettere della parola ``VOLPE'';
\item l'insieme dei cani con~5 zampe;
\item l'insieme delle vocali della parola ``COCCODRILLO'';
\item l'insieme delle vocali dell'alfabeto italiano;
\item l'insieme degli abitanti della luna;
\item l'insieme dei numeri sulla tastiera del telefonino.
\end{enumeratea}
\end{esercizio}

\begin{esercizio}
\label{ese:5.14}
Scrivi per ciascun insieme un possibile insieme universo.
\begin{enumeratea}
\item l'insieme dei rettangoli;
\item l'insieme dei multipli di~3;
\item l'insieme delle lettere della parola ``MATEMATICA'';
\item l'insieme dei libri di matematica;
\item l'insieme dei ragazzi che hanno avuto un'insufficienza in matematica.
\end{enumeratea}
\end{esercizio}

\begin{esercizio}
\label{ese:5.15}
Dato l'insieme~$A = \{\text{0, 2, 5}\}$ determina se le seguenti affermazioni sono vere o false.
\TabPositions{2.5cm}
\begin{enumeratea}
\item $0\in A$. \tab\boxV\quad\boxF
\item $5\in A$. \tab\boxV\quad\boxF
\item $\emptyset \in A$. \tab\boxV\quad\boxF
\item $A\in A$. \tab\boxV\quad\boxF
\item $\np{3,5}\in A$. \tab\boxV\quad\boxF
\end{enumeratea}
\end{esercizio}

\cleardoublepage
