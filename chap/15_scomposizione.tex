% (c) 2012 Claudio Carboncini - claudio.carboncini@gmail.com
% (c) 2012-2013 Dimitrios Vrettos - d.vrettos@gmail.com

\chapter{Scomposizione in fattori}

\section{Cosa vuol dire scomporre in fattori}
Scomporre un polinomio in fattori significa scrivere il polinomio come il prodotto di polinomi e monomi che
moltiplicati tra loro danno come risultato il polinomio stesso. Si può paragonare la scomposizione in fattori
di un polinomio alla scomposizione in fattori dei numeri naturali.

\begin{wrapfloat}{figure}{r}{0pt}
 % (c) 2012 Dimitrios Vrettos - d.vrettos@gmail.com
\begin{tikzpicture}[font=\small]
\matrix(a) [matrix of nodes, anchor=west, minimum width=4.5mm]{
36&2\\
18&2\\
9&3\\
3&3\\
1&{}\\};

\draw (a-1-2.north west)--(a-5-2.south west);
\end{tikzpicture}

\end{wrapfloat}

Per esempio, scomporre il numero~$36$ significa scriverlo come~$2^{2}\cdot 3^{2}$ dove~$2$ e~$3$ sono i suoi fattori primi.
Anche~$36 = 9 \cdot 4$ è una scomposizione, ma non è in fattori primi. Allo stesso modo un polinomio va scomposto in fattori non ulteriormente
scomponibili che si chiamano irriducibili. %(questa dovrebbe andare a sinistra della tabella)

Il polinomio~$3a^{3}b^{2}-3ab^{4}$ si può scomporre in fattori in questo modo \[3ab^{2}(a-b)(a+b)\text{,}\] infatti eseguendo i prodotti si
ottiene \[3ab^{2}(a-b)(a+b)=3ab^{2}(a^{2}+ab-ba-b^{2})=3ab^{2}\left(a^{2}-b^{2}\right)=3a^{3}b^{2}-3ab^{4}.\]
La scomposizione termina quando non è possibile scomporre ulteriormente i fattori individuati.
Come per i numeri la scomposizione in fattori dei polinomi identifica il polinomio in maniera univoca (a meno di multipli).

\begin{definizione}
Un polinomio si dice \emph{riducibile} (scomponibile) se può essere scritto come prodotto di due o più polinomi (detti fattori) di grado maggiore di zero.
In caso contrario esso si dirà \emph{irriducibile}.
\end{definizione}

La caratteristica di un polinomio di essere irriducibile dipende dall'insieme numerico al quale appartengono i coefficienti del polinomio;
uno stesso polinomio può essere irriducibile nell'insieme dei numeri razionali, ma riducibile in quello dei numeri reali o
ancora in quello dei complessi.
Dalla definizione consegue che un polinomio di primo grado è irriducibile.

\begin{definizione}
La \emph{scomposizione in fattori di un polinomio} è la sua scrittura come prodotto di fattori irriducibili.
\end{definizione}
\ovalbox{\risolvi \ref{ese:15.1}}

\section{Raccoglimento totale a fattore comune}
Questo è il primo metodo che si deve cercare di utilizzare per scomporre un polinomio.

Il metodo si basa sulla proprietà distributiva della moltiplicazione rispetto all'addizione.

Prendiamo in considerazione il seguente prodotto:~$a(x+y+z)=ax+ay+az$.
Il nostro obiettivo è ora quello di procedere
da destra verso sinistra, cioè avendo il polinomio~$ax+ay+az$ come possiamo fare per individuare il prodotto che lo ha generato?
In questo caso semplice possiamo osservare che i tre monomi contengono tutti la lettera~$a$, che quindi si può mettere in comune,
o come anche si dice ``in evidenza''.
Perciò scriviamo \[ax+ay+az=a(x+y+z).\]

\begin{exrig}
 \begin{esempio}
Analizziamo la scomposizione in fattori~$3a^{2}b\left(2a^{3}-5b^{2}-7c\right)$.
 \begin{equation*}
   \begin{split}
    3a^{2}b\left(2a^{3}-5b^{2}-7c\right) &=3a^{2}b(2a^{3})+3a^{2}b(-5b^{2})+3a^{2}b(-7c)\\
    &=6a^{5}b-15a^{2}b^{3}-21a^{2}bc.
   \end{split}
 \end{equation*}
L'ultima uguaglianza, letta da destra verso sinistra, è il raccoglimento totale a fattore comune.
Partendo da~$6a^{5}b-15a^{2}b^{3}-21a^{2}bc$ possiamo notare che i coefficienti numerici~$6$, $15$ e~$21$ hanno il~$3$ come fattore in comune.
Notiamo anche che la lettera~$a$ è in comune, come la lettera~$b$. Raccogliendo tutti i fattori comuni si avrà il prodotto
$3a^{2}b\left(2a^{3}-5b^{2}-7c\right)$ di partenza.
 \end{esempio}
\end{exrig}

\begin{procedura}
Mettere in evidenza il fattore comune:
\begin{enumeratea}
\item trovare il~$\mcd$ di tutti i termini che formano il polinomio: tutti i fattori in comune con l'esponente minimo con cui compaiono;
\item scrivere il polinomio come prodotto del~$\mcd$ per il polinomio ottenuto dividendo ciascun monomio del polinomio di partenza per il~$\mcd$;
\item verificare la scomposizione eseguendo la moltiplicazione per vedere se il prodotto dà come risultato il polinomio da scomporre.
\end{enumeratea}
\end{procedura}

\begin{exrig}
 \begin{esempio}
Scomporre in fattori~$5a^{2}x^{2}-10ax^{5}$.
  \begin{enumeratea}
  \item Tra i coefficienti numerici il fattore comune è~$5$;
  \item tra la parte letterale sono in comune le lettere~$a$ e~$x$, la~$a$ con esponente~$1$, la~$x$ con esponente~$2$;
  \item pertanto il~$\mcd$ è~$5ax^{2}$;
  \item passiamo quindi a scrivere~$5a^{2}x^{2}-10ax^{5}=5ax^{2}(\ldots \ldots \ldots)$;
  \item nella parentesi vanno i monomi che si ottengono dalle divisioni~$5a^{2}x^{2}:5ax^{2}=a$ e~$-10ax^{5}:5ax^{2}=-2x^{3}$.
  \end{enumeratea}
  In definitiva~$5a^{2}x^{2}-10ax^{5}=5ax^{2}(a-2x^{3})$.
 \end{esempio}

 \begin{esempio}
Scomporre in fattori~$10x^{5}y^{3}z-15x^{3}y^{5}z-20x^{2}y^{3}z^{2}$.
 \begin{enumeratea}
 \item Trovo tutti i fattori comuni con l'esponente minore per formare il~$\mcd$.\, $\mcd=5x^{2}y^{3}z$;
 \item divido ciascun termine del polinomio per~$5x^{2}y^{3}z$:
   \begin{itemize*}
   \item $10x^{5}y^{3}z:5x^{2}y^{3}z=2x^{3}$;
   \item $-15x^{3}y^{5}z:5x^{2}y^{3}z=-3xy^{2}$;
   \item $-20x^{2}y^{3}z^{2}:5x^{2}y^{3}z=-4z$;
   \end{itemize*}
 \item il polinomio si può allora scrivere come~$5x^{2}y^{3}z (2x^{3}-3xy^{2}-4z)$;
 \item Il fattore da raccogliere a fattore comune può essere scelto con il segno~$+$ (positivo) o con il segno~$−$ (negativo).
    Nell'esempio precedente è valida anche la seguente 
    scomposizione:~$10x^{5}y^{3}z-15x^{3}y^{5}z-20x^{2}y^{3}z^{2}=-5x^{2}y^{3}z (-2x^{3}+3xy^{2}+4z)$.
 \end{enumeratea}
 \end{esempio}

 \begin{esempio}
Scomporre in fattori~$-8x^{2}y^{3}+10x^{3}y^{2}$.
 \begin{enumeratea}
  \item Poiché il primo termine è negativo possiamo mettere a fattore comune un 
  numero negativo. Tra~8 e~10 il~$\mcd$ è~2. Tra~$x^{2}y^{3}$ e~$x^{3}y^{2}$ mettiamo 
  a fattore comune le lettere~$x$ e~$y$, entrambe con esponente~$2$, perché è
     il minimo esponente con cui compaiono. In definitiva il monomio da mettere a fattore comune è~$-2x^{2}y^{2}$;
  \item pertanto possiamo cominciare a scrivere~$-2x^{2}y^{2}(\ldots \ldots \ldots)$;
  \item eseguiamo le divisioni~$-8x^{2}y^{3}:(-2x^{2}y^{2})=+4y$ e  $10x^{3}y^{2}:(-2x^{2}y^{2})=-5x$. 
  I quozienti trovati~$+4y$ e~$-5x$ vanno nelle parentesi.
 \end{enumeratea}
 In definitiva~$-8x^{2}y^{3}+10x^{3}y^{2}=-2x^{2}y^{2}(4y-5x)$.
 \end{esempio}

 \begin{esempio}
Scomporre in fattori~$6a(x-1)+7b(x-1)$.
  \begin{enumeratea}
  \item Il fattore comune è~$(x-1)$, quindi il polinomio si può scrivere come~$(x-1)\cdot [\ldots \ldots \ldots]$;
  \item nella parentesi quadra scriviamo i termini che si ottengono dalle divisioni:
   \begin{itemize*}
    \item $6a(x-1):(x-1)=6a$;
    \item $7b(x-1):(x-1)=7b$.
   \end{itemize*}
  \end{enumeratea}
  In definitiva~$6a(x-1)+7b(x-1)=(x-1)(6a+7b)$.
 \end{esempio}

 \begin{esempio}
Scomporre in fattori~$10(x+1)^{2}-5a(x+1)$.
  \begin{enumeratea}
  \item Il fattore comune è~$5(x+1)$, quindi possiamo cominciare a scrivere~$5(x+1)\cdot [\ldots \ldots \ldots]$;
  \item nella parentesi quadra scriviamo i termini che si ottengono dalle divisioni:
   \begin{itemize*}
    \item $10(x+1)^{2}:5(x+1)=2(x+1)$;
    \item $-5a(x+1):5(x+1)=a$.
   \end{itemize*}
  \end{enumeratea}
  In definitiva~$10(x+1)^{2}-5a(x+1)=5(x+1)\bigl[2(x+1)-a \bigr]$.
 \end{esempio}

\end{exrig}

\ovalbox{\risolvii \ref{ese:15.2}, \ref{ese:15.3}, \ref{ese:15.4}, \ref{ese:15.5}, \ref{ese:15.6}, \ref{ese:15.7}, \ref{ese:15.8}, \ref{ese:15.9}, \ref{ese:15.10},\ref{ese:15.11}, \ref{ese:15.12}, \ref{ese:15.13}}

\vspazio\ovalbox{\ref{ese:15.14}}

\section{Raccoglimento parziale a fattore comune}

Quando un polinomio non ha alcun fattore comune a tutti i suoi termini, possiamo provare a mettere in evidenza tra gruppi di monomi
e successivamente individuare il polinomio in comune.

Osserviamo il prodotto~$(a+b)(x+y+z)=ax+ay+az+bx+by+bz$. Supponiamo ora di avere il polinomio
$ax+ay+az+bx+by+bz$ come possiamo fare a tornare indietro per scriverlo come prodotto di polinomi?

%\newpage
\begin{exrig}
 \begin{esempio}
Scomponiamo in fattori~$ax+ay+az+bx+by+bz$. Non c'è nessun fattore comune a tutto il polinomio.

Proviamo a mettere in evidenza per gruppi di termini. Evidenziamo~$a$ tra i primi tre termini e~$b$ tra gli ultimi tre, 
avremo:~$a(x+y+z)+b(x+y+z)$. Ora risulta semplice vedere che il trinomio~$(x+y+z)$ è in comune e quindi lo possiamo mettere 
in evidenza~$ax+ay+az+bx+by+bz=a(x+y+z)+b(x+y+z)=(x+y+z)(a+b)$.
 \end{esempio}
\end{exrig}

\begin{procedura}
Eseguire il raccoglimento parziale.
\begin{enumeratea}
\item Dopo aver verificato che non è possibile effettuare un raccoglimento a fattore comune totale raggruppo i monomi
   in modo che in ogni gruppo sia possibile mettere in comune qualche fattore;
\item verifico se la nuova scrittura del polinomio ha un polinomio (binomio, trinomio, \ldots) comune a tutti i termini;
\item se è presente il fattore comune a tutti i termini lo metto in evidenza;
\item se il fattore comune non è presente la scomposizione è fallita, allora posso provare a raggruppare
   diversamente i monomi o abbandonare questo metodo.
\end{enumeratea}
\end{procedura}

\begin{exrig}
 \begin{esempio}
Scomporre in fattori~$ax+ay+bx+ab$.
  \begin{enumeratea}
  \item Provo a mettere in evidenza la~$a$ nel primo e secondo termine e la~$b$ nel terzo e quarto termine:~$ax+ay+bx+ab=a(x+y)+b(x+a)$;
  \item in questo caso non c'è nessun fattore comune: il metodo è fallito. In effetti il polinomio non si può scomporre in fattori.
  \end{enumeratea}
 \end{esempio}

 \begin{esempio}
Scomporre in fattori~$bx-2ab+2ax-4a^{2}$.
 \begin{enumeratea}
 \item Non vi sono fattori da mettere a fattore comune totale, proviamo con il raccoglimento parziale:~$b$ nei primi due monomi e~$2a$ negli altri due;
 \item $\mmevid{ev_rosso}{bx}-\mmevid{ev_rosso}{2ab}+\mmevid{ev_verde}{2ax}-\mmevid{ev_verde}{4a^{2}}=b(\mmevid{ev_blu}{x-2a})+2a(\mmevid{ev_blu}{x-2a})=(x-2a)(b+2a)$.
%\underline{bx} -\underline{2ab}+\underline{\underline{2ax}}-\underline{\underline{4a^{2}}}=b(\underline{x-2a})+2a(\underline{x-2a})=(x-2a)(b+2a)$.
 \end{enumeratea}
 \end{esempio}

 \begin{esempio}
Scomporre in fattori~$bx^{3}+2x^{2}-bx-2+abx+2a$.
 \begin{enumeratea}

 \item Raggruppiamo nel seguente modo:~$\mmevid{ev_rosso}{bx^{3}}+\mmevid{ev_verde}{2x^{2}}-\mmevid{ev_rosso}{bx}-\mmevid{ev_verde}{2}+\mmevid{ev_rosso}{abx}+\mmevid{ev_verde}{2a}$ tra quelli evidenziati in \evid{ev_rosso}{\phantom{I}} mettiamo a fattore comune~$bx$ e tra quelli evidenziati in \evid{ev_verde}{\phantom{I}} mettiamo a fattore comune~$2$;

% \item Raggruppiamo nel seguente modo:~$\underline{bx^{3}}+\underline{\underline {2x^{2}}}-\underline{bx}-\underline{\underline~2}
%     +\underline{abx}+\underline{\underline{2a}}$ tra quelli con sottolineatura semplice metto a fattore comune~$bx$, tra quelli
%     con doppia sottolineatura metto a fattore comune~$2$;

 \item $\mmevid{ev_rosso}{bx^{3}}+\mmevid{ev_verde}{2x^{2}}-\mmevid{ev_rosso}{bx}-\mmevid{ev_verde}{2}+\mmevid{ev_rosso}{abx}+\mmevid{ev_verde}{2a}=bx\bigl(\mmevid{ev_blu}{x^{2}-1+a}\bigr)+2\bigl(\mmevid{ev_blu}{x^{2}-1+a}\bigr)=\bigl(x^{2}-1+a\bigr)\bigl(bx+2\bigr)$.

%  \item $\underline{bx^{3}}+\underline{\underline {2x^{2}}}-\underline{bx}-\underline{\underline~2}+\underline{abx}+\underline{\underline{2a}}
%     =bx\bigl(\underline{x^{2}-1+a}\bigr)+2\bigl(\underline{x^{2}-1+a}\bigr)=\bigl(x^{2}-1+a\bigr)\bigl(bx+2\bigr)$.
 \end{enumeratea}
 \end{esempio}

 \begin{esempio}
Scomporre in fattori~$5ab^{2}-10abc-25abx+50acx$.
 \begin{enumeratea}
  \item Il fattore comune è~$5a$, quindi:
    \begin{itemize*}
    \item $5ab^{2}-10abc-25abx+50acx=5a\bigl(b^{2}-2bc-5bx+10cx\bigr)$;
    \end{itemize*}
  \item vediamo se è possibile scomporre il polinomio in parentesi con un raccoglimento parziale~$5a(\mmevid{ev_rosso}{b^{2}}-\mmevid{ev_rosso}{2bc}-\mmevid{ev_verde}{5bx}+\mmevid{ev_verde}{10cx})=5a\bigl[b(\mmevid{ev_blu}{b-2c})-5x(\mmevid{ev_blu}{b-2c})\bigr]=5a(b-2c)(b-5x)$.

%  \item vediamo se è possibile scomporre il polinomio in parentesi con un raccoglimento parziale~$5a(\underline{b^{2}}-\underline{2bc}
%     -\underline{\underline{5bx}}+\underline{\underline{10cx}})=5a\bigl[b(\underline{b-2c})-5x(\underline{b-2c})\bigr]=5a(b-2c)(b-5x)$.
 \end{enumeratea}
 \end{esempio}
\end{exrig}

\ovalbox{\risolvii \ref{ese:15.16}, \ref{ese:15.17}, \ref{ese:15.18}, \ref{ese:15.19}, \ref{ese:15.20}, \ref{ese:15.21}, \ref{ese:15.22}, \ref{ese:15.23}, \ref{ese:15.24},\ref{ese:15.25}, \ref{ese:15.26}}

\ovalbox{\ref{ese:15.27}, \ref{ese:15.28}}

\newpage
% (c) 2012 Claudio Carboncini - claudio.carboncini@gmail.com
% (c) 2012 Dimitrios Vrettos - d.vrettos@gmail.com
\section{Esercizi}
\subsection{Esercizi dei singoli paragrafi}
\subsubsection*{\thechapter.1 - Cosa vuol dire scomporre in fattori}

\begin{esercizio}
\label{ese:15.1}
Associa le espressioni a sinistra con i polinomi a destra.
  \begin{multicols}{2}
\begin{enumeratea}
\item $(a+2b)^{2}$;
\item $3ab^{2}(a^{2}-b)$;
\item $(2a+3b)(a-2b)$;
\item $(3a-b)(3a+b)$;
\item $(a+b)^{3}$;
\item $(a+b+c)^{2}$;
\item $2a^{2}-4ab+3ab-6b^{2}$;
\item $a^{2}+4ab+4b^{2}$;
\item $9a^{2}-b^{2}$;
\item $3a^{3}b^{2}-3ab^{3}$;
\item $a^{2}+b^{2}+c^{2}+2ab+2bc+2ac$;
\item $a^{3}+3a^{2}b+3ab^{2}+b^{3}$.
\end{enumeratea}
  \end{multicols}
\end{esercizio}

\subsubsection*{\thechapter.2 - Raccoglimento totale a fattore comune}

\begin{esercizio}[\Ast]
\label{ese:15.2}
Scomponi in fattori raccogliendo a fattore comune.
\begin{multicols}{2}
\begin{enumeratea}
 \item $ax+3a^{2}x-abx$;
 \item $15b^{2}+12bc+21abx+6ab^{2}$;
 \item $15x^{2}y-10xy+25x^{2}y^{2}$.
\end{enumeratea}
\end{multicols}
\end{esercizio}

\begin{esercizio}[\Ast]
\label{ese:15.3}
Scomponi in fattori raccogliendo a fattore comune.
\begin{multicols}{2}
\begin{enumeratea}
 \item $-12a^{8}b^{9}-6a^{3}b^{3}-15a^{4}b^{3}$;
 \item $2ab^{2}+2b^{2}c-2a^{2}b^{2}-2b^{2}c^{2}$;
 \item $2m^{7}+8m^{6}+8m^{5}$.
\end{enumeratea}
\end{multicols}
\end{esercizio}

\begin{esercizio}[\Ast]
\label{ese:15.4}
Scomponi in fattori raccogliendo a fattore comune.
\begin{multicols}{3}
\begin{enumeratea}
 \item $9x^{2}b+6xb+18xb^{2}$;
 \item $20a^{5}+15a^{7}+10a^{4}$;
 \item $x^{2}b-x^{5}-4x^{3}b^{2}$.
\end{enumeratea}
\end{multicols}
\end{esercizio}

\begin{esercizio}
\label{ese:15.5}
Scomponi in fattori raccogliendo a fattore comune.
\begin{multicols}{3}
\begin{enumeratea}
 \item $3xy+6x^{2}$;
 \item $b^{3}+\dfrac{1}{3}b$;
 \item $3xy-12y^{2}$;
 \item $x^{3}-ax^{2}$;
 \item $9a^{3}-6a^{2}$;
 \item $5x^{2}-15x$.
\end{enumeratea}
\end{multicols}
\end{esercizio}

\begin{esercizio}
\label{ese:15.6}
Scomponi in fattori raccogliendo a fattore comune.
\begin{multicols}{3}
\begin{enumeratea}
 \item $18x^{2}y-12y^{2}$;
 \item $4x^{2}y-x^{2}$;
 \item $5x^{3}-2x^{2}$;
 \item $-2x^{3}+2x$;
 \item $3a+3$;
 \item $-8x^{2}y^{3}-10x^{3}y^{2}$.
\end{enumeratea}
\end{multicols}
\end{esercizio}

\begin{esercizio}
\label{ese:15.7}
Scomponi in fattori raccogliendo a fattore comune.
\begin{multicols}{2}
\begin{enumeratea}
 \item $\dfrac{2}{3}a^{2}b-\dfrac{4}{3}a^{4}b^{3}-\dfrac{5}{9}a^{2}b^{2}$;
 \item $12a^{3}x^{5}-18ax^{6}-6a^{3}x^{4}+3a^{2}x^{4}$;
 \item $\dfrac{2}{3}a^{4}bc^{2}-4ab^{3}c^{2}+\dfrac{10}{3}abc^{2}$;
 \item $-{\dfrac{3}{5}}a^{4}bx+\dfrac{3}{2}ab^{4}x-2a^{3}b^{2}x$.
\end{enumeratea}
\end{multicols}
\end{esercizio}

%\newpage
\begin{esercizio}
\label{ese:15.8}
Scomponi in fattori raccogliendo a fattore comune.
\begin{multicols}{2}
\begin{enumeratea}
 \item $-{\dfrac{5}{2}}a^{3}b^{3}-\dfrac{5}{3}a^{4}b^{2}+\dfrac{5}{6}a^{3}b^{4}$;
 \item $91m^{5}n^{3}+117m^{3}n^{4}$;
 \item $\dfrac{2}{3}a^{2}x+\dfrac{5}{4}ax^{2}-\dfrac{5}{4}ax$;
 \item $-5a^{2}+10ab^{2}-15a$.
\end{enumeratea}
\end{multicols}
\end{esercizio}

\begin{esercizio}
\label{ese:15.9}
Scomponi in fattori raccogliendo a fattore comune.
\begin{multicols}{3}
\begin{enumeratea}
 \item $ab^{2}-a+a^{2}$;
 \item $2b^{6}+4b^{4}-b^{9}$;
 \item $2a^{2}b^{2}x-4a^{2}b$;
 \item $-a^{4}-a^{3}-a^{5}$;
 \item $-3a^{2}b^{2}+6ab^{2}-15b$;
 \item $a^{2}b-b+b^{2}$.
\end{enumeratea}
\end{multicols}
\end{esercizio}

\begin{esercizio}
\label{ese:15.10}
Scomponi in fattori raccogliendo a fattore comune.
\begin{multicols}{3}
\begin{enumeratea}
 \item $2b^{6}+4b^{4}-b^{9}$;
 \item $-5a^{4}-10a^{2}-30a$;
 \item $-a^{2}b^{2}-a^{3}b^{5}+b^{3}$;
 \item $-2x^{6}+4x^{5}-6x^{3}y^{9}$;
 \item $-2x^{2}z^{3}+4z^{5}-6x^{3}z^{3}$;
 \item $-{\dfrac{4}{9}}x+\dfrac{2}{3}x^{2}-\dfrac{1}{3}x^{3}$.
\end{enumeratea}
\end{multicols}
\end{esercizio}

\begin{esercizio}
\label{ese:15.11}
Scomponi in fattori raccogliendo a fattore comune.
\begin{multicols}{2}
\begin{enumeratea}
 \item $\dfrac{1}{2}a^{2}+\dfrac{1}{2}a$;
 \item $a^{n}+a^{n-1}+a^{n-2}$;
 \item $\dfrac{1}{3}ab^{3}+\dfrac{1}{6}a^{3}b^{2}$;
 \item $a^{n}+a^{2n}+a^{3n}$.
\end{enumeratea}
\end{multicols}
\end{esercizio}

\begin{esercizio}
\label{ese:15.12}
Scomponi in fattori raccogliendo a fattore comune.
\begin{multicols}{2}
\begin{enumeratea}
 \item $2x^{2n}-6x^{(n-1)}+4x^{(3n+1)}$;
 \item $a^{2}x^{n-1}-2a^{3}x^{n+1}+a^{4}x^{2n}$;
 \item $a(x+y)-b(x+y)$;
 \item $(x+y)^{3}-(x+y)^{2}$.
\end{enumeratea}
\end{multicols}
\end{esercizio}

\begin{esercizio}[\Ast]
\label{ese:15.13}
Scomponi in fattori raccogliendo a fattore comune.
 \begin{multicols}{2}
 \begin{enumeratea}
 \item $a^{n}+a^{n+1}+a^{n+2}$;
 \item $(a+2)^{3}-(a+2)^{2}-a-2$;
 \item $2a(x-2)+3x(x-2)^{2}-(x-2)^{2}$.
\end{enumeratea}
 \end{multicols}
\end{esercizio}

\begin{esercizio}[\Ast]
Scomponi in fattori raccogliendo a fattore comune.
\label{ese:15.14}
 \begin{multicols}{2}
 \begin{enumeratea}
 \item $x^{2}(a+b)^{3}+x^{3}(a+b)+x^{5}(a+b)^{2}$;
 \item $3(x+y)^{2}-6(x+y)+2x(x+y)$.
\end{enumeratea}
 \end{multicols}
\end{esercizio}

\begin{esercizio}
\label{ese:15.15}
Scomponi in fattori raccogliendo a fattore comune.
\begin{multicols}{2}
\begin{enumeratea}
 \item $5y^{3}(x-y)^{3}-3y^{2}(x-y)$;
 \item $5a(x+3y)-3(x+3y)$;
 \item $2x(x-1)-3a^{2}(x-1)$;
 \item $2(x-3y)-y(3y-x)$.
\end{enumeratea}
\end{multicols}
\end{esercizio}

\begin{esercizio}[\Ast]
Scomponi in fattori raccogliendo a fattore comune.
\label{ese:15.16}
\begin{multicols}{2}
 \begin{enumeratea}
 \item $3x^{2}(a+b)-2x^{3}(a+b)+5x^{5}(a+b)$;
 \item $(2x-y)^{2}-5x^{3}(2x-y)-3y(2x-y)^{3}$.
\end{enumeratea}
\end{multicols}
\end{esercizio}
%\newpage
\subsubsection*{\thechapter.3 - Raccoglimento parziale a fattore comune}

\begin{esercizio}[\Ast]
\label{ese:15.17}
Scomponi in fattori con il raccoglimento parziale a fattore comune, se possibile.
\begin{multicols}{3}
 \begin{enumeratea}
 \item $2x-2y+ax-ay$;
 \item $3ax-6a+x-2$;
 \item $ax+bx-ay-by$.
\end{enumeratea}
\end{multicols}
\end{esercizio}

\begin{esercizio}
\label{ese:15.18}
Scomponi in fattori con il raccoglimento parziale a fattore comune, se possibile.
\begin{multicols}{2}
\begin{enumeratea}
 \item $3ax-9a-x+3$;
 \item $ax^{3}+ax^{2}+bx+b$;
 \item $2ax-4a-x+2$;
 \item $b^{2}x+b^{2}y+2ax+2ay$.
\end{enumeratea}
\end{multicols}
\end{esercizio}

\begin{esercizio}[\Ast]
\label{ese:15.19}
Scomponi in fattori con il raccoglimento parziale a fattore comune, se possibile.
\begin{multicols}{3}
 \begin{enumeratea}
 \item $3x^{3}-3x^{2}+3x-3$;
 \item $x^{3}-x^{2}+x-1$;
 \item $ay+2x^{3}-2ax^{3}-y$.
\end{enumeratea}
\end{multicols}
\end{esercizio}

\begin{esercizio}
\label{ese:15.20}
Scomponi in fattori con il raccoglimento parziale a fattore comune, se possibile.
\begin{multicols}{2}
\begin{enumeratea}
 \item $-x^{3}+x^{2}+x-1$;
 \item $x^{3}+x^{2}-x-1$;
 \item $x^{3}-1-x+x^{2}$;
 \item $-x^{3}-x-1-x^{2}$.
\end{enumeratea}
\end{multicols}
\end{esercizio}

\begin{esercizio}
\label{ese:15.21}
Scomponi in fattori con il raccoglimento parziale a fattore comune, se possibile.
\begin{multicols}{2}
\begin{enumeratea}
 \item $x^{3}+x^{2}+x+1$;
 \item $b^{2}x-b^{2}y+2x-2y$;
 \item $b^{2}x-b^{2}y-2ax-2ay$;
 \item $xy+x+ay+a+by+b$.
\end{enumeratea}
\end{multicols}
\end{esercizio}

\begin{esercizio}
\label{ese:15.22}
Scomponi in fattori con il raccoglimento parziale a fattore comune, se possibile.
\begin{multicols}{2}
\begin{enumeratea}
 \item $3x+6+ax+2a+bx+2b$;
 \item $2x-2+bx-b+ax-a$;
 \item $2x-2+bx-b-ax+a$;
 \item $2x+2+bx-b-ax+a$.
\end{enumeratea}
\end{multicols}
\end{esercizio}

\begin{esercizio}
\label{ese:15.23}
Scomponi in fattori con il raccoglimento parziale a fattore comune, se possibile.
\begin{multicols}{2}
\begin{enumeratea}
 \item $2x-b+ax-a-2+bx$;
 \item $a^{3}+2a^{2}+a+2$;
 \item $a^{2}x+ax-a-1$;
 \item $3xy^{3}-6xy-ay^{2}+2a$.
\end{enumeratea}
\end{multicols}
\end{esercizio}

\begin{esercizio}
\label{ese:15.24}
Scomponi in fattori con il raccoglimento parziale a fattore comune, se possibile.
\begin{multicols}{2}
\begin{enumeratea}
 \item $a^{2}x^{3}+a^{2}x^{2}+a^{2}x-2x^{2}-2x-2$;
 \item $3x^{4}-3x^{3}+3x^{2}-3x$;
 \item $2ax-2a+abx-ab+a^{2}x-a^{2}$;
 \item $3x^{4}y^{4}-6x^{4}y^{2}-ax^{3}y^{3}+2ax^{3}y$.
\end{enumeratea}
\end{multicols}
\end{esercizio}

\begin{esercizio}
\label{ese:15.25}
Scomponi in fattori con il raccoglimento parziale a fattore comune, se possibile.
\begin{multicols}{2}
\begin{enumeratea}
 \item $b^{2}x-2bx+by-2y$;
 \item $\dfrac{2}{3}x^{3}-\dfrac{1}{3}x^{2}+2x-1$;
 \item $ax+bx+2x-a-b-2$;
 \item $3(x+y)^{2}+5x+5y$.
\end{enumeratea}
\end{multicols}
\end{esercizio}

%\newpage
\begin{esercizio}[\Ast]
\label{ese:15.26}
Scomponi in fattori con il raccoglimento parziale a fattore comune, se possibile.
\begin{multicols}{2}
\begin{enumeratea}
 \item $bx^{2}-bx+b+x^{2}-x+1$;
 \item $a^{3}-a^{2}b^{2}-ab+b^{3}$;
 \item $\dfrac{1}{5}a^{2}b+3ab^{2}-\dfrac{1}{3}a-5b$.
\end{enumeratea}
\end{multicols}
\end{esercizio}

\begin{esercizio}
\label{ese:15.27}
Scomponi in fattori con il raccoglimento parziale a fattore comune, se possibile.
\begin{multicols}{2}
\begin{enumeratea}
 \item $3x^{4}+9x^{2}-6x^{3}-18x$;
 \item $2a-a^{2}+8b-4ab$;
 \item $4x^{2}+3a+4xy-4ax-3y-3x$;
 \item $3x^{4}-3x^{3}+2x-2$.
\end{enumeratea}
\end{multicols}
\end{esercizio}

\begin{esercizio}[\Ast]
\label{ese:15.28}
Scomponi in fattori con il raccoglimento parziale a fattore comune, se possibile.
\begin{multicols}{2}
 \begin{enumeratea}
 \item $(a-2)(a-3)+ab-2b$;
 \item $\dfrac{1}{8}x^{3}-2xy^{2}+\dfrac{1}{2}yx^{2}-8y^{3}$;
 \item $ab-bx^{2}-\dfrac{2}{3}ax+\dfrac{2}{3}x^{3}$.
\end{enumeratea}
\end{multicols}
\end{esercizio}

\begin{esercizio}[\Ast]
\label{ese:15.29}
Scomponi in fattori con il raccoglimento parziale a fattore comune, se possibile.
\begin{enumeratea}
 \item $45x^{3}+15xy+75x^{2}y+21x^{2}y^{2}+7y^{3}+35xy^{3}$;
 \item $10x^3-12x^2-5xy+6y$;
 \item $6a^3+3a^2b-2ab^3-b^4$.
\end{enumeratea}
\end{esercizio}

\begin{esercizio}[\Ast]
\label{ese:15.30}
Scomponi in fattori raccogliendo prima a fattore comune totale e poi parziale.
\begin{multicols}{2}
 \begin{enumeratea}
 \item $a^{14}+4a^{10}-2a^{12}-8a^{8}$;
 \item $3x^{2}(x+y)^{2}+5x^{3}+5x^{2}y$;
 \item $ax^{3}y+ax^{2}y+axy+ay$.
\end{enumeratea}
\end{multicols}
\end{esercizio}

\begin{esercizio}
\label{ese:15.31}
Scomponi in fattori raccogliendo prima a fattore comune totale e poi parziale.
\begin{multicols}{2}
\begin{enumeratea}
 \item $b^{2}x+b^{2}y-2bx-2by$; %ex63
 \item $b^{2}x-2bx-2by+b^{2}y$;%ex65
 \item $2ab^{2}+2b^{2}c-2a^{2}b^{2}-2ab^{2}c$; %ex67
 \item $3ax+6a+a^{2}x+2a^{2}+abx+2ab$.%ex69
\end{enumeratea}
\end{multicols}
\end{esercizio}

\begin{esercizio}[\Ast]
\label{ese:15.32}
Scomponi in fattori raccogliendo prima a fattore comune totale e poi parziale.
\begin{enumeratea}
 \item $2^{11}x^{2}+2^{12}x+2^{15}x+2^{16}$;
 \item $6x^{2}+6xy-3x(x+y)-9x^{2}(x+y)^{2}$;
 \item $2x^{3}+2x^{2}-2ax^{2}-2ax$.

\end{enumeratea}
\end{esercizio}

\begin{esercizio}
\label{ese:15.33}
Scomponi in fattori raccogliendo prima a fattore comune totale e poi parziale.
\begin{multicols}{2}
\begin{enumeratea}
 \item $2bx^{2}+4bx-2x^{2}-4ax$;
 \item $x^{4}+x^{3}-x^{2}-x$;
 \item $15x(x+y)^{2}+5x^{2}+5xy$;
 \item $2a^{2}mx-2ma^{2}-2a^{2}x+2a^{2}$.
\end{enumeratea}
\end{multicols}
\end{esercizio}

\begin{esercizio}[\Ast]
\label{ese:15.34}
Scomponi in fattori raccogliendo prima a fattore comune totale e poi parziale.
\begin{enumeratea}
 \item $\dfrac{2}{3}ax^{3}-\dfrac{1}{3}ax^{2}+\dfrac{2}{3}ax-\dfrac{1}{3}a$;
 \item $\dfrac{7}{3}x^{2}-\dfrac{7}{3}xy+\dfrac{1}{9}x^{3}-\dfrac{1}{9}x^{2}y-\dfrac{5}{9}(x^{2}-xy)$;
 \item $2b(x+1)^{2}-2bax-2ba+4bx+4b$.
\end{enumeratea}
\end{esercizio}

\subsection{Risposte}

\paragraph{\thechapter.2}
a)~$ax(3a-b+1)$,\quad b)~$3b(7ax+2ab+5b+4c)$, \quad c)~$5xy(5xy+3x-2)$.

\paragraph{\thechapter.3}
a)~$-3a^{3}b^{3}\left(4a^{5}b^{6}+5a+2\right)$,\quad b)~$2b^{2}(a+c-a^{2}-c^{2})$, \quad c)~$2m^{5}\left(m+2\right)^{2}$.

\paragraph{\thechapter.4}
a)~$3bx(3x+6b+2)$,\quad b)~$5a^{4}\left(3a^{3}+4a+2\right)$, \quad c)~$-x^{2}\left(x^{3}+4b^{2}x-b\right)$.

\paragraph{\thechapter.13}
a)~$a^{n}(1+a+a^{2})$,\quad b)~$(a+2)\left(a^{2}+3a+1\right)$, \quad c)~$(x-2)\left(3x^2-7x+2a+2\right)$.

\paragraph{\thechapter.14}
a)~$x^{2}(a+b)(ax^{3}+bx^{3}+x+a^{2}+2ab+b^{2})$,\quad b)~$(x+y)\left(5x+3y-6\right)$.

\paragraph{\thechapter.16}
a)~$x^{2}(a+b)(5x^{3}-2x+3)$,\quad b)~$(2x-y)\left(2x-y-5x^3-12x^2y+12xy^2-3y^3\right)$.

\paragraph{\thechapter.17}
a)~$(x-y)(2+a)$,\quad b)~$(x-2)(3a+1)$, \quad c)~$(a+b)(x-y)$.

\paragraph{\thechapter.19}
a)~$(3x-3)\left(x^2+1\right)$,\quad b)~$(x-1)\left(x^{2}+1\right)$, \quad c)~$(a-1)\left(y-2x^{3}\right)$.

\paragraph{\thechapter.26}
a)~$(b+1)(x^{2}-x+1)$,\quad b)~$\left(a^{2}-b\right)\left(a-b^{2}\right)$, \quad c)~$\left(\frac{3}{5}ab-1\right)\left(\frac{1}{3}a+5b\right)$.

\paragraph{\thechapter.28}
a)~$(a-2)(a-3+b)$,\quad b)~$(x+4y)\left(\frac{1}{8}x^2-2y^2\right)$, \quad c)~$\left(a-x^2\right)\left(b-\frac{2}{3}x\right)$.

\paragraph{\thechapter.29}
a)~$\left(15x+7y^{2}\right)\left(3x^{2}+y+5xy\right)$,\quad b)~$\left(2x^2-y\right)(5x-6)$, \quad c)~$(3a^2-b^3)(2a+b)$.

\paragraph{\thechapter.30}
a)~$a^{8}\left(a^{2}-2\right)\left(a^{4}+4\right)$,\quad b)~$x^{2}(x+y)(3x+3y+5)$, \quad c)~$ay(x+1)(x^{2}+1)$.

\paragraph{\thechapter.32}
a)~$2^{11}(x+2)(x+16)$,\quad b)~$-3x(x+y)\left(3x^2+3xy-1\right)$, \quad c)~$2x(x+1)(x-a)$.

\paragraph{\thechapter.34}
a)~$\frac{1}{3}a(x^{2}+1)(2x-1)$,\quad b)~$\frac{1}{9}x(x-y)(16+x)$, \quad c)~$2b(x+1)(x-a+3)$.


\cleardoublepage
